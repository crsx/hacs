\documentclass[letterpaper]{article}
%% $Id: setup.tex,v 1.11 2013/12/05 04:10:03 krisrose Exp $

%% Format.
\usepackage[utf8]{inputenc}
\usepackage{comment}
\usepackage{amsmath,amssymb,stmaryrd,textcomp}
\allowdisplaybreaks %style
\usepackage[normalem]{ulem}
\usepackage{pgfpages}
\usepackage{utf8math}
\usepackage{url}
\usepackage{xspace}
\usepackage{fancyvrb}
\usepackage{xcolor}
\usepackage{graphicx}
\usepackage{hyperref}
\usepackage{rcs}

%% Macros.
\newcommand{\HAX}{\text{HACS}\xspace}
\newcommand{\CRSX}{\text{CRSX}\xspace}
\newcommand{\ie}{\textit{i.e.}\xspace}
\newcommand{\eg}{\textit{e.g.}\xspace}
\newcommand{\etc}{\textit{etc.}\xspace}
\newcommand{\aka}{\textit{aka.}\xspace}
\newcommand{\TBD}[1][]{\textit{To Be Done…#1}\marginpar{\bf\color{red} TBD}}

%% Misc.
\def\Bigstrut{\ensuremath{\vphantom{\Bigm|}}}
\newcommand\caret{\mathbin{\char`\^}}

%% Units
\usepackage{amsthm}
\swapnumbers %style
\newtheorem{theorem}{Theorem}[section]
\newtheorem{lemma}[theorem]{Lemma}
\theoremstyle{definition}
\newtheorem{definition}[theorem]{Definition}
\newtheorem{proposition}[theorem]{Proposition}
\newtheorem{example}[theorem]{Example}
\newtheorem{practice}[theorem]{Practice}
\newtheorem{notation}[theorem]{Notation}
\newtheorem{manual}[theorem]{Manual}
%\newenvironment{manual}{\small\begin{manual-body}}{\end{manual-body}}
\newtheorem{invariant}[theorem]{Invariant}
\newtheorem{requirements}[theorem]{Requirements}
\newtheorem{limitations}[theorem]{Limitations}
\newtheorem{commands}[theorem]{Commands}
\newtheorem{error}[theorem]{Error}
\newtheorem{note}[theorem]{Note}
\newtheorem{plan}[theorem]{Plan}
%\theoremstyle{remark}
\newtheorem{remark}[theorem]{Remark}

%% Old code listings.
\usepackage{alltt}
\usepackage{fancyvrb}
\CustomVerbatimEnvironment{code}{Verbatim}{tabsize=4,fontsize=\small,numberblanklines=false,xleftmargin=\parindent}
\CustomVerbatimCommand{\inputcode}{VerbatimInput}{tabsize=4,fontsize=\footnotesize,numbers=left,numberblanklines=false,xleftmargin=\parindent}
%\DefineShortVerb{\"} % makes " work in verbatim even when \active


%% New code listings...
\usepackage{listings}
\lstset{basicstyle=\ttfamily}

%% HACS.
\def\Identifier#1{#1}
\def\Keyword#1{\textsf{\color{gray}#1}}

\lstdefinelanguage{HACS}{basicstyle=\fontfamily{ccr}\selectfont,%
  extendedchars=true,inputencoding=utf8,sensitive,%
  identifierstyle=\Identifier,%
  string=[d]{"},morestring=[d]{'},upquote=true,showstringspaces=false,%
  comment=[l]{//},morecomment=[s]{/*}{*/},commentstyle=\color{blue}\rm,%
  keywordstyle=\Keyword,%
  keywords={attribute,data,fragment,import,main,module,nested,property,rule,scheme,simplify,sort,%
    space,static,sugar,symbol,tag,token,default,error,binds,as},%
  literate=%
    {⟦}{{\ensuremath{\llbracket}}}1 {⟧}{{\ensuremath{\rrbracket}}}1 {⟨}{{\ensuremath{\langle}}}1 {⟩}{{\ensuremath{\rangle}}}1 %
    {→}{{\ensuremath{\rightarrow}}}2 {↑}{{\ensuremath{\uparrow}}}1 {↓}{{\ensuremath{\downarrow}}}1 {¬}{{\ensuremath{\lnot}}}2 {¶}{{\P}}1 %
    {ε}{{\ensuremath{\varepsilon}}}1 {…}{{\ensuremath{\dots}}}2 {∧}{{\ensuremath{\!\wedge}}}2 {∨}{{\ensuremath{\!\vee}}}2 {↦}{{\ensuremath{\mapsto}}}2 %
}[comments,strings,keywords]
%%
\lstnewenvironment{hacs}[1][]{\lstset{language=HACS,basewidth={.53em,.43em},columns=flexible,numberstyle=\scriptsize,numberblanklines=false,firstnumber=auto,xrightmargin=1pc,#1}\upshape}{}
\newcommand\inputhacs[2][]{\lstinputlisting[language=HACS,basewidth={.53em,.43em},columns=flexible,numberstyle=\scriptstyle,numberblanklines=false,xrightmargin=1pc,#1]{#2}}
\newcommand\hacsc{\lstinline[language=HACS]}
\lstMakeShortInline[language=HACS,columns=fullflexible]"


%%% \usepackage{listings}
%%% 
%%% %% HACS.
%%% \lstdefinelanguage{HACS}{basicstyle=\rm,%
%%%   numberstyle=\scriptsize,
%%%   numbersep=7pt,
%%%   numberblanklines=false,
%%%   extendedchars=true,inputencoding=utf8,sensitive,%
%%%   identifierstyle=\textit,%
%%%   string=[d]{"},morestring=[m]{'},upquote=true,stringstyle=\ttfamily,%showstringspaces=true,%
%%%   comment=[l]{//},morecomment=[s]{/*}{*/},commentstyle=\color{gray}\upshape,%
%%%   keywordstyle=\color{black}\bfseries,%
%%%   keywords={attribute,data,default,fragment,import,main,module,nested,property,rule,scheme,simplify,sort,space,static,sugar,symbol,tag,token},%
%%%   ndkeywordstyle={\textup},%
%%%   literate=%
%%%     {⟦}{{\ensuremath{\llbracket}}}1 {⟧}{{\ensuremath{\rrbracket}}}1 {⟨}{{\ensuremath{\langle}}}1 {⟩}{{\ensuremath{\rangle}}}1 %
%%%     {→}{\ensuremath{\rightarrow}}2 {↑}{\ensuremath{\uparrow}}1 {↓}{\ensuremath{\downarrow}}1 {ε}{\ensuremath{\varepsilon}}1 %
%%%     {…}{\ensuremath{\dots}}2 {¬}{\ensuremath{\lnot}}1 {∧}{\ensuremath{\wedge}}2 {∨}{\ensuremath{\vee}}2 {¶}{\P}1 {↦}{\ensuremath{\mapsto}}2 %
%%% }[comments,strings,keywords]
%%% %%
%%% \lstnewenvironment{hacs}[1][]{\lstset{language=HACS,basewidth={.55em,.4em},columns=flexible,#1}\upshape}{}
%%% \newcommand\inputhacs[2][]{\lstinputlisting[language=HACS,basewidth={.55em,.4em},columns=flexible,#1]{#2}}
%%% \newcommand\hacsc{\lstinline[language=HACS]}
%%% \lstMakeShortInline[language=HACS,columns=fullflexible]"
%%% 
%%% %% Tiger.
%%% \lstdefinelanguage{Tiger}{basicstyle=\normalsize,%
%%%   extendedchars=true,inputencoding=utf8,sensitive,%
%%%   identifierstyle=\textit,%
%%%   string=[d]{"},upquote=true,stringstyle=\ttfamily,showstringspaces=true,%
%%%   comment=[s]{/*}{*/},commentstyle=\color{grey}\rm,%
%%%   keywordstyle=\color{black}\bfseries,%
%%%   keywords={type,array,of,var,function,nil,let,in,end}%
%%% }[comments,strings,keywords]
%%% %%
%%% \lstnewenvironment{tiger}[1][]{\lstset{language=Tiger,basewidth={.5em,.4em},columns=flexible,#1}\upshape}{}
%%% \newcommand\inputtiger[2][]{\lstinputlisting[language=Tiger,basewidth={.5em,.4em},columns=flexible,#1]{#2}}
%%% \newcommand\tigerc{\lstinline[language=Tiger]}

%% Identifiers
\newcommand{\set}[1]{\textit{#1}}
\newcommand{\op}[1]{\ensuremath{\operatorname{#1}}}

\usepackage[all]{xy}
\SelectTips{eu}{}

\makeatletter
\def\compacttableofcontents{{\if@twocolumn\else \quotation\small \fi
    \noindent\textbf{Contents:}~%
    \def\protect{\relax}%
    \def\contentsline##1{\csname contentslineA##1\endcsname}%
    \def\numberline##1{\DN@{##1}\ifx\next@\empty\else##1.~\fi}%
    \def\fresh##1{\relax \let\fresh=\nextfresh}%
    \def\nextfresh##1{\relax##1 }%
    \def\contentslineAsection##1##2##3{\fresh,##1~(##2)\ignorespaces}%
    \def\contentslineAsubsection##1##2##3{\ignorespaces}%
    \@starttoc{toc}.\endquotation}}
\makeatother


%% Style.
\bibliographystyle{plain}
\usepackage{cite}

%% Topmatter.
\title{ Higher-order Attribute Contraction Schemes }
\author{ Kristoffer H. Rose \\
 Two Sigma Investments, LLC \& Courant, New York University }

\begin{document}
\maketitle

\begin{abstract}
  Higher-order Attribute Contraction Schemes---or \HAX---is a practical formal language for
  specifying executable compilers and other program manipulation tools. \HAX achieves this by
  allowing programming directly with source code constructs that are very close to the standard
  formal notations used in compiler teaching and research, and by combining these with native
  support for variable scope management that exploits the full power of the underlying higher-order
  rewriting engine.  In this paper we present the special notations that \HAX provides to make
  compiler writers more productive, and we outline how they translate into the underlying rewriting
  platform.
\end{abstract}

%------------------------------------------------------------------------

\section{Introduction}\label{sec:intro}

Many systems for writing programs that manipulate other programs, so-called \emph{meta-programming},
have emerged over the years, ranging from generic specification languages, where the goal is to not
define how but only declare the semantics of the program manipulation, all the way to tools that
support specific aspects of program execution or compiler generation (we comment on several in the
conclusion).

One such system was the CRSX system~\cite{Rose:1996} developed for industrial use at IBM Research by
a team led by the author~\cite{Rose:hor2007,Rose:hor2010,Rose:rta2011,crsx} .  CRSX is a language
based on \emph{higher-order rewriting}~\cite{Jouannaud:klop2005} combined with \emph{higher-order
  abstract syntax} (HOAS) \cite{PfenningElliot:pldi1988} and extensions for handling environments
and using pluggable parsers.  The programming of the IBM Data\-Power XQuery
compiler~\cite{dp60:ibm2013} using CRSX proved that the approach can drastically reduce the
development time of a compiler (the cited XQuery compiler was estimated to be developed in a quarter
of the traditional development time) as well as resulting in a rather more compact and high level
source program.

However, the CRSX notation, based on combinatory reduction systems~\cite{Klop:1980,Klop+:tcs1993},
which combines λ calculus~\cite{Church:1941,Barendregt:1984} and term rewriting
systems~\cite{Klop:1992:Handbook}, has proven to be unwieldy for several reasons, first of all by
being quite different from standard notations used in compiler construction reference
works~\cite{Aho+:2006}. Adding a polymorphic type system to CRSX (effectively changing them to
\emph{contraction systems}~\cite{Aczel:1978} and be similar to Inductive-type-systems
\cite{BlanquiJouannaudOkada:tcs2002}) helped but did not make the system easy enough to, for
example, teach compiler programming.

\HAX stands for \emph{Higher-order Attribute Contraction Schemes}, and is an attempt to remedy this
by providing a front-end for CRSX that allows the use of standard notations and concepts of (formal)
programming language descriptions to directly program compilers and other systems for manipulating
code. As the name suggests, the notation combines HOAS with concepts from attribute
grammars~\cite{Knuth:mst1968} and other common compiler specification
notations~\cite{Aho+:2006}. \HAX has been successfully used to teach the graduate computer science
compiler construction class at New York University~\cite{RoseRose:cims2015}.

In this paper we illustrate how \HAX is used in practice through an annotated example, and for each
stage we explain how the involved \HAX notations are translated into the underlying higher-order
rewriting formalisms.
Section~\ref{sec:parse} explains how to write parsers, %
Section~\ref{sec:analysis} explains mechanisms for semantic analysis, %
Section~\ref{sec:cogen} explains mechanisms for code generation, %
Section~\ref{sec:run} completes and demonstrates the sample compiler, %
Section~\ref{sec:further} discusses aspects of \HAX that we have not discussed and that need further
development, %
and finally Section~\ref{sec:conc} concludes and compares to related work.

%------------------------------------------------------------------------

\section{Parser}
\label{sec:parse}

Parser generators have been automated for more than fourty years~\cite{GruneJacobs:2008}, and \HAX
follows this tradition and uses a generic LL(1) parser generation mechanism on top of
JavaCC~\cite{JavaCC}, with four extensions that are by now pretty standard in parser generators:
%%
\begin{itemize}

\item \emph{Precedence} of operators is handled automatically by allowing ``@\textit{precedence}''
  marks on individual productions and nonterminal references.

\item \emph{Direct left recursion} is eliminated using the standard technique~\cite{Aho+:2006}.

\item \emph{Left factoring} is performed to permit common prefixes of alternative productions.

\item \emph{Syntactic sugar} allows notation that serves to disambiguate or provide convenient forms
  of the input but is removed when constructing the output tree.

\end{itemize}
%%
In addition, \HAX supports a variant of HOAS for safe handling of scoped and global variable
symbols.
%%
All this is best illustrated with an example. We have chosen the most canonical example we could
think of as our example language, namely a simply typed λ calculus, as it exhibits the aspects of
\HAX that we wish to discuss.

\begin{figure}[t]\small
  \begin{hacs}[texcl,xleftmargin=2em,numbers=left,firstnumber=3]
// Tokens.
token ID | [a-z] [a-z0-9_]*  ;
token NUM | [0-9]+ ([.] [0-9]+)? ([Ee] [-+]? [0-9]+)? ;
token ADDOP | [-+] ;

// Variables.
sort V  | symbol ⟦⟨ID⟩⟧ ;

// Simple types (with numbers).
sort T  | ⟦⟨T@1⟩ → ⟨T⟩⟧  | ⟦num⟧@1   | sugar ⟦(⟨T#⟩)⟧@1 →   T# ;

// Simply typed λ terms (with numeric operations and tests).
main sort L
| ⟦ λ ⟨V binds x⟩ : ⟨T⟩ . ⟨L[x as L]⟩ ⟧
| ⟦ ⟨L@2⟩ ? ⟨L⟩ : ⟨L@1⟩ ⟧@1
| ⟦ ⟨L@2⟩ ⟨ADDOP⟩ ⟨L@3⟩ ⟧@2
| ⟦ ⟨L@3⟩ ⟨L@4⟩ ⟧@3
| ⟦ ⟨V⟩ ⟧@4
| ⟦ ⟨NUM⟩ ⟧@4
| sugar ⟦ ( ⟨L#⟩ ) ⟧@4  →  L#
;
  \end{hacs}
  \caption{\HAX input grammar for simply typed λ calculus.}
  \label{fig:parser}
\end{figure}

\begin{example}[input grammar]
  %%
  Our example language is a simply typed λ calculus with numbers, additive operators, and a ternary
  ``positive number'' test operator.  The \HAX specification of the grammar for our example language
  is given in Figure~\ref{fig:parser}. Notice the following:
  %%
  \begin{enumerate}

  \item Tokens are specified with standard regular expressions~\cite{man:regex} in lines 3--6 (as
    well as ``words'' used in the grammar productions) .

  \item Grammars are specified by declaring the ``sorts'' that correspond to node types to be used
    in the abstract syntax tree (AST); for all declarations each alternative is introduced by a
    vertical bar~"|".

  \item The special body form "symbol ⟦⟨ID⟩⟧" in line 9 makes the "V" sort special, allowing
    automatic management of scoping for variables (more on this below).

  \item In general, all concrete syntax is enclosed in "⟦…⟧", which can be seen as syntactic
    ``quasi-quotes''~\cite{Quine:1940} with "⟨…⟩" supplying the corresponding ``unquotes.'' (The
    notation may seem overly symbol-heavy for parsers but we shall see how it generalizes nicely in
    the subsequent stages.)

  \item The "sort T" declaration in line 12 illustrates the use of precedence annotations and
    syntactic sugar. We first define the usual right-recursive λ calculus function type notation,
    and then allow parenthesis for explicit grouping, which is then discarded by the parser.  Note
    that there are two uses of $→$: the first (inside $⟦…⟧$) is interpreted literally as the
    function type arrow that is part of the syntax of types. The second is part of the "sugar"
    declaration that identifies what part of the syntax should be preserved (the "T#" part; the "#"
    marks the "T" part as copyable).

  \item The "sort L" declarations in lines 15--23 define standard syntax for simply typed λ calculus
    with arithmetic. The only added notation is the HOAS binding in line~16: we here declare the
    usual rule that each λ-bound variable is scoped over the λ-body. This is achieved by two
    annotations: "binds x" on the "V" reference establishes that the "V" is a binder symbol, and the
    "[x as L]" annotation on the "L" λ-body declares this to be the scope of it and that the
    bound occurrences of the "V" token inside the "L" subterm occur as sort "L", consistent with the
    syntax in line~20. (The "x" is merely a placeholder to link the two, in case there are multiple
    binders in a single production.)

  \item The "@"-precedence markers reveal that the abstraction in line 16 and ternary operator in
    line 17 are right recursive, whereas additive expressions in line 18 and application in line 19
    are left recursive.

  \end{enumerate}
  %%
  \HAX input specifications are Unicode (UTF-8) files, and the manual gives the encodings of all
  special characters used (along with instructions for entering them conveniently).
  %%
\end{example}

For generating the user's parser, the implementation consists of a translation into a
JavaCC~\cite{JavaCC} grammar that generates the abstract syntax tree (with sugar and precedence
information removed) using standard techniques~\cite{Aho+:2006}. (To check the internal consistency
of the specification, \HAX also generates two further internal parsers, which we discuss in the next
section.)

%------------------------------------------------------------------------

\section{Analysis \& Rewriting}
\label{sec:analysis}

\HAX supports the specification of analysis and rewriting of HOAS~\cite{PfenningElliot:pldi1988}
abstract syntax trees as defined in the previous section using a variation of higher-order
rewriting~\cite{Jouannaud:klop2005} extended with special support for attribute grammars and
recursive translation schemes.  Specifically, the programmer has access to
%%
\begin{itemize}

\item \emph{syntax-directed} patterns over the defined syntax;

\item rules that \emph{synthesize attributes};

\item rules that \emph{recursively pass parameters}, both usual positional and implicit named
  parameters (that also cover the rôle of \emph{inherited attributes});

\item there is special notation for \emph{immutable symbol table management}; and

\item special care is taken to allow natural use of HOAS with a corresponding \emph{polymorphic sort
    system}.

\end{itemize}

\begin{figure}[p]\small
  \begin{hacs}[texcl,xleftmargin=2em,numbers=left,firstnumber=25]
attribute ↑t(T); // assigned type
sort L | ↑t ; // type assigned to every (sub)term

// Scheme to assign types.
sort L | scheme AddTypes(L) ;
{
  AddTypes(#L) → WaitForType(Lte(#L)) ;

  // First pass: Distribute declared types to all variables.
  attribute ↓te{V:T}; // type environment
  sort L | scheme Lte(L) ↓te ; // carrier scheme

  Lte(⟦ λ x : ⟨T#tx⟩ . ⟨L#L[⟦x⟧]⟩ ⟧)
     →   ⟦ λ v : ⟨T#tx⟩ . ⟨L Lte(#L[⟦v⟧]) ↓te{⟦v⟧ : #tx}⟩ ⟧ ;

  Lte(⟦ ⟨L#1⟩ ? ⟨L#2⟩ : ⟨L#3⟩ ⟧)
     →   ⟦ ⟨L Lte(#1)⟩ ? ⟨L Lte(#2)⟩ : ⟨L Lte(#3)⟩ ⟧ ;
  Lte(⟦ ⟨L#1⟩ ⟨ADDOP#op⟩ ⟨L#2⟩ ⟧)
     →   ⟦ ⟨L Lte(#1)⟩ ⟨ADDOP#op⟩ ⟨L Lte(#2)⟩ ⟧ ;
  Lte(⟦ ⟨L#1⟩ ⟨L#2⟩ ⟧)		→   ⟦ ⟨L Lte(#1)⟩ ⟨L Lte(#2)⟩ ⟧ ;
  Lte(⟦ ⟨NUM#n⟩ ⟧)		→   ⟦ ⟨NUM#n⟩ ⟧ ;

  Lte(⟦ v ⟧) ↓te{⟦v⟧ : #t}	→   ⟦ v ⟧ ↑t(#t) ;
  Lte(⟦ v ⟧) ↓te{¬⟦v⟧}		→   error⟦Undeclared variable ⟨v⟩.⟧ ;

  // Second pass: populate all types.
  sort L | scheme WaitForType(L) ;
  WaitForType(#L ↑t(#t)) →     #L ;

  // Rules to synthesize ↑t (after variables have been assigned).
  ⟦λ x : ⟨T#tx⟩ . ⟨L#L[⟦x⟧] ↑t(#tL)⟩⟧		↑t(⟦ ⟨T#tx⟩ → ⟨T#tL⟩ ⟧) ;
  ⟦⟨L#1↑t(#t1)⟩ ⟨L#2 ↑t(#t2)⟩⟧			↑t(ApType(#t1, #t2)) ;
  ⟦⟨L#1↑t(#t1)⟩?⟨L#2↑t(#t2)⟩:⟨L#3↑t(#t3)⟩⟧ ↑t(IfType(#t1,#t2,#t3)) ;
  ⟦⟨L#1↑t(#t1)⟩ ⟨ADDOP#op⟩ ⟨L#2 ↑t(#t2)⟩⟧ ↑t(OpType(#t1, #t2)) ;
  ⟦⟨NUM#n⟩⟧	↑t(⟦num⟧) ;

  sort T | ApType(T, T) ;
  ApType(⟦ ⟨T#ta⟩ → ⟨T#tr⟩ ⟧, #ta) →     #tb ;
  default ApType(#t1, #t2)→  error⟦Function not applied to proper argument⟧ ;

  sort T | IfType(T, T, T) ;
  IfType(⟦num⟧, #t, #t) →   #t ;
  default IfType(#t1, #t2, #t3)→  error⟦Bad test or divergent case types⟧ ;

  sort T | OpType(T, T) ;
  OpType(⟦num⟧, ⟦num⟧) →     ⟦num⟧ ;
  default OpType(#t1, #t2)→  error⟦Numeric operator on non-numeric types⟧ ;
}
  \end{hacs}
  \caption{Type checker for typed λ calculus.}
  \label{fig:check}
\end{figure}

\begin{example}[analysis]
  Figure~\ref{fig:check} demonstrates how a standard type checker for our λ calculus is specified in
  \HAX.
  %%
  \begin{enumerate}

  \item Line 25 declares the "t" (type) attribute as \emph{synthesized}, indicated by the "↑", and
    with values of the sort~"T".

  \item Line 26 declares that every "L" (λ term) can have a synthesized "t" attribute.

  \item Line 29 declares the type checking scheme, "AddTypes", to take one argument λ term and
    return another. (We have opted for the simpler variant that any type error immediately reports a
    fatal error instead of doing the more realistic thing of synthesizing a proper result state with
    error messages; our example also ignores reporting line numbers.)

  \item The rest of the fragment is enclosed in "{}"s, localizing the following declarations.

  \item Line 31 gives the rule for the "AddTypes" scheme, which will be done in \emph{two passes},
    expressed as two nested calls. The first pass is represented by the inner call, "Lte", which
    will serve to annotate all variables occurrences with their declared type. The second pass, the
    outer call, is "WaitForType", which is responsible for making sure everything else is type
    annotated.

  \item Lines 33--48 detail the first pass. It starts with line 34, which declares the "te"
    inherited ($↓$) attribute as a map from "V" to "T" values. As mentioned above, \HAX requires
    that inherited attributes are realized as named parameters, and this association is declared in
    line 35, where we state that the scheme "Lte" takes an "L" parameter and "te" map.

  \item "Lte" is a \emph{syntax-directed} scheme, which we realize by giving a \emph{rule} for each
    form of the "L" argument abstract syntax. A rule consists of a \emph{pattern}, an $→$, and a
    \emph{replacement}, much like one of the cases of a case-defined function in a modern functional
    language.

  \item Line 40, has the pattern for the ternary case of "Lte": a copy of the grammar production
    without precedence markers but where each non-terminal subtree has been named with a
    ``meta-variable'' like "#1", which can the be used to refer to the subtree in the replacement.

  \item Line 41 has (the "→" and) the replacement for the ternary operator. In this case we just
    return the same ternary operation, except each of the fragments have been replaced with a
    recursive call of the scheme, \eg, the "⟨L#1⟩" part of the pattern is replaced with %
    "⟨L Lte(#1)⟩" in the replacement, which specifies that the "L" fragment is computed by a recursive
    call.

  \item Line 41 also illustrates another point: \emph{there is no mention of the inherited
      attribute} (or \emph{named parameter}). The rule is that each inherited attribute passed on
    the defined scheme is implicitly copied to all recursive calls of schemes carrying the same
    inherited attribute.

  \item Lines 42--45 have similar rules for the other syntax cases, except variables and
    abstractions.

  \item Lines 47--48 have the rules for variables. First notice that the variable is \emph{not}
    wrapped as "⟨V#v⟩": this is another consequence of "V" being declared a "symbol" sort back in
    line~7: for symbols, \HAX allows using actual parsed instances of the variable as
    meta-variables.

  \item Lines 47--48 are also special because the two otherwise identical patterns are each equipped
    with a \emph{map constraint}: line 47 will only be applied when the matched variable is mapped
    by the "te" inherited attribute/named parameter, and we can match the type it maps to to "#t"
    (which is always possible). Line 48 only matches when there is no map for "v" in the map. Note
    that since "v" is parsed, it is \emph{always} specified inside "⟦…⟧".

  \item The replacement in line 47 gives the same variable occurrence but now with an associated
    \emph{synthesized} ("↑") attribute "t", which was the purpose of the pass.

  \item Finally, we can explain the first "Lte" rule in lines 37--38, which involves HOAS
    matching. The pattern in line 37 clearly matches a λ abstraction with no further constraints:
    any (parsed) variable "x", any type "#tx", and any body expression with "x"s bound inside,
    written "#L[⟦x⟧]" (since "x" must be parsed).

  \item In the replacement in line 38 we construct a new λ abstraction, except it has different
    components:
    \begin{itemize}
    \item We refer to the parsed variable symbol as "v".
    \item The type component is the same.
    \item The "L" component it obtained by a recursive call, as in the other cases.
    \item The main term to recurse over is "#L[⟦v⟧]", which is just the original body modified to
      use the "v" parsed variable.
    \item The recursive call has an \emph{extended} "te" mapping, where the binding "⟦v⟧ : #tx" has
      been added.
    \end{itemize}

  \item Lines 50--71 define the subsequent synthesis of types on all nodes. Line 51 defines the sort
    of the top scheme, with the rule in line 52 defining that when applied it just waits for a "#L"
    subterm where the synthesized ("↑") attribute "t" has a value, and then just returns the
    subterm.

  \item A pattern like the one in line 52 will induce \HAX to search for a way to synthesize
    "t". This is provided by the \emph{synthesis rules} in lines 54--59. These do \emph{not} have an
    ``$→$'' in them but just a data argument pattern followed by "↑t" with a value to
    synthesize. The patterns are just like the previous case: one for each production \emph{except}
    variable occurrences, as these are assumed to have already been equipped with a synthesized "t"
    value.  For example, a numeric token synthesizes a numeric type in line 59 and the λ abstraction
    synthesizes the right function type in line~55.

  \item Lines 56--58 all invoke \emph{helper schemes} in lines 61--71 to synthesize a type. These
    each have one good case and a ``fall-back'' case, marked "default". Line 66, for example,
    encodes that test for positive must have a numeric type and two case types that are the same
    type to succeed, otherwise it fails; the other helpers should be equally self-explanatory.

  \end{enumerate}
\end{example}

To implement rule systems such as the above, \HAX uses the following techniques:
%%
\begin{itemize}

\item To check that the user uses sorts and attributes correctly, \HAX builds two special parsers
  based on the user's specification:
  \begin{itemize}

  \item A variant of the \HAX parser itself tweaked to understand what the sort is for every place
    where "⟦…⟧" can occur and restricted to only accept declared semantic constructors and
    attributes in their proper locations.

  \item A version of the user syntax parser extended so it can parse all the contents of "⟦…⟧"
    fragments for each sort.

  \end{itemize}
  In practice this means that \emph{sort checking is done by the parser}, which is necessary when
  custom syntax is involved. (A similar approach is used by the Rascal
  language~\cite{Bos+:eptcs2011}.)

\item When a pattern in a rule requires the value of a particular synthesized attribute for a
  subterm then that kicks off a recursive pass of the subterm. The rule in line 33 of
  Figure~\ref{fig:check}, for example, is transformed internally to a pair of platform rules similar
  to
  \begin{hacs}
      WaitForType(#L)  →    WaitForType2(Needs_t(#L)) ;
      WaitForType2(#L{"t"=#t}) →       #L ;
  \end{hacs}
  where the second rule is now guaranteed to be applicable once the "Needs_t" traversal finishes and
  has populated the ``annotation field'' "t" of the subtree.  It is worth noting that this only
  works efficiently because the underlying CRSX platform optimizes the reduction strategy to have
  good locality, so if multiple passes are requested, and show up as nested ``"Needs_"$a$''
  wrappers, then these are evaluated together, effectively using lazy evaluation to avoid multiple
  traversals of the argument subtree. Multiple passes are only required when a wrapper like
  "WaitForType2" with an explicit dependency is inserted.

\item Every recursive scheme is internally carrying a record with all inherited attributes, and this
  record is populated for every rule with the values from the context call, which is what gives the
  implicit passing of inherited attributes. Environment constraints (in the pattern) are just
  conditionals on the incoming value, and environment extensions (in the replacement) are just
  wrapped around the value passed on.

\end{itemize}
%%
While the above transformations are rather naïve, they tend to work very well when the underlying
formalism is pure term rewriting: because there is no fixed evaluation strategy, the CRSX system can
(and does) decide to use more efficient strategies than the usual innermost (call-by-value)
mechanism. It is also essential that the formalism is proper higher-order rewriting, because this
permits evaluation under binders, something that HOAS implementations usually do not permit.

%------------------------------------------------------------------------

\section{Code Generation}
\label{sec:cogen}

Next we show how \HAX permits code generation using the same mechanisms as above. The trick is to
define the target language in the same way as the source and any intermediate languages. Symbol
management can be used to allocate labels and other internal names safely.

\begin{figure}[p]%[t]\scriptsize
  \begin{hacs}[texcl,xleftmargin=2em,numbers=left,firstnumber=74]
  // Generate Java types.

  sort JT | ⟦Fun < ⟨JT⟩, ⟨JT⟩ > ⟧    | ⟦Double⟧ ;
  | scheme T2JT(T) ;
  T2JT(⟦num⟧) →    ⟦Double⟧ ;
  T2JT(⟦⟨T#1⟩→⟨T#2⟩⟧) →     ⟦Fun<⟨JT T2JT(#1)⟩, ⟨JT T2JT(#2)⟩>⟧ ;

  // Java expression templates.

  sort J | template EmitJFun(JT#tx, JT#tb, [a]J#[a as J])
    → ⟦ new Fun <⟨JT#tx⟩,⟨JT#tb⟩> ( ) {
          public ⟨JT#tb⟩ ap ( ⟨JT#tx⟩ ⟨V binds arg⟩ ) { return ⟨J#[arg as J]⟩; }
         } ⟧;

  sort J | template EmitJApp(J#1, J#2) →     ⟦ ⟨J#1⟩ . ap (⟨J#2⟩) ⟧ ;

  sort J | template EmitJIf(J#1, J#2, J#3)
    → ⟦ ( ⟨J#1⟩ > 0.0 ? ⟨J#2⟩ : ⟨J#3⟩ ) ⟧ ;
  
  sort JOp | ⟦⟨ADDOP⟩⟧ ;
  sort J | template EmitJOp EmitJOp(JOp#op, J#1, J#2)
    → ⟦ ( ⟨J#1⟩ ⟨JOp#op⟩ ⟨J#2⟩ ) ⟧ ;

  sort J | template EmitJNum(NUM#n) →       ⟦ ( double ) ⟨NUM#n⟩ ⟧ ;

  sort J | template EmitJRef(V#) →    ⟦ ⟨V#⟩ ⟧ ;

  // Generate expression code.

  sort J | scheme L2J(L) ;

  L2J(⟦ λ x : ⟨T#tx⟩ . ⟨L#L[⟦x⟧] ↑t(#tb)⟩ ⟧)
    → EmitJFun(T2JT(#tx), T2JT(#tb), [a]L2J(#L[a])) ;

  L2J(⟦ ⟨L#1⟩ ⟨L#2⟩ ⟧)→  EmitJApp(L2J(#1), L2J(#2)) ;

  L2J(⟦ ⟨L#1⟩ ? ⟨L#2⟩ : ⟨L#3⟩ ⟧)→   EmitJIf(L2J(#1), L2J(#2), L2J(#3)) ;

  L2J(⟦ ⟨L#1⟩ ⟨ADDOP#op⟩ ⟨L#2⟩ ⟧)
    → EmitJOp(⟦⟨ADDOP#op⟩⟧, L2J(#1), L2J(#2)) ;

  L2J(⟦ ⟨NUM#n⟩ ⟧) →     EmitJNum(#n) ;

  L2J(⟦ v ⟧) →  EmitJRef(⟦v⟧) ;
  \end{hacs}
  \caption{Java expression code generation.}
  \label{fig:cogen}
\end{figure}

\begin{example}
  Figure~\ref{fig:cogen} shows our main Java code generator, which translates a λ term to a Java
  expression, with the following points:
  %%
  \begin{enumerate}

  \item The first few lines, 76--79, define a syntax-directed translation scheme "T2JT" from simple
    λ types to the Java generic representation "JT" of the type that we will use.

  \item The next lines, 83--99, define the shapes of Java code that we will generate. This uses the
    new construct "template", which is merely a shorthand for new syntax with an associated scheme:
    ``$|~"template"~A→B$'' is the same as ``$|~"scheme"~A \mid B ; A→B $,'' where the first
    instances have been cleared of "#"s. (Templates help eliminate typing mistakes in the output
    syntax.)

  \item The following block, lines 103--116, defines the syntax-directed scheme "L2J" for
    translating λ expressions to Java ones.

  \item Line 88, for example, contains a simple template "EmitJApp" expressing that a λ application
    is translated to a Java use of the ".ap" method on the function (object). The template is used
    in line~108, which translates a λ application to method invocation of the translated subterms.

  \item Similarly, lines 90--91 and 110 handle positive number tests, 93--95 and 112--113 handle
    binary operators, 97 and 115 handle numeric constants, and 99 and 117 handle variable
    references.

  \item The one case that uses HOAS in a non-trivial way is the translation of λ abstractions. Line
    83 defines the template "EmitJFun" with three arguments: two Java types and one special higher
    order argument: the notation "[a]J#[a as J]" means a "J"-sorted value with an unbound (also
    called ``free'') variable (also of sort "J"). This is also reflected by the template in lines
    84--86, which also captures the HOAS binding of the used Java argument, "arg", in the usual way.
    The "L2J" rule in lines 105--106 translates a λ abstraction; notice how this captures both of
    the argument and return types of the abstraction -- the latter from the synthesized attribute --
    and then emits the function template with a third argument that invokes the "L2J" scheme
    \emph{under} the passed binder by writing it as "[a]L2J(#L[a])". (The variable is here unparsed
    and should not be in "⟦…⟧".)

  \end{enumerate}
  %%
  Notice that we have chosen to use recursive syntax-directed schemes for the translation to Java in
  the example, however, the scheme could equally well have been specified as a synthesized attribute
  on the λ terms. Semantically, they are equivalent. The advantage to this is that the code then
  becomes ``memoized,'' which ensures that it only gets computed once. The advantage of a recursive
  scheme is that the new structure is indepedent from the old, which both allows and may require
  multiple computations.
\end{example}

All the code generation rules discussed here translate directly into the underlying higher order
rewriting mechanism~\cite{crsx}, so require no new implementation tricks.

%------------------------------------------------------------------------

\section{Completing the Compiler}
\label{sec:run}

To run the compiler, we just need to pack it up as a ``module'' and provide a top-level scheme.

\begin{figure}[p]
  \begin{hacs}[texcl,xleftmargin=2em,numbers=left,firstnumber=1]
module org.crsx.hacs.samples.LambdaJava {
  \end{hacs}
  \begin{center}
  \fbox{Code from Figures~\ref{fig:parser}, \ref{fig:check}, and \ref{fig:cogen}.}
  \end{center}
  \vspace*{-1em}
  \begin{hacs}[texcl,xleftmargin=2em,numbers=left,firstnumber=119,deletekeywords={static,main}]
  // Java output program template.
  sort JClass | template EmitJClass(J#) →    ⟦
    package org.crsx.hacs.samples;
    public class Sample {
      public abstract class Fun<A,R> { abstract public R ap(A arg); }
      public static void main ( String [ ] args ) { System.out.println(⟨J#⟩) ; }
    }
  ⟧ ;

  // Main.
  sort JClass | Compile(L) ;
  Compile(#L) →    EmitJClass(L2J(AddTypes(#L))) ;
}
  \end{hacs}
  \caption{Complete compiler module.}
  \label{fig:module}
  \vspace*{2em}

  %\scriptsize
  \inputJava[texcl,xleftmargin=2em,numbers=left,tabsize=4]{Sample.java}

  \caption{Result of compiling $(λ f:\text{num}→\text{num}.f\,2)(λ a:\text{num} . a \mathbin{?} a : 0-a) $.}
  \label{fig:java}\end{figure}

\begin{example}
  Figure~\ref{fig:module} shows the completed compiler, wrapped as a module and with a top-level
  scheme "Compile", which emits a Java class with the result of first adding types, translating to
  Java expressions, and finally wrapping in a Java class, corresponding to the previous sections.

  Storing this into a \HAX file called \emph{LambdaJava.hx}, we can compile and run it as follows:
  \begin{code}
    $ hacs LambdaJava.hx
    …
    $ ./LambdaJava.run --scheme=Compile \\
            --term="(λ f:num→num.f 2)(λ a:num . a ? a : 0-a)"
  \end{code}
  %%$
  which will result in output that we can reformat to what shown in Figure~\ref{fig:java}, a program
  that will print \texttt{2.0}. The (evolving) example is included in the \HAX release.
\end{example}

%------------------------------------------------------------------------

\section{Possible Extensions}
\label{sec:further}

While \HAX serves its purpose, there are still many things that can be improved, essentially
supporting further features of formal notation. As we are engaging our students, we also hope to
engage the community in finding what the best notation is for working with compilers in this way.

In this section we mention some \HAX features that we have not covered, and some that we have
planned but not yet fully implemented.
%%
\begin{enumerate}

\item \HAX used to permit Kleene operators ("*", "+", "?") in grammar rules but it turns out that
  this is no big help given that attribute and propagation rules have to be specified explicitly in
  any case.

\item Similarly to the way inheritance rules can be inferred for missing cases, we have experimented
  with doing the same for synthesis rules. For synthesis rules with a single subterm with the
  synthesized attribute this is not a problem, however, when more than one synthesized subattribute
  has to be combined, such as in lines 54--56 in Figure~\ref{fig:check}, then the combining
  operation is needed. We have experimented with definitions of a default ``join and unit'' style of
  semantic operators, but have not found this clear enough for general use.

\item We have experimented with allowing the use of \emph{inference rules} in the style of
  \emph{Centaur}~\cite{Borras+:sde1988} but with higher-order matching instead of Prolog unification
  (with its notorious ``non-occur check'' semantics). This provides an alternative to the attribute
  grammar style used in this presentation, which is sometimes more convenient.

\item \HAX also permits specifying \emph{evaluation semantics} by giving rules for actual
  evaluation, essentially generating an \emph{interpreter} for the language, using the underlying
  higher-order rewriting platform as the implementation language. We could, for example,
  ``implement'' our example language by translating all syntactic applications to semantic ones,
  like
  \begin{hacs}
  sort L | scheme Load(L);
  Load(⟦⟨L#F⟩ ⟨L#A⟩⟧ → Ap(Load(#F),Load(#A));
  …
  \end{hacs}
  (with ``"…"'' meaning simple recursive cases for the rest of "Load"), and then evaluate the
  semantic applications:
  \begin{hacs}
  sort L | scheme Ap(L,L);
  Ap(⟦λx:⟨T#tx⟩.⟨L#B[⟦x⟧]⟩⟧, #A⟧ →     #B[#A] ;
  \end{hacs}
  which uses the full higher order rewriting mechanism to implement the simply typed λ calculus with
  the usual β reduction rule: "#B[#A]" denotes the body of the λ abstraction except all the
  occurrences of the matched bound variable, "x", are replaced with a full copy of the "#A" term.

\end{enumerate}
%%
Here are some notations that we are considering:
%%
\begin{enumerate}[resume]

\item The current notation requires that vacuous recursive scheme rules, like those in line 42--47
  of Figure~\ref{fig:check}, are explicitly given. Since recursive (syntax directed) schemes must be
  fully defined over their (syntactic) sort, these rules could be automatically derived from the
  fact that the scheme ("Lte") is declared as carrying an inherited attribute, which makes it syntax
  directed by definition.

\item We are considering following the ASF+SDF family and permit also direct use of inherited
  attribute rules. In our notation, the inheritance corresponding to line 39--40 in
  Figure~\ref{fig:check} would look something like this:
  \begin{hacs}
      ⟦ λ x : ⟨T#tx⟩ . ⟨L#L[⟦x⟧]  ↓te{⟦x⟧ : #tx}⟩ ⟧   ↓te ;
  \end{hacs}
  This essentially shifts to an implicit carrier scheme. The difficulty with rules like this is that
  they mix the pattern and replacement parts in a more complex manner than the synthesis rules, and
  it is not clear how to explain this well. Of course this also moves the burden of calculating the
  strategy for sequencing the attribute propagation onto the system.

\item A trend in the language specification culture is to express parts of the semantics using
  constraints. We would like \HAX to be able to support specifications involving constraints, which
  would allow a much simpler notation for such things as type inference algorithms, for
  example. This would require that the runtime includes an SMT solver.

\end{enumerate}
%%
There are many more mundane issues, such as allowing the user to omit unused "#"-markers in patterns
that are not used.

%------------------------------------------------------------------------

\section{Conclusion}
\label{sec:conc}

We have presented \HAX as a compiler programming language through a small but hopefully illustrative
example, and we have shown how the compiler-specific notations translate into existing rewrite
notations. \HAX is available from the NYU compiler construction class web
site~\cite{RoseRose:cims2015} and all source code is in the CRSX github project~\cite{crsx}.

Compiler generation is a long and proud tradition in computer science
\cite{GilbertMcLellan:sjcp1967,%
  FeldmanGries:cacm1968,%
  Mosses:daimi1979,%
  Gaudel:fpc1981,%
  BjornerJones:1982,%
  CameronIto:toplas1984,%
  Wand:sn1984,%
  Borras+:sde1988,%
  Brand+:toplas2002,%
  Bos+:eptcs2011}, and \HAX truly stands on the shoulders of giants. What \HAX does differently is
combining the state of the art in compiler generation with the full power of higher-order rewriting,
integrating the parsing and sort systems tightly.

We would like to give credit to SIS~\cite{Mosses:daimi1979}, which shares the use of
\emph{simplification} to a λ-calculus based formalism with \HAX, as illustrated, for example, by how
the interpretation rule can be encoded (as shown in the previous section).

The most prominent system that supports implementation of compilers in formal (attribute grammar)
form is ASF+SDF~\cite{Brand+:toplas2002}, which is based on first order rewriting. While modules
have been added for symbol table management, these lack the full integration and easy way to handle
scoped intermediate languages. The successor, Rascal~\cite{Bos+:eptcs2011} adds a module for HOAS,
but Rascal specifications operate in a world of side effects, which we find hard to reconcile with
higher-order term structures (with scopes).

The notion of ``higher-order'' used by \HAX is similar to but not quite the same as in higher-order
attribute grammars (HAG)~\cite{VogtSwierstraKuiper:pldi1989}. Like HAGs, \HAX specifications permit
constructing and passing of abstract syntax fragments in attributes -- indeed our example uses this
in the type synthesis rules -- but the ``higher order'' aspect of \HAX also covers the rewriting
side, where we can build parameterized abstractions over any part of a specification, including with
attributes. Indeed, one can use substitution inside attributes, have synthetic attributes block
schemes (as we have seen), have lack of substitution block synthesis (leave ``unsynthesizable
variables'' in a term), \etc

\HAX supports full HOAS, even with nested reductions under unevaluated binders, yet the applications
we consider are quite close to the examples presented for PHOAS by
Chlipala~\cite{Chlipala:icfp2008}. We are hoping to understand how many of the results of Chlipala
carry over to \HAX.

Finally, it is clear that the formal foundation of \HAX, while based in the solid culture of higher
order rewriting, still needs to be fully qualified and categorized in this world, so we can express
precisely (and hopefully also implement) standard tools such as termination checking for the
language. This is the subject of our upcoming EU-funded Marie Skłodowska-Curie ``HORIP'' action
collaboration with Cynthia Kop.

\paragraph*{Acknowledgements.} The author would like to thank my coteacher at NYU, Eva Rose, our
graders, Ma Weicheng and José Pablo Cambronero, as well as our students in the compiler construction
class, for constructive comments to and insightful questions on \HAX.\footnote{A special shout-out
  goes to John Downs for starting the Emacs \texttt{M-x hacs-mode} syntax highligting
  mode~\cite{git:hacsel}.}
%%
The implementation of \HAX would not have been remotely possible without the CRSX team at IBM:
Morris Matsa, Scott Boag, and especially Lionel Villard, who suffered through understanding and then
programming both core fragments of and using the CRSX system that is still used underneath \HAX.
%%
And finally \HAX owes its sanity to a collaboration with Cynthia Kop, both as an intern with the
Watson team in 2011, which created the polymorphic sort checker, in her thesis work~\cite{Kop:2012}
and our continuing collaboration.


%------------------------------------------------------------------------

\bibliography{crs}

\end{document}


%------------------------------------------------------------------------
% Tell Emacs that this is a LaTeX document and how it is formatted:
% Local Variables:
% mode:latex
% fill-column:100
% TeX-master: t
% TeX-auto-untabify: nil
% End:
