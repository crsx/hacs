%% $Id: setup.tex,v 1.11 2013/12/05 04:10:03 krisrose Exp $

%% Format.
\usepackage[utf8]{inputenc}
\usepackage{comment}
\usepackage{amsmath,amssymb,stmaryrd,textcomp}
\allowdisplaybreaks %style
\usepackage[normalem]{ulem}
\usepackage{pgfpages}
\usepackage{utf8math}
\usepackage{url}
\usepackage{xspace}
\usepackage{fancyvrb}

%% Ignore frame environments in article.
\makeatletter
\ifx\beamer@currentmode\undefined
  \usepackage{xcolor}
\fi
\makeatother

\usepackage{graphicx}
\usepackage{hyperref}
\usepackage{rcs}
\usepackage{enumitem}

%% Macros.
\newcommand{\HAX}{\text{H\kern-.1em A\kern-.1em C\kern-.05em S}\xspace}
\newcommand{\hax}{\ensuremath{\hslash}\xspace}
\newcommand{\bhax}{\ensuremath{\pmb{\hslash}}\xspace}
\newcommand{\CRSX}{\text{CRSX}\xspace}
\newcommand{\ie}{\textit{i.e.}\xspace}
\newcommand{\eg}{\textit{e.g.}\xspace}
\newcommand{\etc}{\textit{etc.}\xspace}
\newcommand{\aka}{\textit{aka.}\xspace}
\newcommand{\TBD}[2][]{\textit{To Be Done: #2~}\marginpar{\bf\color{red} TBD #1}}

%% Misc.
\def\Bigstrut{\ensuremath{\vphantom{\Bigm|}}}
\newcommand\caret{\mathbin{\char`\^}}
\newcommand{\ov}[1]{\ensuremath{\mkern2mu\overline{\mkern-1.8mu{#1}\mkern-.8mu}\mkern1mu}}

%% Units
\ifx\theorem\undefined
  \usepackage{amsthm}
  \swapnumbers %style
  \newtheorem{theorem}{Theorem}[section]
  \newtheorem{lemma}[theorem]{Lemma}
  \theoremstyle{definition}
  \newtheorem{definition}[theorem]{Definition}
  \newtheorem{example}[theorem]{Example}
  \newtheorem{note}[theorem]{Note}
\fi
\ifx\remark\undefined
  \theoremstyle{definition}
  \newtheorem{proposition}[theorem]{Proposition}
  \newtheorem{practice}[theorem]{Practice}
  \newtheorem{notation}[theorem]{Notation}
  \newtheorem{manual}[theorem]{Manual}
  %\newenvironment{manual}{\small\begin{manual-body}}{\end{manual-body}}
  \newtheorem{invariant}[theorem]{Invariant}
  \newtheorem{requirements}[theorem]{Requirements}
  \newtheorem{limitations}[theorem]{Limitations}
  \newtheorem{commands}[theorem]{Commands}
  \newtheorem{error}[theorem]{Error}
  \newtheorem{plan}[theorem]{Plan}
  %\theoremstyle{remark}
  \newtheorem{remark}[theorem]{Remark}
\fi

%% Old code listings.
\usepackage{alltt}
\usepackage{fancyvrb}
\CustomVerbatimEnvironment{code}{Verbatim}{tabsize=4,fontsize=\small,numberblanklines=false,xleftmargin=\parindent}
\CustomVerbatimCommand{\inputcode}{VerbatimInput}{tabsize=4,fontsize=\footnotesize,numbers=left,numberblanklines=false,xleftmargin=\parindent}
%\DefineShortVerb{\"} % makes " work in verbatim even when \active


%% New code listings...
\usepackage{listings}
\lstset{basicstyle=\ttfamily}

%%  Java.
\lstloadlanguages{Java}
\newcommand\inputJava[2][]{\lstinputlisting[language=Java,%
  basicstyle=\fontfamily{ccr}\selectfont,%
  extendedchars=true,inputencoding=utf8,sensitive,%
  commentstyle=\color{blue}\sl,%
  identifierstyle=\textit,%
  keywordstyle=\kw,%
  basewidth={.5em,.4em},columns=flexible,numberstyle=\scriptsize,numberblanklines=false,xrightmargin=1pc,%
  #1]{#2}}

\lstnewenvironment{Java}[1][]{\lstset{language=Java,%
  basicstyle=\fontfamily{ccr}\selectfont,%
  extendedchars=true,inputencoding=utf8,sensitive,%
  commentstyle=\color{blue}\sl,%
  identifierstyle=\textit,%
  keywordstyle=\kw,%
  basewidth={.5em,.4em},columns=flexible,numberstyle=\scriptsize,numberblanklines=false,xrightmargin=1pc,%
  #1}\upshape}{}

%% HACS.

%% Inside listings \Identifier handles these cases:
%% - #xx_n → \mv{xx}\ensuremath{_{\text{n}}}
%% - $xx_n → \kw{xx}\ensuremath{_{\text{n}}}
%% - Xx_n → \con{xx}\ensuremath{_{\text{n}}}
%% - xx_n → \var{xx}\ensuremath{_{\text{n}}}
%% - x_n → \ensuremath{x_{n}}
{\makeatletter \catcode`!=6 \catcode`\#=11 \catcode`\$=11 \catcode`\_=11 %
\gdef\justh@sh{\lst@um# }
\gdef\just@t{\lst@um @ }
\gdef\justd@llar{\lst@um$ }
\gdef\just@nder{\lst@um_ }
\gdef\Identifier!1{\expandafter\Identifier@two\the\lst@token \lst@um_ @}
\gdef\Identifier@two!1!2\lst@um_!3@{{%
  \def\1{!1}\def\2{!2}\def\3{!3}%
  %\message{FIRST=\meaning\1 SECOND=\meaning\2 THIRD=\meaning\3}%
  \ifx\3\empty\else
    \expandafter\Identifier@subscript\3@%
  \fi
  \ifx\1\justh@sh              \mv{\expandafter\def\csname lst@um-\endcsname{\text{-}}\2\3}%g
  \else\ifx\1\justd@llar
    \ifx\2\empty               \kw{\$\3}%
    \else                      \kw{\expandafter\def\csname lst@um-\endcsname{\text{-}}\2\3}%
    \fi
  \else\ifx\1\just@at
    \kw{\expandafter\def\csname lst@um-\endcsname{\text{-}}\1\2\3}%
  \else\ifx\1\just@nder
    \con{\expandafter\def\csname lst@um-\endcsname{\text{-}}\1\2\3}%
  \else
    \edef\4{`\1=\noexpand\the\noexpand\uccode`\1}%
    \expandafter\ifnum\4\relax \con{\expandafter\def\csname lst@um-\endcsname{\text{-}}\1\2\3}%
    \else\ifx\2\empty          \ensuremath{\1\3}%
    \else                      \var{\expandafter\def\csname lst@um-\endcsname{\text{-}}\1\2\3}%
  \fi\fi\fi\fi\fi\fi}}
\gdef\Identifier@subscript!1\lst@um_ @{\def\3{\ensuremath{\sb{\textup{!1}}}}}
}%$
\newcommand{\mv}[1]{\textsc{#1}}
%\newcommand{\kw}[1]{\text{\rm\textbf{\color{gray}#1}}}
\newcommand{\kw}[1]{\textsf{\bf\color{gray}#1}}
\newcommand{\cw}[1]{\textsf{\color{gray}#1}}
\newcommand{\con}[1]{\textup{#1}}
\newcommand{\var}[1]{\textit{#1}}

\lstdefinelanguage{HACS}{basicstyle=\fontfamily{ccr}\selectfont,%
  extendedchars=true,inputencoding=utf8,sensitive,%
  identifierstyle=\Identifier,%
  string=[d]{"},morestring=[d]{'},upquote=true,showstringspaces=false,%
  comment=[l]{//},morecomment=[s]{/*}{*/},commentstyle=\color{blue}\rm,%
  keywordstyle=\kw,%
  keywords={as,attribute,data,fragment,global,import,main,module,nested,property,rule,scheme,simplify,sort,%
    space,static,sugar,symbol,tag,token,default,error,binds,template,variable},%
  literate=%
    {⟦}{{\ensuremath{\llbracket}}}1 {⟧}{{\ensuremath{\rrbracket}}}1 {⟨}{{\ensuremath{\langle}}}1 {⟩}{{\ensuremath{\rangle}}}1 %
    {[}{{\ensuremath{[\mkern.5mu}}}1 {]}{{\ensuremath{\mkern.5mu]}}}1 {⟨}{{\ensuremath{\langle}}}1 {⟩}{{\ensuremath{\rangle}}}1 %
    {→}{{\ensuremath{\rightarrow}}}2 {↑}{{\ensuremath{\uparrow}}}1 {↓}{{\ensuremath{\downarrow}}}1 {¬}{{\ensuremath{\lnot}}}2 {¶}{{\P}}1 %
    {ε}{{\ensuremath{\varepsilon}}}1 {…}{{\ensuremath{\dots}}}2 {∧}{{\ensuremath{\!\wedge}}}2 {∨}{{\ensuremath{\!\vee}}}2 {↦}{{\ensuremath{\mapsto}}}2 %
    {α}{{\ensuremath{\alpha}}}1 {β}{{\ensuremath{\beta}}}1 {γ}{{\ensuremath{\gamma}}}1 %
    {λ}{{\ensuremath{\lambda}}}1 {Λ}{{\ensuremath{\Lambda}}}1 %
    {σ}{{\ensuremath{\sigma}}}1 {Σ}{{\ensuremath{\Sigma}}}1 {π}{{\ensuremath{\pi}}}1 {Π}{{\ensuremath{\Pi}}}1 %
    {λ̅}{{\ensuremath{\overline{\lambda}}}}1 {‾}{{\ensuremath{\overline{\phantom{x}}}}}1 %
    {¹}{{\ensuremath{^1}}}1 {²}{{\ensuremath{^2}}}2 %
    {@0}{{\cw{@0}}}2 {@1}{{\cw{@1}}}2 {@2}{{\cw{@2}}}2 {@3}{{\cw{@3}}}2 {@4}{{\cw{@4}}}2 %
    {@5}{{\cw{@5}}}2 {@6}{{\cw{@6}}}2 {@7}{{\cw{@7}}}2 {@8}{{\cw{@8}}}2 {@9}{{\cw{@9}}}2 %
    {@}{{\cw{@}}}1 %
}[comments,strings,keywords]
%%
\lstnewenvironment{hacs}[1][]{\lstset{language=HACS,basewidth={.53em,.43em},columns=flexible,numberstyle=\scriptsize\color{blue},numberblanklines=false,firstnumber=auto,xrightmargin=1pc,#1}\upshape}{}
\newcommand\inputhacs[2][]{\lstinputlisting[language=HACS,basewidth={.53em,.43em},columns=flexible,numberstyle=\scriptsize,numberblanklines=false,xrightmargin=1pc,#1]{#2}}
\newcommand\hacsc{\lstinline[language=HACS]}
\newcommand\h{\lstinline[language=HACS]}
\lstMakeShortInline[language=HACS,columns=fullflexible]"

%%% \usepackage{listings}
%%% 
%%% %% HACS.
%%% \lstdefinelanguage{HACS}{basicstyle=\rm,%
%%%   numberstyle=\scriptsize,
%%%   numbersep=7pt,
%%%   numberblanklines=false,
%%%   extendedchars=true,inputencoding=utf8,sensitive,%
%%%   identifierstyle=\textit,%
%%%   string=[d]{"},morestring=[m]{'},upquote=true,stringstyle=\ttfamily,%showstringspaces=true,%
%%%   comment=[l]{//},morecomment=[s]{/*}{*/},commentstyle=\color{gray}\upshape,%
%%%   keywordstyle=\color{black}\bfseries,%
%%%   keywords={attribute,data,default,fragment,import,main,module,nested,property,rule,scheme,simplify,sort,space,static,sugar,symbol,tag,token},%
%%%   ndkeywordstyle={\textup},%
%%%   literate=%
%%%     {⟦}{{\ensuremath{\llbracket}}}1 {⟧}{{\ensuremath{\rrbracket}}}1 {⟨}{{\ensuremath{\langle}}}1 {⟩}{{\ensuremath{\rangle}}}1 %
%%%     {→}{\ensuremath{\rightarrow}}2 {↑}{\ensuremath{\uparrow}}1 {↓}{\ensuremath{\downarrow}}1 {ε}{\ensuremath{\varepsilon}}1 %
%%%     {…}{\ensuremath{\dots}}2 {¬}{\ensuremath{\lnot}}1 {∧}{\ensuremath{\wedge}}2 {∨}{\ensuremath{\vee}}2 {¶}{\P}1 {↦}{\ensuremath{\mapsto}}2 %
%%% }[comments,strings,keywords]
%%% %%
%%% \lstnewenvironment{hacs}[1][]{\lstset{language=HACS,basewidth={.55em,.4em},columns=flexible,#1}\upshape}{}
%%% \newcommand\inputhacs[2][]{\lstinputlisting[language=HACS,basewidth={.55em,.4em},columns=flexible,#1]{#2}}
%%% \newcommand\hacsc{\lstinline[language=HACS]}
%%% \lstMakeShortInline[language=HACS,columns=fullflexible]"
%%% 
%%% %% Tiger.
%%% \lstdefinelanguage{Tiger}{basicstyle=\normalsize,%
%%%   extendedchars=true,inputencoding=utf8,sensitive,%
%%%   identifierstyle=\textit,%
%%%   string=[d]{"},upquote=true,stringstyle=\ttfamily,showstringspaces=true,%
%%%   comment=[s]{/*}{*/},commentstyle=\color{grey}\rm,%
%%%   keywordstyle=\color{black}\bfseries,%
%%%   keywords={type,array,of,var,function,nil,let,in,end}%
%%% }[comments,strings,keywords]
%%% %%
%%% \lstnewenvironment{tiger}[1][]{\lstset{language=Tiger,basewidth={.5em,.4em},columns=flexible,#1}\upshape}{}
%%% \newcommand\inputtiger[2][]{\lstinputlisting[language=Tiger,basewidth={.5em,.4em},columns=flexible,#1]{#2}}
%%% \newcommand\tigerc{\lstinline[language=Tiger]}

%%% From stackoverflow
%\lstset{
%         basicstyle=\footnotesize\ttfamily, % Standardschrift
%         %numbers=left,               % Ort der Zeilennummern
%         numberstyle=\tiny,          % Stil der Zeilennummern
%         %stepnumber=2,               % Abstand zwischen den Zeilennummern
%         numbersep=5pt,              % Abstand der Nummern zum Text
%         tabsize=2,                  % Groesse von Tabs
%         extendedchars=true,         %
%         breaklines=true,            % Zeilen werden Umgebrochen
%         keywordstyle=\color{red},
%            frame=b,         
% %        keywordstyle=[1]\textbf,    % Stil der Keywords
% %        keywordstyle=[2]\textbf,    %
% %        keywordstyle=[3]\textbf,    %
% %        keywordstyle=[4]\textbf,   \sqrt{\sqrt{}} %
%         stringstyle=\color{white}\ttfamily, % Farbe der String
%         showspaces=false,           % Leerzeichen anzeigen ?
%         showtabs=false,             % Tabs anzeigen ?
%         xleftmargin=17pt,
%         framexleftmargin=17pt,
%         framexrightmargin=5pt,
%         framexbottommargin=4pt,
%         %backgroundcolor=\color{lightgray},
%         showstringspaces=false      % Leerzeichen in Strings anzeigen ?        
% }

%% Identifiers
\newcommand{\set}[1]{\textit{#1}}
\newcommand{\op}[1]{\ensuremath{\operatorname{#1}}}

\usepackage[all]{xy}
\SelectTips{eu}{}

\makeatletter
\def\compacttableofcontents{{\if@twocolumn\else \quotation\small \fi
    \noindent\textbf{Contents:}~%
    \def\protect{\relax}%
    \def\contentsline##1{\csname contentslineA##1\endcsname}%
    \def\numberline##1{\DN@{##1}\ifx\next@\empty\else##1.~\fi}%
    \def\fresh##1{\relax \let\fresh=\nextfresh}%
    \def\nextfresh##1{\relax##1 }%
    \def\contentslineAsection##1##2##3{\fresh,##1~(##2)\ignorespaces}%
    \def\contentslineAsubsection##1##2##3{\ignorespaces}%
    \@starttoc{toc}.\endquotation}}
\makeatother

%% WORKING \Cases{...} (based on \cases from plain TeX -- the one in
%% AMS-LaTeX is seriously buggy):
\makeatletter
\def\Cases#1{\mbox{$
  \left\{\,\vcenter{\let\\=\cr \normalbaselines\m@th
    \ialign{$\vphantom(##\hfil$&\quad##\hfil\crcr#1\crcr}}\right.$}}
\makeatother

%%% Local Variables:
%%% mode: plain-tex
%%% TeX-master: t
%%% End:
