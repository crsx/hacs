%% HACS 2 NOTES.
%%
%% Copyright (c) 2015 Kristoffer Rose <krisrose@crsx.org>
%% 
%%
\documentclass[11pt]{article} %style: font size.
\usepackage[utf8]{inputenc}

\usepackage[type={CC},modifier={by},version={4.0}]{doclicense}
\newcommand{\basecopyright}{\noindent
  \HAX is © 2011, 2015 Kristoffer Rose and released under the
  \href{https://www.eclipse.org/legal/epl-v10.html}{Eclipse Public License 1.0}.\\
  \noindent Documentation is \doclicenseImage[imagewidth=3em]
  2011,2015 Kristoffer Rose.}
\newcommand{\documentcopyright}{\basecopyright}

%% Style.
\usepackage[margin=.7in]{geometry}
\usepackage[T1]{fontenc}
\bibliographystyle{plainurl}
\renewcommand{\rmdefault}{pplx}\usepackage{eulervm}\AtBeginDocument{\SelectTips{eu}{11}}

%% Base format.
%% $Id: setup.tex,v 1.11 2013/12/05 04:10:03 krisrose Exp $

%% Format.
\usepackage[utf8]{inputenc}
\usepackage{comment}
\usepackage{amsmath,amssymb,stmaryrd,textcomp}
\allowdisplaybreaks %style
\usepackage[normalem]{ulem}
\usepackage{pgfpages}
\usepackage{utf8math}
\usepackage{url}
\usepackage{xspace}
\usepackage{fancyvrb}
\usepackage{xcolor}
\usepackage{graphicx}
\usepackage{hyperref}
\usepackage{rcs}

%% Macros.
\newcommand{\HAX}{\text{HACS}\xspace}
\newcommand{\CRSX}{\text{CRSX}\xspace}
\newcommand{\ie}{\textit{i.e.}\xspace}
\newcommand{\eg}{\textit{e.g.}\xspace}
\newcommand{\etc}{\textit{etc.}\xspace}
\newcommand{\aka}{\textit{aka.}\xspace}
\newcommand{\TBD}[1][]{\textit{To Be Done…#1}\marginpar{\bf\color{red} TBD}}

%% Misc.
\def\Bigstrut{\ensuremath{\vphantom{\Bigm|}}}
\newcommand\caret{\mathbin{\char`\^}}

%% Units
\usepackage{amsthm}
\swapnumbers %style
\newtheorem{theorem}{Theorem}[section]
\newtheorem{lemma}[theorem]{Lemma}
\theoremstyle{definition}
\newtheorem{definition}[theorem]{Definition}
\newtheorem{proposition}[theorem]{Proposition}
\newtheorem{example}[theorem]{Example}
\newtheorem{practice}[theorem]{Practice}
\newtheorem{notation}[theorem]{Notation}
\newtheorem{manual}[theorem]{Manual}
%\newenvironment{manual}{\small\begin{manual-body}}{\end{manual-body}}
\newtheorem{invariant}[theorem]{Invariant}
\newtheorem{requirements}[theorem]{Requirements}
\newtheorem{limitations}[theorem]{Limitations}
\newtheorem{commands}[theorem]{Commands}
\newtheorem{error}[theorem]{Error}
\newtheorem{note}[theorem]{Note}
\newtheorem{plan}[theorem]{Plan}
%\theoremstyle{remark}
\newtheorem{remark}[theorem]{Remark}

%% Old code listings.
\usepackage{alltt}
\usepackage{fancyvrb}
\CustomVerbatimEnvironment{code}{Verbatim}{tabsize=4,fontsize=\small,numberblanklines=false,xleftmargin=\parindent}
\CustomVerbatimCommand{\inputcode}{VerbatimInput}{tabsize=4,fontsize=\footnotesize,numbers=left,numberblanklines=false,xleftmargin=\parindent}
%\DefineShortVerb{\"} % makes " work in verbatim even when \active


%% New code listings...
\usepackage{listings}
\lstset{basicstyle=\ttfamily}

%% HACS.
\def\Identifier#1{#1}
\def\Keyword#1{\textsf{\color{gray}#1}}

\lstdefinelanguage{HACS}{basicstyle=\fontfamily{ccr}\selectfont,%
  extendedchars=true,inputencoding=utf8,sensitive,%
  identifierstyle=\Identifier,%
  string=[d]{"},morestring=[d]{'},upquote=true,showstringspaces=false,%
  comment=[l]{//},morecomment=[s]{/*}{*/},commentstyle=\color{blue}\rm,%
  keywordstyle=\Keyword,%
  keywords={attribute,data,fragment,import,main,module,nested,property,rule,scheme,simplify,sort,%
    space,static,sugar,symbol,tag,token,default,error,binds,as},%
  literate=%
    {⟦}{{\ensuremath{\llbracket}}}1 {⟧}{{\ensuremath{\rrbracket}}}1 {⟨}{{\ensuremath{\langle}}}1 {⟩}{{\ensuremath{\rangle}}}1 %
    {→}{{\ensuremath{\rightarrow}}}2 {↑}{{\ensuremath{\uparrow}}}1 {↓}{{\ensuremath{\downarrow}}}1 {¬}{{\ensuremath{\lnot}}}2 {¶}{{\P}}1 %
    {ε}{{\ensuremath{\varepsilon}}}1 {…}{{\ensuremath{\dots}}}2 {∧}{{\ensuremath{\!\wedge}}}2 {∨}{{\ensuremath{\!\vee}}}2 {↦}{{\ensuremath{\mapsto}}}2 %
}[comments,strings,keywords]
%%
\lstnewenvironment{hacs}[1][]{\lstset{language=HACS,basewidth={.53em,.43em},columns=flexible,numberstyle=\scriptsize,numberblanklines=false,firstnumber=auto,xrightmargin=1pc,#1}\upshape}{}
\newcommand\inputhacs[2][]{\lstinputlisting[language=HACS,basewidth={.53em,.43em},columns=flexible,numberstyle=\scriptstyle,numberblanklines=false,xrightmargin=1pc,#1]{#2}}
\newcommand\hacsc{\lstinline[language=HACS]}
\lstMakeShortInline[language=HACS,columns=fullflexible]"


%%% \usepackage{listings}
%%% 
%%% %% HACS.
%%% \lstdefinelanguage{HACS}{basicstyle=\rm,%
%%%   numberstyle=\scriptsize,
%%%   numbersep=7pt,
%%%   numberblanklines=false,
%%%   extendedchars=true,inputencoding=utf8,sensitive,%
%%%   identifierstyle=\textit,%
%%%   string=[d]{"},morestring=[m]{'},upquote=true,stringstyle=\ttfamily,%showstringspaces=true,%
%%%   comment=[l]{//},morecomment=[s]{/*}{*/},commentstyle=\color{gray}\upshape,%
%%%   keywordstyle=\color{black}\bfseries,%
%%%   keywords={attribute,data,default,fragment,import,main,module,nested,property,rule,scheme,simplify,sort,space,static,sugar,symbol,tag,token},%
%%%   ndkeywordstyle={\textup},%
%%%   literate=%
%%%     {⟦}{{\ensuremath{\llbracket}}}1 {⟧}{{\ensuremath{\rrbracket}}}1 {⟨}{{\ensuremath{\langle}}}1 {⟩}{{\ensuremath{\rangle}}}1 %
%%%     {→}{\ensuremath{\rightarrow}}2 {↑}{\ensuremath{\uparrow}}1 {↓}{\ensuremath{\downarrow}}1 {ε}{\ensuremath{\varepsilon}}1 %
%%%     {…}{\ensuremath{\dots}}2 {¬}{\ensuremath{\lnot}}1 {∧}{\ensuremath{\wedge}}2 {∨}{\ensuremath{\vee}}2 {¶}{\P}1 {↦}{\ensuremath{\mapsto}}2 %
%%% }[comments,strings,keywords]
%%% %%
%%% \lstnewenvironment{hacs}[1][]{\lstset{language=HACS,basewidth={.55em,.4em},columns=flexible,#1}\upshape}{}
%%% \newcommand\inputhacs[2][]{\lstinputlisting[language=HACS,basewidth={.55em,.4em},columns=flexible,#1]{#2}}
%%% \newcommand\hacsc{\lstinline[language=HACS]}
%%% \lstMakeShortInline[language=HACS,columns=fullflexible]"
%%% 
%%% %% Tiger.
%%% \lstdefinelanguage{Tiger}{basicstyle=\normalsize,%
%%%   extendedchars=true,inputencoding=utf8,sensitive,%
%%%   identifierstyle=\textit,%
%%%   string=[d]{"},upquote=true,stringstyle=\ttfamily,showstringspaces=true,%
%%%   comment=[s]{/*}{*/},commentstyle=\color{grey}\rm,%
%%%   keywordstyle=\color{black}\bfseries,%
%%%   keywords={type,array,of,var,function,nil,let,in,end}%
%%% }[comments,strings,keywords]
%%% %%
%%% \lstnewenvironment{tiger}[1][]{\lstset{language=Tiger,basewidth={.5em,.4em},columns=flexible,#1}\upshape}{}
%%% \newcommand\inputtiger[2][]{\lstinputlisting[language=Tiger,basewidth={.5em,.4em},columns=flexible,#1]{#2}}
%%% \newcommand\tigerc{\lstinline[language=Tiger]}

%% Identifiers
\newcommand{\set}[1]{\textit{#1}}
\newcommand{\op}[1]{\ensuremath{\operatorname{#1}}}

\usepackage[all]{xy}
\SelectTips{eu}{}

\makeatletter
\def\compacttableofcontents{{\if@twocolumn\else \quotation\small \fi
    \noindent\textbf{Contents:}~%
    \def\protect{\relax}%
    \def\contentsline##1{\csname contentslineA##1\endcsname}%
    \def\numberline##1{\DN@{##1}\ifx\next@\empty\else##1.~\fi}%
    \def\fresh##1{\relax \let\fresh=\nextfresh}%
    \def\nextfresh##1{\relax##1 }%
    \def\contentslineAsection##1##2##3{\fresh,##1~(##2)\ignorespaces}%
    \def\contentslineAsubsection##1##2##3{\ignorespaces}%
    \@starttoc{toc}.\endquotation}}
\makeatother


%% Topmatter.
\title{
  Evolution From \HAX\ 1 to 2
}
\author{
  Kristoffer H. Rose\\
  Two Sigma Investments / New York University
}

\begin{document}
\maketitle

\begin{abstract}\noindent
  This document discusses curious implementation details of \HAX version 1 and how they should be
  fixed and new things added in \HAX version 2.

  \compacttableofcontents

  \vspace*{2em}\small\color{gray}\noindent%
  \documentcopyright
\end{abstract}


\section{Lexing \& Parsing}\label{sec:parsing}

In version 1, the Prep script translates a grammar to three parsers in mixed pg/JavaCC form:
%%
\begin{description}

\item[User input parser.] Has all user syntax as specified by the grammar with no recognition of
  \HAX meta-notation ("⟦⟨⟩⟧"). Generates user AST terms. Only parser that is used by the final
  user script.

\item[Raw parser.] Recognizes all declared ``raw'' constructs as well as generic \HAX syntax
  (meta-variables with \#, \etc). All material in syntax brackets ("⟦⟧") results in invocation of
  the embedded parser below. Generates Cook-ready internal \HAX AST.

\item[Embedded parser.] Recognizes the user's specified syntax extended with \HAX meta-notation for
  embedding ``raw'' \HAX into syntax. All raw syntax is delegated to the raw parser above. Generates
  Cook-ready internal \HAX AST.

\end{description}
%%
In version 2, we plan for several tweaks to this:
%%
\begin{enumerate}

\item A syntactic production (both \kw{scheme} and data) can be defined \kw{private}, which means
  that the construct is only included in the embedded syntax parser, not the user input parser.

\item We reintroduce automatic list generation by having derived non-terminals of the form $NM$
  where $N$ is a non-terminal and $M$ is one of "?", "*", and "+"$U$, with the usual RE semantics
  and $U$ an optional unit as in \kw{token} declarations allowed just for "+". The resulting
  non-terminal is registered as such, making the following kinds of definitions possible:
  %%
  \begin{itemize}

  \item For $M="?"$ the required patterns are:
    \begin{hacs}[mathescape]
    sort … | scheme F($N$?);
    F(⟦ ⟨$N$ #1⟩ ⟧) →  …;
    F(⟦ ⟧) →  …;
    \end{hacs}

  \item For $M="*"$ the required patterns are:
    %%
    \begin{hacs}[mathescape]
    sort … | scheme F($N$*);
    F(⟦ ⟨$N$ #1⟩ ⟨$N$* #rest⟩ ⟧) →  …;
    F(⟦ ⟧) →  …;
    \end{hacs}
    The first rule is applied for every list element, and the second at the end of the list
    (innermost).

  \item For $M="+"U$, the patterns include the special $U$ separator and must have this shape:
    \begin{hacs}[mathescape]
    sort … | scheme F($N$+$U$);
    F(⟦ ⟨$N$ #1⟩ $U$ ⟨$N$+$U$ #rest⟩ ⟧) →  …;
    F(⟦ ⟨$N$ #1⟩ ⟧) →  …;
    \end{hacs}
    where the second rule is applied to just the last list element (so innermost). The first rule
    should insert the $U$ literally, of course, for example like this:
    \begin{hacs}[mathescape]
    sort Bool | scheme IsEmpty(Elem+[,]);
    IsEmpty(⟦ ⟨Elem #1⟩ ,  ⟨Elem+[,] #rest⟩ ⟧) →  True;
    IsEmpty(⟦ ⟨Elem #1⟩ ⟧) →  False;
    \end{hacs}

  \end{itemize}

\item We plan to allow modular parsers, described in the Modules section~\ref{sec:modules}.
\end{enumerate}


\section{Modules}\label{sec:modules}

In version 1, one can only have one top level module. In version 2, modules can \emph{import} other
modules.

The semantics of multiple modules is as follows:
%%
\begin{enumerate}

\item Each module defines a \emph{separate parser family}. As in version 1, this effectively means
  three parsers:
  \begin{itemize}
  \item A
  \end{itemize}


\end{enumerate}



\end{document}


%%---------------------------------------------------------------------
% Tell Emacs that this is a LaTeX document and how it is formatted:
% Local Variables:
% mode:latex
% fill-column:100
% TeX-master: t
% End:
