\documentclass[pdftex,aspectratio=169,14pt]{beamer}
%% $Id: lecture-setup.tex,v 1.12 2013/12/05 04:28:18 krisrose Exp $

%% Format.
\usepackage[utf8]{inputenc}
\usepackage{comment}
\usepackage{amsmath,amssymb,amsthm,stmaryrd,textcomp}
\usepackage[normalem]{ulem}
\usepackage{pgfpages}
\usepackage{utf8math}
\usepackage{url}
\usepackage{xspace}
\usepackage{fancyvrb,alltt}
\usepackage{graphicx}
\usepackage{color}
\definecolor{gray}{gray}{0.5}
\definecolor{lightgray}{rgb}{.9,.9,.9}
\definecolor{darkgray}{rgb}{.4,.4,.4}
\definecolor{purple}{rgb}{0.65, 0.12, 0.82}
\preparecolorset{rgb}{}{}{%
AliceBlue,.94,.972,1;%
AntiqueWhite,.98,.92,.844;%
Aqua,0,1,1;%
Aquamarine,.498,1,.83;%
Azure,.94,1,1;%
Beige,.96,.96,.864;%
Bisque,1,.894,.77;%
Black,0,0,0;%
BlanchedAlmond,1,.92,.804;%
Blue,0,0,1;%
BlueViolet,.54,.17,.888;%
Brown,.648,.165,.165;%
BurlyWood,.87,.72,.53;%
CadetBlue,.372,.62,.628;%
Chartreuse,.498,1,0;%
Chocolate,.824,.41,.116;%
Coral,1,.498,.312;%
CornflowerBlue,.392,.585,.93;%
Cornsilk,1,.972,.864;%
Crimson,.864,.08,.235;%
Cyan,0,1,1;%
DarkBlue,0,0,.545;%
DarkCyan,0,.545,.545;%
DarkGoldenrod,.72,.525,.044;%
DarkGray,.664,.664,.664;%
DarkGreen,0,.392,0;%
DarkGrey,.664,.664,.664;%
DarkKhaki,.74,.716,.42;%
DarkMagenta,.545,0,.545;%
DarkOliveGreen,.332,.42,.185;%
DarkOrange,1,.55,0;%
DarkOrchid,.6,.196,.8;%
DarkRed,.545,0,0;%
DarkSalmon,.912,.59,.48;%
DarkSeaGreen,.56,.736,.56;%
DarkSlateBlue,.284,.24,.545;%
DarkSlateGray,.185,.31,.31;%
DarkSlateGrey,.185,.31,.31;%
DarkTurquoise,0,.808,.82;%
DarkViolet,.58,0,.828;%
DeepPink,1,.08,.576;%
DeepSkyBlue,0,.75,1;%
DimGray,.41,.41,.41;%
DimGrey,.41,.41,.41;%
DodgerBlue,.116,.565,1;%
FireBrick,.698,.132,.132;%
FloralWhite,1,.98,.94;%
ForestGreen,.132,.545,.132;%
Fuchsia,1,0,1;%
Gainsboro,.864,.864,.864;%
GhostWhite,.972,.972,1;%
Gold,1,.844,0;%
Goldenrod,.855,.648,.125;%
Gray,.5,.5,.5;%
Green,0,.5,0;%
GreenYellow,.68,1,.185;%
Grey,.5,.5,.5;%
Honeydew,.94,1,.94;%
HotPink,1,.41,.705;%
IndianRed,.804,.36,.36;%
Indigo,.294,0,.51;%
Ivory,1,1,.94;%
Khaki,.94,.9,.55;%
Lavender,.9,.9,.98;%
LavenderBlush,1,.94,.96;%
LawnGreen,.488,.99,0;%
LemonChiffon,1,.98,.804;%
LightBlue,.68,.848,.9;%
LightCoral,.94,.5,.5;%
LightCyan,.88,1,1;%
LightGoldenrod,.933,.867,.51;%
LightGoldenrodYellow,.98,.98,.824;%
LightGray,.828,.828,.828;%
LightGreen,.565,.932,.565;%
LightGrey,.828,.828,.828;%
LightPink,1,.712,.756;%
LightSalmon,1,.628,.48;%
LightSeaGreen,.125,.698,.668;%
LightSkyBlue,.53,.808,.98;%
LightSlateBlue,.518,.44,1;%
LightSlateGray,.468,.532,.6;%
LightSlateGrey,.468,.532,.6;%
LightSteelBlue,.69,.77,.87;%
LightYellow,1,1,.88;%
Lime,0,1,0;%
LimeGreen,.196,.804,.196;%
Linen,.98,.94,.9;%
Magenta,1,0,1;%
Maroon,.5,0,0;%
MediumAquamarine,.4,.804,.668;%
MediumBlue,0,0,.804;%
MediumOrchid,.73,.332,.828;%
MediumPurple,.576,.44,.86;%
MediumSeaGreen,.235,.7,.444;%
MediumSlateBlue,.484,.408,.932;%
MediumSpringGreen,0,.98,.604;%
MediumTurquoise,.284,.82,.8;%
MediumVioletRed,.78,.084,.52;%
MidnightBlue,.098,.098,.44;%
MintCream,.96,1,.98;%
MistyRose,1,.894,.884;%
Moccasin,1,.894,.71;%
NavajoWhite,1,.87,.68;%
Navy,0,0,.5;%
NavyBlue,0,0,.5;%
OldLace,.992,.96,.9;%
Olive,.5,.5,0;%
OliveDrab,.42,.556,.136;%
Orange,1,.648,0;%
OrangeRed,1,.27,0;%
Orchid,.855,.44,.84;%
PaleGoldenrod,.932,.91,.668;%
PaleGreen,.596,.985,.596;%
PaleTurquoise,.688,.932,.932;%
PaleVioletRed,.86,.44,.576;%
PapayaWhip,1,.936,.835;%
PeachPuff,1,.855,.725;%
Peru,.804,.52,.248;%
Pink,1,.752,.796;%
Plum,.868,.628,.868;%
PowderBlue,.69,.88,.9;%
Purple,.5,0,.5;%
Red,1,0,0;%
RosyBrown,.736,.56,.56;%
RoyalBlue,.255,.41,.884;%
SaddleBrown,.545,.27,.075;%
Salmon,.98,.5,.448;%
SandyBrown,.956,.644,.376;%
SeaGreen,.18,.545,.34;%
Seashell,1,.96,.932;%
Sienna,.628,.32,.176;%
Silver,.752,.752,.752;%
SkyBlue,.53,.808,.92;%
SlateBlue,.415,.352,.804;%
SlateGray,.44,.5,.565;%
SlateGrey,.44,.5,.565;%
Snow,1,.98,.98;%
SpringGreen,0,1,.498;%
SteelBlue,.275,.51,.705;%
Tan,.824,.705,.55;%
Teal,0,.5,.5;%
Thistle,.848,.75,.848;%
Tomato,1,.39,.28;%
Turquoise,.25,.88,.815;%
Violet,.932,.51,.932;%
VioletRed,.816,.125,.565;%
Wheat,.96,.87,.7;%
White,1,1,1;%
WhiteSmoke,.96,.96,.96;%
Yellow,1,1,0;%
YellowGreen,.604,.804,.196}
\usepackage{hyperref}
\usepackage{rcs}

%% Macros.
\newcommand{\HAX}{\text{HACS}\xspace}
\newcommand{\CRSX}{\text{CRSX}\xspace}
\newcommand{\ie}{\textit{i.e.}\xspace}
\newcommand{\eg}{\textit{e.g.}\xspace}
\newcommand{\etc}{\textit{etc.}\xspace}
\newcommand{\aka}{\textit{aka.}\xspace}
\newcommand{\TBD}[1][]{\textit{To Be Done…#1}\marginpar{\bf TBD}}

\newcommand*{\cf}[1]{\texttt{\small #1}}
\newcommand*{\si}[1]{\textit{\small #1}}
\newcommand*{\surl}[1]{\mbox{\small\url{#1}}}

\newcommand{\http}[1]{\href{http://#1}{\emph{#1}}}
\newcommand{\mailto}[1]{\href{mailto:#1}{\emph{#1}}}

\newcommand{\qqquad}{{\qquad\quad}}
\newcommand{\qqqquad}{{\qquad\qquad}}
\newcommand{\opt}{{\ensuremath{_{\text{\it opt}}}}}

\newcommand\caret{\mathbin{\char`\^}}

\def\SubScript{\textit{SubScript}\xspace}
\def\vacuum{\endgraf\vskip 0pt plus 1filll\relax}
\def\tup#1{\ensuremath{\left\langle{#1}\right\rangle}}
\def\mc#1{\fbox{\ensuremath{#1}}}

\def\first{\textsc{first}}
\def\follow{\textsc{follow}}
\def\tup#1{\ensuremath{\left\langle{#1}\right\rangle}}
\def\q#1{\,\text{``#1''}\,}
\def\qm#1{\,\text{``\ensuremath{#1}''}\,}
\def\t#1{\textbf{#1}}
\def\nt#1{{\textit{#1}}}

\def\alertonly#1#2#3{\alert<#1>{\only<#2>{#3}}}
\def\alertthen#1#2#3{\alert<#1>{\uncover<#2>{#3}}}

%% WORKING \Cases{...} (based on \cases from plain TeX -- the one in
%% AMS-LaTeX is seriously buggy):
\makeatletter
\def\Cases#1{\mbox{$
  \left\{\,\vcenter{\let\\=\cr \normalbaselines\m@th
    \ialign{$\vphantom(##\hfil$&\quad##\hfil\crcr#1\crcr}}\right.$}}
\makeatother

%% Environments.

%% Comments.
\RequirePackage{verbatim}
\def\ignoreenvironment#1{%
  \expandafter\let\csname #1\endcsname=\comment
  \expandafter\let\csname end#1\endcsname=\endcomment}

%% Problem/solution...
\def\setupdefault{\ignoreenvironment{solution}}
\def\setupsolutions{\newenvironment{solution}[1][Answer]{\color{gray}\proof[##1]}{\endproof}}
\def\afterdash#1-#2-#3@@{#2}
\csname setup\expandafter\afterdash\jobname-default-@@\endcsname
%\expandafter\def\expandafter\setupafterdashjobcs\expandafter{\csname setup\expandafter\afterdash\jobname-default-@@\endcsname}
%\setupafterdashjobcs

%% Old code listings.
\usepackage{fancyvrb}
\CustomVerbatimEnvironment{code}{Verbatim}{tabsize=4,fontsize=\small,numbers=left,numberblanklines=false,xleftmargin=\parindent}
\CustomVerbatimCommand{\inputcode}{VerbatimInput}{tabsize=4,fontsize=\footnotesize,numbers=left,numberblanklines=false,xleftmargin=\parindent}
%\DefineShortVerb{\"} % makes " work in verbatim even when \active

%% New code listings...
\usepackage{listings}
\lstset{basicstyle=\ttfamily}

%% HACS.
\def\Identifier#1{#1}
\def\Keyword#1{\textsf{\bf#1}}

\lstdefinelanguage{HACS}{basicstyle=\fontfamily{ccr}\selectfont,%
  extendedchars=true,inputencoding=utf8,sensitive,%
  identifierstyle=\Identifier,%
  string=[d]{"},morestring=[d]{'},upquote=true,stringstyle=\ttfamily,showstringspaces=true,%
  comment=[l]{//},morecomment=[s]{/*}{*/},commentstyle=\color{blue}\rm,%
  keywordstyle=\Keyword,%
  keywords={attribute,data,fragment,import,module,nested,property,rule,scheme,simplify,sort,%
    space,static,sugar,symbol,tag,token,default,error},%
  literate=%
    {⟦}{{\ensuremath{\llbracket}}}1 {⟧}{{\ensuremath{\rrbracket}}}1 {⟨}{{\ensuremath{\langle}}}1 {⟩}{{\ensuremath{\rangle}}}1 %
    {→}{{\ensuremath{\rightarrow}}}2 {↑}{{\ensuremath{\uparrow}}}1 {↓}{{\ensuremath{\downarrow}}}1 {¬}{{\ensuremath{\lnot}}}2 {¶}{{\P}}1 %
    {ε}{{\ensuremath{\varepsilon}}}1 {…}{{\ensuremath{\dots}}}2 {∧}{{\ensuremath{\!\wedge}}}2 {∨}{{\ensuremath{\!\vee}}}2 {↦}{{\ensuremath{\mapsto}}}2 %
    {¹}{{\ensuremath{^1}}}1 {²}{{\ensuremath{^2}}}2 %
}[comments,strings,keywords]
%%
\lstnewenvironment{hacs}[1][]{\lstset{language=HACS,basewidth={.55em,.45em},columns=flexible,#1}\upshape}{}
\newcommand\inputhacs[2][]{\lstinputlisting[language=HACS,basewidth={.55em,.45em},columns=flexible,#1]{#2}}
\newcommand\hacsc{\lstinline[language=HACS]}
%\lstMakeShortInline[language=HACS,columns=fullflexible]"

%% Tiger.
\lstdefinelanguage{Tiger}{basicstyle=\normalsize,%
  extendedchars=true,inputencoding=utf8,sensitive,%
  identifierstyle=\textit,%
  string=[d]{"},upquote=true,stringstyle=\ttfamily,showstringspaces=true,%
  comment=[s]{/*}{*/},commentstyle=\color{gray}\rm,%
  keywordstyle=\bfseries,%
  keywords={type,array,of,var,function,nil,let,in,end,if,then,else,break,while,do},%
  literate=%
    {≡}{{\ensuremath{\equiv}}}1 %
}[comments,strings,keywords]
%%
\lstnewenvironment{tiger}[1][]{\lstset{language=Tiger,basewidth={.5em,.35em},columns=flexible,#1}\upshape}{}
\newcommand\inputtiger[2][]{\lstinputlisting[language=Tiger,basewidth={.5em,.35em},columns=flexible,#1]{#2}}
\newcommand\tigerc{\lstinline[language=Tiger]}

\lstdefinelanguage{JavaScript}{
  keywords={break, case, catch, continue, debugger, default, delete, do, else, false, finally, for, function, if, in, instanceof, new, null, return, switch, this, throw, true, try, typeof, var, while, with, class, export, throw, implements, import, this, extends},
  morecomment=[l]{//},
  morecomment=[s]{/*}{*/},
  morestring=[b]',
  morestring=[b]",
  ndkeywords={boolean, any, number, string, void},
  keywordstyle=\color{blue}\bfseries,
  ndkeywordstyle=\color{darkgray}\bfseries,
  identifierstyle=\color{black},
  commentstyle=\color{purple}\ttfamily,
  stringstyle=\color{red}\ttfamily,
  sensitive=true
}

\lstnewenvironment{JavaScript}[1][]{\lstset{
   language=JavaScript,
   backgroundcolor=\color{lightgray},
   extendedchars=true,
   basicstyle=\ttfamily,
   showstringspaces=false,
   showspaces=false,
   numbers=left,
   numberstyle=\scriptsize,
   numbersep=7pt,
   numberblanklines=false,
   tabsize=2,
   breaklines=true,
   showtabs=false,
   captionpos=b,
   #1
}\upshape}{}

\newcommand\inputJavaScript[2][]{\lstset{
   language=JavaScript,
   backgroundcolor=\color{lightgray},
   extendedchars=true,
   basicstyle=\ttfamily,
   showstringspaces=false,
   showspaces=false,
   numbers=left,
   numberstyle=\scriptsize,
   numbersep=7pt,
   tabsize=2,
   breaklines=true,
   showtabs=false,
   captionpos=b,
   #1
}{#2}}

\newcommand\JS[1][]{\lstinline[
   language=JavaScript,
   backgroundcolor=\color{lightgray},
   extendedchars=true,
   basicstyle=\small\ttfamily,
   showstringspaces=false,
   showspaces=false,
   #1]}

%% Identifiers
\newcommand{\set}[1]{\textit{#1}}
\newcommand{\op}[1]{\ensuremath{\operatorname{#1}}}

\endinput

%% Last to pickup right catcodes...
\usepackage[graph,arrow,curve,color,tips,frame]{xy}
\def\mtxt#1{\hbox{\footnotesize$\begin{array}{l}#1\end{array}$}}

\let\next\relax %workaround...

%% T diagram helpers
\def\teediagram#1{\xygraph{!{(0.6,0):}#1}}
\newgraphescape{I}#1#2#3#4{
  []*+[#4]{#1}="I#1#2#3..s"
  [d]*+[#4]{#2}="I#1#2#3..m"
  (
  []!{ "I#1#2#3..m"."I#1#2#3..s"="I#1#2#3.m";
       "I#1#2#3..s"."I#1#2#3..m"="I#1#2#3.s" }
    , "I#1#2#3.s"!{+UL}
    -@[#4] "I#1#2#3.s"!{+UR}
    -@[#4] "I#1#2#3.s"!{+DR}
    -@[#4] "I#1#2#3.s"!{+DL}
    -@[#4] "I#1#2#3.s"!{+UL}
  )
}
\newgraphescape{C}#1#2#3#4#5{
  []*+[#5]{#1}="C#1#2#3#4..s" :@[#5] [rr] *+[#5]{#2}="C#1#2#3#4..t"
  [dl]*+[#5]{#3}="C#1#2#3#4.m"
  (
    []!{ "C#1#2#3#4..s"."C#1#2#3#4..t"="C#1#2#3#4.s";
         "C#1#2#3#4..t"."C#1#2#3#4..s"="C#1#2#3#4.t" }
    , "C#1#2#3#4.s"!{+UL}
    -@[#5] "C#1#2#3#4.s"!{+UR}
    -@[#5] "C#1#2#3#4.s"!{+DR}
    -@[#5] "C#1#2#3#4.m"!{+UR}
    -@[#5] "C#1#2#3#4.m"!{+DR}
    -@[#5] "C#1#2#3#4.m"!{+DL}
    -@[#5] "C#1#2#3#4.m"!{+UL}
    -@[#5] "C#1#2#3#4.s"!{+DL}
    -@[#5] "C#1#2#3#4.s"!{+UL}
  )
}
\newgraphescape{W}#1#2#3#4#5{
  []*+[#5]{#1}="I#1#2#3#4..s" :[rrrr] *+[#5]{#2}="I#1#2#3#4..t"
  [dll]*+[#5]{#3}="I#1#2#3#4.m"
  !{ "I#1#2#3#4..s"."I#1#2#3#4..t"="I#1#2#3#4.t"
     ."I#1#2#3#4..s"="I#1#2#3#4.s" }
  ( "I#1#2#3#4.s"!{+UL}
    -@[#5] "I#1#2#3#4.s"!{+UR}
    -@[#5] "I#1#2#3#4.s"!{+DR}
    -@[#5] "I#1#2#3#4.m"!{+UR}
    -@[#5] "I#1#2#3#4.m"!{+DR}
    -@[#5] "I#1#2#3#4.m"!{+DL}
    -@[#5] "I#1#2#3#4.m"!{+UL}
    -@[#5] "I#1#2#3#4.s"!{+DL}
    -@[#5] "I#1#2#3#4.s"!{+UL}
  )
}

\newcommand{\circled}[1]{\text{\scriptsize$\vcenter{\xygraph{*+[o][F-]{#1}}}$}}
\newcommand{\Circled}[1]{\text{\scriptsize$\vcenter{\xygraph{*+[o][F=]{#1}}}$}}


%% MACROS.

%%\def\nt#1{\text{\rm\normalfont\emph{#1}}}
\def\SubScript{\textit{SubScript}\xspace}

%\usepackage{colortbl}
%\usepackage{amssymb}
%\usepackage{wrapfig}
%\usepackage{stmaryrd} 
%\usepackage{colortbl}
%\usepackage{soul}
%\usepackage{graphicx}

\newcommand{\arrz}{\rightarrow}
%\newcommand{\abs}[2]{\lambda #1.#2}
\newcommand{\abs}[2]{#1.#2}
\newcommand{\meta}[2]{#1(#2)}

\newcommand{\avar}{x}
\newcommand{\bvar}{y}
\newcommand{\cvar}{z}
\newcommand{\dvar}{a}
\newcommand{\afun}{\symb{f}}
\newcommand{\bfun}{\symb{g}}
\newcommand{\cfun}{h}
\newcommand{\aterm}{s}
\newcommand{\bterm}{t}
\renewcommand{\ell}{l}

\newcommand{\symb}[1]{\mathtt{#1}}
\newcommand{\append}{\symb{append}}
\newcommand{\nil}{\symb{nil}}
\newcommand{\cons}{\symb{cons}}
\newcommand{\map}{\symb{map}}
\newcommand{\lijst}{\symb{list}}
\newcommand{\Int}{\symb{int}}
\newcommand{\suc}{\symb{s}}
\newcommand{\nul}{\symb{0}}
\newcommand{\plus}{\symb{plus}}

\definecolor{darkgreen}{RGB}{0,155,0}
\newcommand{\DARKBLUE}[1]{{\color{blue}#1}}   % FEMKE
\newcommand{\BLUE}[1]{{\usebeamercolor{structure}\color{fg}#1}}
\newcommand{\GREEN}[1]{{\color{darkgreen}#1}}
\newcommand{\RED}[1]{{\color{red}#1}}
\newcommand{\BROWN}[1]{{\color{brown}#1}}

\newcommand{\opx}[1]{\operatorname{\textcolor{blue}{#1}}}

%% STYLE.
\usepackage[T1]{fontenc}
\usepackage{charter}
\renewcommand{\ttdefault}{txtt}
\usefonttheme{serif}
\SelectTips{lu}{10 scaled 2074}
\def\q#1{~\text{`\texttt{#1}'}~}
\def\t#1{\texttt{#1}}
\def\tup#1{\ensuremath{\left\langle{#1}\right\rangle}}
\useinnertheme{rounded}
\SelectTips{lu}{10 scaled 2074}

%% Topmatter
\title{Experience with Higher Order Rewriting\\from the\\Compiler Teaching Trenches}
\author[Kris Rose]{Kristoffer H. Rose\\Two Sigma \emph{Labs} \& NYU}
\date{Saturday, July 12, 2014\\HOR 2014}

%%
\begin{document}
\frame{\titlepage}

%%\begin{frame}{Outline}
%%  \tableofcontents
%%\end{frame}

%%%===========================================

\section{Compilers}
\begin{frame}{Outline}
  \tableofcontents[current]
\end{frame}

\begin{frame}{Compilers and Teaching}
  \begin{itemize}

  \item Programming \structure{Culture} \pause bordering on \alert{Art}.\pause

  \item Masterwork is ``Dragon Book.''\pause

  \item We teach \structure{semantic techniques} \\\pause%
    ---but abandon them to \alert{hand-written algorithmic code}.\pause

  \item Little compiler programmer support after parser generation \\\pause%
    ---but tools are emerging (the \structure{meta-programming} trend).
    
  \item HACS aims to change this, and provide \structure{high level
      support for full compiler programming experience}.\pause

  \end{itemize}
\end{frame}

\begin{frame}[fragile,shrink]{Compiler Structure}
  \vspace*{-1em}
  \begin{displaymath}
    \xygraph{!{;(.66,0):}
      []*+\txt{source program}
      :@(l,l)  [d]*+[r][F:green]\txt{Lexical Analysis}="L"
      :@(l,l) _{\txt{Tokens}} [d]*+[r][F:green]\txt{Syntax Analysis}="P"
      :@(l,l) _{\txt{Tree}}  [d]*+[r][F:red]\txt{Semantic Analysis}="A"
      :@(l,l) _{\txt{Tree}} [d]*+[r][F:red]\txt{Intermediate Representation Generator}="IR"
                              ( "P"[r(7)]*+[F:red]\txt{Symbol\\Table}
                                ( -@. "L"!{+R} , -@. "P"!{+R} , -@. "A"!{+R} , -@. "IR"!{!C} ) )
      :@(l,l) _{\txt{IR}} [d]*+[r][F:red]\txt{Optimizer}
      :@(l,l) _{\txt{IR}} [d]*+[r][F:red]\txt{Code Generator}
      :@(l,l) _{\txt{ASM}} [d]*+[r][F:red]\txt{Machine-Dependent Code Optimizer}
      :@(l,l)  [d]*+\txt{target machine code}
    }
  \end{displaymath}
\end{frame}

\begin{frame}{What Formalizations are we Teaching?}
  \begin{description}
  \item[Lexical Analysis.] \textcolor{green}{Regular Expressions.}\pause
  \item[Syntax Analysis.] \textcolor{green}{LL/LALR Parser Generators.}\pause
  \item[Semantic Analysis.] \alert{Attribute Grammars.}\pause
  \item[IR Generator.] \alert{Translation Schemes.}\pause
  \item[Optimizer.] \alert{Attribute Grammars, Translation Schemes, Custom Algorithms, \dots}\pause
  \item[Code Generator.] \alert{Translation Schemes.}\pause
  \item[Peep-hole optimizer.] \alert{All bets are off\dots}\pause\\[1em]

  \item[Symbol Table.] \alert{Side effects and explicit scope structures.}
  \end{description}
\end{frame}

\begin{frame}{HACS---\emph{Formally Founded} Compiler Generator}
  \begin{enumerate}
  \item Designed to directly support \structure{existing formal notations}.
  \item Formally defined in terms of a ``core'' language, which is in fact CRSX.
  \item Integrated with special purpose \structure{compiler support algorithms}.
  \item Uses \structure{rewriting} for recursive translation schemes.
  \end{enumerate}
  \pause
  \vspace*{1em}
  \alert{--- with attribute dependency controls}
\end{frame}

%%%===========================================

\section{Attribute Grammars}
\frame{\tableofcontents[current]}

\begin{frame}{On Attribute Grammars}
  \begin{itemize}
  \item Based on Knuth's seminal 1968 ``Semantics of Context-Free Languages''
  \item Dominate teaching in compiler construction
  \item Can be adapted to imperative programming (\texttt{\$} in \texttt{yacc})
  \item Compatible with rewriting, can be implemented with rewriting (ASF+SDF)
  \end{itemize}
\end{frame}

\begin{frame}[shrink]{``Syntax-Directed Definition''}
  \begin{displaymath}
    \begin{array}{r@{\,}l|lr}
      \hline
      \hline
      \multicolumn{2}{c|}{\textsc{Production}}  & \textsc{Semantic Rules} &\vphantom{\bigm|}\\
      \hline\vphantom{\Bigm|}
      S &→ \textbf{name} := E_1; S_2
      & E_1.e = S.e; S_2.e = \op{extend}(S.e, \textbf{name}.sym, E_1.t); &(S1) \\[-1pt]
      && \quad S.ok = \op{IsType}(E_1.t) ∧ S_2.ok
      \\[\jot]
      &~\mid \{~S_1~\}~S_2 & S_1.e = S.e; S_2.e = S.e; S.ok = S_1.ok ∧ S_2.ok &(S2)
      \\
      &~\mid \op{printInt}~E_1; S_2 & E_1.e = S.e; S.ok=\op{Check}(E_1.t=\op{Int}) ∧ S_2.ok &(S3)
      \\
      &~\mid \op{printFloat}~E_1; S_2 & E_1.e = S.e; S.ok=\op{Check}(E_1.t=\op{Float}) ∧ S_2.ok &(S4)
      \\
      &~\mid ε & S.ok = \op{True} &(S5)
      \\[1ex]
      E &→ E_1 + E_2 & E_1.e=E.e; E_2.e=E.e; E.t = \op{Unif}(E_1.t, E_2.t) &(E1)\\
      &~\mid E_1 \ast E_2 & E_1.e=E.e; E_2.e=E.e; E.t = \op{Unif}(E_1.t, E_2.t) &(E2)\\
      &~\mid \textbf{int} & E.t = \op{Int}&(E3)\\
      &~\mid \textbf{float} & E.t = \op{Float}&(E4)\\
      &~\mid \textbf{name} & E.t = \text{if}~\op{defined}(E.e,\textbf{name}.sym)\\[-1pt]
      && \qquad\quad\text{then}~\op{lookup}(E.e,\textbf{name}.sym)&(E5)\\[-1pt]
      && \qquad\quad\text{else} \op{TypeErr}\\
      \hline
    \end{array}
  \end{displaymath}
\end{frame}

\begin{frame}[fragile]{``\texttt{x:=1+2; printInt x}''}
  \vspace*{-2em}
  \begin{displaymath}
    \xygraph{!{(1.2,0):(0,.9)::}
      []{S_1\colon{:=}}
      ( -[dlll]{x}
      , -[dl]{E_1\colon{+}}
        ( -[ddl]{E_2\colon1}
        , -[ddr]{E_3\colon2}
        )
      , -[drrr]{S_2\colon\op{printInt}}
        ( -[ddl]{E_4\colon x}
        , -[ddr]{S_3\colon ε}
        )
      )
    }
  \end{displaymath}
\end{frame}

\begin{frame}[fragile]{``\texttt{x:=1+2; printInt x}'' with Attributes}
  \vspace*{-2em}
  \begin{displaymath}
    \xygraph{!{(1.2,0):(0,.9)::}
      []{S_1\colon{:=}}
      (-@.[ur(.6)]*+{\scriptstyle S_1.e=\{\};~S_1.ok=\op{IsType}(E_1.t)∧S_2.ok=\uncover<6->{\opx{True}}}="S1")
      ( -[dlll]{x}
      , -[dl]{E_1\colon{+}}
        (-@.[dr(.7)]*+[r]{\scriptstyle E_1.t=\op{Unif}(E_2.t,E_3.t)\uncover<2->{=\opx{Int}}}="E1")
        ( -[ddl]{E_2\colon1}   (-@.[d(.8)]*+{\scriptstyle E_2.t=\op{Int}}="E2")
        , -[ddr]{E_3\colon2}   (-@.[d(.8)]*+{\scriptstyle E_3.t=\op{Int}}="E3")
        )
      , -[drrr]{S_2\colon\op{printInt}}
        (-@.[ur(.9)]*+\txt{$\scriptstyle S_2.e=\op{extend}(S_1.e,x,E_1.t)=\uncover<3->{\{x↦\opx{Int}\}}$\\
                           $\scriptstyle S_2.ok=\op{Check}(E_4.t=\op{Int})∧S_3.ok=\uncover<5->{\opx{True}}$}="S2")
          ( -[ddl]{E_4\colon x}
            (-@.[d(.8)]*+\txt{$\scriptstyle E_4.e=S_2.e=\uncover<4->{\{x↦\opx{Int}\}}$\\
                              $\scriptstyle E_4.t=\op{lookup}(E_4.e,x)=\uncover<4->{\opx{Int}}$}="E4")
        , -[ddr]{S_3\colon ε}  (-@.[d(.8)]*+{\scriptstyle {S_3.ok=\op{True}}}="S3")
        )
      )
    }
  \end{displaymath}
\end{frame}

\begin{frame}[fragile]{``\texttt{x:=1+2; printInt x}'' with Attributes}
  \vspace*{-2em}
  \begin{displaymath}
    \xygraph{!{(1.2,0):(0,.9)::}
      []{S_1\colon{:=}}
      (-@.[ur(.6)]*+{\scriptstyle S_1.e=\{\};~S_1.ok=\op{IsType}(E_1.t)∧S_2.ok={\opx{True}}}!{!C}="S1")
      ( -[dlll]{x}
      , -[dl]{E_1\colon{+}}
        (-@.[dr(.7)]*+[r]{\scriptstyle E_1.t=\op{Unif}(E_2.t,E_3.t){=\opx{Int}}}!{!C}="E1")
        ( -[ddl]{E_2\colon1}   (-@.[d(.8)]*+{\scriptstyle E_2.t=\op{Int}}!{!C}="E2")
        , -[ddr]{E_3\colon2}   (-@.[d(.8)]*+{\scriptstyle E_3.t=\op{Int}}!{!C}="E3")
        )
      , -[drrr]{S_2\colon\op{printInt}}
        (-@.[ur(.9)]*+\txt{$\scriptstyle S_2.e=\op{extend}(S_1.e,x,E_1.t)={\{x↦\opx{Int}\}}$\\
                           $\scriptstyle S_2.ok=\op{Check}(E_4.t=\op{Int})∧S_3.ok={\opx{True}}$}="S2")
          ( -[ddl]{E_4\colon x}  (-@.[d(.8)]*+{\scriptstyle {E_4.t=\op{lookup}(S_2.e,x)}={\opx{Int}}}!{!C}="E4")
        , -[ddr]{S_3\colon ε}  (-@.[d(.8)]*+{\scriptstyle {S_3.ok=\op{True}}}!{!C}="S3")
        )
      )
      "E2"!{!UR} :@.@(ur,l)@[red] "E1"!{!L}
      "E3"!{!UR} :@.@(ur,dr)@[red] "E1"!{!L}
      "E1" :@.@(u,l)@[red] "S2"!{!L!/u1ex/}
      "S2"!{!R!/u1ex/} -@.@(r,ur)@[red] [d(.8)]!{+/r1ex/}*{} :@.@(dl,u)@[red] "E4"
      "E4" :@.@(ul,l)@[red] "S2"!{!L!/d1ex/}
      "S3" :@.@(ul,l)@[red] "S2"!{!L!/d1ex/}
      "S2"!{!R!/d1ex/} -@.@(r,r)@[red] [ul(.8)]!{+/r1ex/}*{} :@.@(l,r)@[red] "S1"!{!R}
    }
  \end{displaymath}
\end{frame}

\begin{frame}{Attribute Propagation Patterns}
  \begin{itemize}

  \item Synthesized attributes \alert{mostly} depend only on synthesis.\\
    (An AG is ``\structure{S-attributed}'' if this is exclusively the case.)

  \item Cross-synthesized-inherited attributes depend \alert{left to right}.\\
    (It is ``\structure{L-attributed},'' like inference systems.)

  \end{itemize}
\end{frame}

%%%===========================================

\section{HACS by Example}
\frame{\tableofcontents[current]}

\begin{frame}{HACS Features}
  \begin{itemize}
  \item Parser generation notation.\\
    \structure{Supports Higher-Order Abstract Syntax.}
  \item Rules for \structure{synthesis}.
  \item Rules for \structure{recursive compilation schemes}.
  \item Polymorphic \structure{sort system}.
  \item All rules and sorts use \structure{native user syntax}.
  \end{itemize}
\end{frame}

\begin{frame}{Words, words, words…}
  \begin{center}
    \alert{Let's code.}
  \end{center}
\end{frame}

\begin{frame}{Dependency Games}
  \begin{itemize}
  \item Synthesized attributes \structure{tied to AST node sorts}\\\pause%
    ---give rules for \alert{propagation per node form}.\pause
  \item Inherited attributes \structure{tied to recursive schemes}\\\pause%
    --- give rules for \alert{distribution per scheme}.
  \end{itemize}
\end{frame}

\begin{frame}[fragile]{Synthesis Propagation of $t$}
  \begin{displaymath}
    \begin{array}{r@{\,}l|lr}
      \hline\strut
      E &→ E_1 + E_2 & E_1.e=E.e; E_2.e=E.e; E.t = \op{Unif}(E_1.t, E_2.t) \qquad&\thetag{E1}\\[\jot]
      \hline
    \end{array}
  \end{displaymath}
  \pause
\begin{hacs}
attribute ↑t(Type);
sort Exp | ⟦ ⟨Exp⟩ + ⟨Exp⟩ ⟧ | ↑t ;
⟦ ⟨Exp#1 ↑t(#t1)⟩ + ⟨Exp#2 ↑t(#t2)⟩ ⟧ ↑t(Unif(#t1,#t2)) ;
sort Type | scheme Unif(Type,Type);
…
\end{hacs}
\end{frame}

\begin{frame}[fragile]{Inherited Distribution of $e$}
  \begin{displaymath}
    \begin{array}{r@{\,}l|lr}
      \hline\strut
      E &→ E_1 + E_2 & E_1.e=E.e; E_2.e=E.e; E.t = \op{Unif}(E_1.t, E_2.t) \qquad&\thetag{E1}\\[\jot]
      \hline
    \end{array}
  \end{displaymath}
  \pause
\begin{hacs}
attribute ↓e{Name:Type};
sort Exp | scheme ⟦ TA ⟨Exp⟩⟧ ↓e ;
⟦ TA (⟨Exp#1⟩ + ⟨Exp#2⟩) ⟧
  → ⟦ (TA ⟨Exp#1⟩) + (TA ⟨Exp#2⟩) ⟧;
\end{hacs}
\end{frame}

\begin{frame}[fragile]{Left-to-Right Dependency $t$ to $e$}
  \vspace*{-1em}
  \begin{displaymath}
    \begin{array}{r@{\,}l|l}
      \hline\strut
      S &→ \textbf{name} := E_1; S_2
      & E_1.e = S.e; \\
      && S_2.e = \op{extend}(S.e, \textbf{name}.sym, E_1.t); \\
      && S.ok = \op{IsType}(E_1.t) ∧ S_2.ok\\
      \hline
    \end{array}
  \end{displaymath}
  \pause\footnotesize
\begin{hacs}
attribute ↓e{Name:Type};
sort Stat | ⟦ ⟨[x:Name]⟩ := ⟨Exp⟩ ; ⟨Stat[x:Exp]⟩ ⟧ | scheme ⟦ TA { ⟨Stat⟩ } ⟧ ↓e ;

⟦ TA { id := ⟨Exp#1⟩; ⟨Stat#2[⟦id⟧]⟩ } ⟧ → ⟦ TA2 { id := TA ⟨Exp#1⟩; ⟨Stat#2[⟦id⟧]⟩ } ⟧;
{
  | scheme ⟦ TA2 { ⟨Stat⟩ } ⟧ ↓e;
  ⟦ TA2 { id := ⟨Exp#1 ↑t(#t1)⟩; ⟨Stat#2[⟦id⟧]⟩ } ⟧
    → ⟦ id := ⟨Exp#1⟩; ⟨Stat ⟦TA {⟨Stat#2[⟦id⟧]⟩}⟧ ↓e{⟦id⟧:#t1}⟩ ⟧;
}
\end{hacs}
\end{frame}

\begin{frame}[fragile]{HACS Program (I)}\footnotesize
\begin{hacs}[texcl]
module "edu.nyu.cims.cc.Check" {

// PARSER.

space [ \t\n] ;
token INT      | [0-9]+ ;
token FLOAT    | [0-9]+ [.] [0-9]* ([Ee] [-+]? [0-9]+)? ;
token ID       | [a-z] [a-zA-Z0-9_]* ;
sort Name | symbol ⟦⟨ID⟩⟧ ;

sort S  | ⟦ ⟨[x:Name]⟩ := ⟨E⟩; ⟨S[x:Name]⟩ ⟧ | ⟦ { ⟨S⟩ } ⟨S⟩ ⟧
        | ⟦ printInt ⟨E⟩; ⟨S⟩ ⟧ | ⟦ printFloat ⟨E⟩; ⟨S⟩ ⟧
        | ⟦⟧ ;

sort E  | ⟦ ⟨E⟩ + ⟨E@1⟩ ⟧ | ⟦ ⟨E@1⟩ * ⟨E@2⟩ ⟧@1
        | ⟦ ⟨INT⟩ ⟧@2 | ⟦ ⟨FLOAT⟩ ⟧@2 | ⟦ ⟨Name⟩ ⟧@2
        | sugar ⟦ ( ⟨E#1⟩ ) ⟧@2 → E#1 ;
\end{hacs}
\end{frame}

\begin{frame}[fragile]{HACS Program (II)}\footnotesize
\begin{hacs}[texcl]
// SEMANTIC SORTS \& HELPERS.

sort Bool | ⟦T⟧ | ⟦F⟧;
| scheme ⟦⟨Bool⟩∧⟨Bool⟩⟧;  ⟦T∧T⟧ → ⟦T⟧;  default ⟦⟨Bool#1⟩∧⟨Bool#2⟩⟧ → ⟦F⟧;

sort Type | Int | Float | TypeErr;
| scheme Unif(Type,Type);
Unif(Int, Int) → Int;      Unif(Int, Float) → Float;
Unif(Float, Int) → Float;  Unif(Float, Float) → Float;
default Unif(#1, #2) → TypeErr ;

sort Bool;
| scheme IsType(Type);  IsType(Int) → ⟦T⟧;  IsType(Float) → ⟦T⟧;  IsType(TypeErr) → ⟦F⟧;
| scheme CheckInt(Type);  CheckInt(Int) → ⟦T⟧;  default CheckInt(#) → ⟦F⟧;
| scheme CheckFloat(Type);  CheckFloat(Float) → ⟦T⟧;  default CheckFloat(#) → ⟦F⟧;
\end{hacs}
\end{frame}

\begin{frame}[fragile]{HACS Program (III)}\footnotesize
\begin{hacs}[texcl]
// TYPE CHECKING.

sort CheckResult | ⟦Yes⟧;
| scheme Check(S) ;
Check(#s↑ok(⟦T⟧)) → ⟦Yes⟧ ;
Check(#s↑ok(⟦F⟧)) → error⟦Type Error.⟧ ;

// Attributes.
attribute ↑ok(Bool);
attribute ↑t(Type);
attribute ↓e{Name:Type};

// Rules for $S$ sort:
// -- Synthesizes attribute $ok$.
// -- Inherited attribute $S.e$ is distributed by Te scheme (and helper Te2).
sort S | ↑ok | scheme ⟦Te ⟨S⟩⟧ ↓e | scheme ⟦Te2 ⟨S⟩⟧ ↓e ;
\end{hacs}
\end{frame}

\begin{frame}[fragile]{HACS Program (IV)}\footnotesize
\begin{hacs}[texcl]
// $(S1)\quad S→\textbf{name}:=E_1;S_2$ has three stages:
// 1. $E_1.e = S.e$ and recurse over $E_1$ (to get automatic propagation).
⟦ Te x := ⟨E#1⟩; ⟨S#2[x]⟩ ⟧  →  ⟦ Te2 x := Te⟨E#1⟩; ⟨S#2[x]⟩ ⟧ ;
// 2. When $E_1.t$ is available then $S_2.e = \op{extend}(S.e, \textbf{name}.sym, E_1.t)$ and recurse over $S_2$.
⟦ Te2 x := ⟨E#1↑t(#t1)⟩; ⟨S#2[x]⟩ ⟧  →  ⟦ x := ⟨E#1⟩; Te⟨S#2[x]↓e{x:#t1}⟩ ⟧ ;
// 3. When $S_2.ok$ is (also) available then we can synthesize $S.ok = \op{IsType}(E_1.t) ∧ S_2.ok$.
⟦ x := ⟨E#1↑t(#t1)⟩; ⟨S#2[x]↑ok(#ok2)⟩ ⟧ ↑ok(⟦⟨Bool IsType(#t1)⟩∧⟨Bool#ok2⟩⟧) ;

// $(S2)\quad S→\{S_1\}\,S_2$:
// 1. $S_1.e=S.e$ and $S_2.e=S.e$ are propagated and recursed over.
⟦ Te { ⟨S#1⟩ } ⟨S#2⟩ ⟧  →  ⟦ { Te⟨S#1⟩ } Te⟨S#2⟩ ⟧ ;
// 2. When $S_1.ok$ and $S_2.ok$ are available then synthesize $S.ok=S_1.ok∧S_2.ok$.
⟦ { ⟨S#1↑ok(#ok1)⟩ } ⟨S#2↑ok(#ok2)⟩ ⟧ ↑ok(⟦⟨Bool#ok1⟩ ∧ ⟨Bool#ok2⟩⟧) ;
\end{hacs}
\end{frame}

\begin{frame}[fragile]{HACS Program (V)}\footnotesize
\begin{hacs}[texcl]
// $(S3)\quad S→\op{printInt}\,E_1; S_2$:
// 1. $E_1.e=S.e$ and $S_2.e=S.e$ are propagated and recursed over.
⟦ Te printInt ⟨E#1⟩; ⟨S#2⟩ ⟧  →  ⟦ printInt Te⟨E#1⟩; Te⟨S#2⟩ ⟧ ;
// 2. When $S_1.t$ and $S_2.ok$ are available then synthesize $S.ok=\op{Check}(\op{Int}=E_1.t) ∧ S_2.ok$.
⟦ printInt ⟨E#1↑t(#t1)⟩; ⟨S#2↑ok(#ok2)⟩ ⟧ ↑ok(⟦ ⟨Bool CheckInt(#t1)⟩ ∧ ⟨Bool#ok2⟩ ⟧) ;

// $(S4)\quad S→\op{printFloat}\,E_1; S_2$:
// 1. $E_1.e=S.e$ and $S_2.e=S.e$ are propagated and recursed over.
⟦ Te printFloat ⟨E#1⟩; ⟨S#2⟩ ⟧  →  ⟦ printFloat Te⟨E#1⟩; Te⟨S#2⟩ ⟧ ;
// 2. When $S_1.t$ and $S_2.ok$ are available then synthesize $S.ok=\op{Check}(\op{Float}=E_1.t) ∧ S_2.ok$.
⟦ printFloat ⟨E#1↑t(#t1)⟩; ⟨S#2↑ok(#ok2)⟩ ⟧ ↑ok(⟦ ⟨Bool CheckFloat(#t1)⟩ ∧ ⟨Bool#ok2⟩ ⟧) ;

// $(S5)\quad S→ε$: No distribution necessary; synthesize $S.ok=\op{True}$.
⟦ Te ⟧  →  ⟦⟧ ;
⟦ ⟧ ↑ok(⟦T⟧) ;
\end{hacs}
\end{frame}

\begin{frame}[fragile]{HACS Program (VI)}\footnotesize
\begin{hacs}[texcl]
// Rules for $E$ sort:
// -- Synthesizes attribute $E.t$.
// -- Inherited attribute $E.e$ is distributed by Te scheme (with helper Te2).
sort E | ↑t | scheme ⟦Te ⟨E⟩⟧ ↓e ;

// $(E1)\quad E→E_1+E_2$:
// 1. $E_1.e=E.e$ and $E_2.e=E.e$ recursively propagated (note use of parenthesis sugar).
⟦ Te ( ⟨E#1⟩ + ⟨E#2⟩ ) ⟧  →  ⟦ (Te⟨E#1⟩) + (Te⟨E#2⟩) ⟧ ;
// 2. When $E_1.t$ and $E_2.t$ available then synthesize $E.t=\op{Unif}(E_1.t,E_2.t)$.
⟦ ⟨E#1↑t(#t1)⟩ + ⟨E#2↑t(#t2)⟩ ⟧ ↑t(Unif(#t1,#t2)) ;

// $(E2)\quad E→E_1*E_2$:
// 1. $E_1.e=E.e$ and $E_2.e=E.e$ recursively propagated (note use of parenthesis sugar).
⟦ Te ( ⟨E#1⟩ * ⟨E#2⟩ ) ⟧  →  ⟦ (Te⟨E#1⟩) * (Te⟨E#2⟩) ⟧ ;
// 2. When $E_1.t$ and $E_2.t$ available then synthesize $E.t=\op{Unif}(E_1.t,E_2.t)$.
⟦ ⟨E#1↑t(#t1)⟩ + ⟨E#2↑t(#t2)⟩ ⟧ ↑t(Unif(#t1,#t2)) ;
\end{hacs}
\end{frame}

\begin{frame}[fragile]{HACS Program (VII)}\footnotesize
\begin{hacs}[texcl]
// $(E3)\quad E→\textbf{int}$:
// 1. Propagation leaf: set $E.t=\op{Int}$ directly.
⟦ Te ⟨INT#1⟩ ⟧  →  ⟦ ⟨INT#1⟩ ⟧ ↑t(Int) ;
// 2. Synthesize $E.t=\op{Int}$.
⟦ ⟨INT#1⟩ ⟧ ↑t(Int) ;

// $(E4)\quad E→\textbf{float}$:
// 1. Propagation leaf: set $E.t=\op{Float}$ directly.
⟦ Te ⟨FLOAT#1⟩ ⟧  →  ⟦ ⟨FLOAT#1⟩ ⟧ ↑t(Float) ;
// 2. Synthesize $E.t=\op{Float}$.
⟦ ⟨FLOAT#1⟩ ⟧ ↑t(Float) ;

//$(E5)\quad S→\textbf{name}$: There are two disjoint propagation cases:
// a. $\op{defined}(E.e,\textbf{name}.sym)$ so $E.t=\op{lookup}(E.e,\textbf{name}.sym)$.
⟦ Te x ⟧ ↓e{x : #t} → ⟦ x ⟧ ↑t(#t) ;
// b. $¬\op{defined}(E.e,\textbf{name}.sym)$ so $E.t=\op{TypeErr}$.
⟦ Te x ⟧ ↓e{¬x} → ⟦ x ⟧ ↑t(TypeErr) ;
}
\end{hacs}
\end{frame}

\section{Conclusions}

%%%===========================================

\begin{frame}{Conclusions}

\begin{itemize}
\item HACS is in use to teach compilers at NYU.
\item HACS supports ``Dragon Book Notions'' well.
\item Higher Order concepts work in teaching.
\item Attribute dependencies are difficult.
\end{itemize}

\pause
\vspace*{2em}
\begin{center}
\alert{Questions?}
\end{center}

\end{frame}

%%%===========================================

\end{document}

%%---------------------------------------------------------------------
% Tell Emacs that this is a LaTeX document and how it is formatted:
% Local Variables:
% mode:latex
% fill-column:100
% End:
