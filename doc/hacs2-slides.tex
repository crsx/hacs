\documentclass[pdftex,aspectratio=169,14pt]{beamer}
%% $Id: lecture-setup.tex,v 1.12 2013/12/05 04:28:18 krisrose Exp $

%% Format.
\usepackage[utf8]{inputenc}
\usepackage{comment}
\usepackage{amsmath,amssymb,amsthm,stmaryrd,textcomp}
\usepackage[normalem]{ulem}
\usepackage{pgfpages}
\usepackage{utf8math}
\usepackage{url}
\usepackage{xspace}
\usepackage{fancyvrb,alltt}
\usepackage{graphicx}
\usepackage{color}
\definecolor{gray}{gray}{0.5}
\definecolor{lightgray}{rgb}{.9,.9,.9}
\definecolor{darkgray}{rgb}{.4,.4,.4}
\definecolor{purple}{rgb}{0.65, 0.12, 0.82}
\preparecolorset{rgb}{}{}{%
AliceBlue,.94,.972,1;%
AntiqueWhite,.98,.92,.844;%
Aqua,0,1,1;%
Aquamarine,.498,1,.83;%
Azure,.94,1,1;%
Beige,.96,.96,.864;%
Bisque,1,.894,.77;%
Black,0,0,0;%
BlanchedAlmond,1,.92,.804;%
Blue,0,0,1;%
BlueViolet,.54,.17,.888;%
Brown,.648,.165,.165;%
BurlyWood,.87,.72,.53;%
CadetBlue,.372,.62,.628;%
Chartreuse,.498,1,0;%
Chocolate,.824,.41,.116;%
Coral,1,.498,.312;%
CornflowerBlue,.392,.585,.93;%
Cornsilk,1,.972,.864;%
Crimson,.864,.08,.235;%
Cyan,0,1,1;%
DarkBlue,0,0,.545;%
DarkCyan,0,.545,.545;%
DarkGoldenrod,.72,.525,.044;%
DarkGray,.664,.664,.664;%
DarkGreen,0,.392,0;%
DarkGrey,.664,.664,.664;%
DarkKhaki,.74,.716,.42;%
DarkMagenta,.545,0,.545;%
DarkOliveGreen,.332,.42,.185;%
DarkOrange,1,.55,0;%
DarkOrchid,.6,.196,.8;%
DarkRed,.545,0,0;%
DarkSalmon,.912,.59,.48;%
DarkSeaGreen,.56,.736,.56;%
DarkSlateBlue,.284,.24,.545;%
DarkSlateGray,.185,.31,.31;%
DarkSlateGrey,.185,.31,.31;%
DarkTurquoise,0,.808,.82;%
DarkViolet,.58,0,.828;%
DeepPink,1,.08,.576;%
DeepSkyBlue,0,.75,1;%
DimGray,.41,.41,.41;%
DimGrey,.41,.41,.41;%
DodgerBlue,.116,.565,1;%
FireBrick,.698,.132,.132;%
FloralWhite,1,.98,.94;%
ForestGreen,.132,.545,.132;%
Fuchsia,1,0,1;%
Gainsboro,.864,.864,.864;%
GhostWhite,.972,.972,1;%
Gold,1,.844,0;%
Goldenrod,.855,.648,.125;%
Gray,.5,.5,.5;%
Green,0,.5,0;%
GreenYellow,.68,1,.185;%
Grey,.5,.5,.5;%
Honeydew,.94,1,.94;%
HotPink,1,.41,.705;%
IndianRed,.804,.36,.36;%
Indigo,.294,0,.51;%
Ivory,1,1,.94;%
Khaki,.94,.9,.55;%
Lavender,.9,.9,.98;%
LavenderBlush,1,.94,.96;%
LawnGreen,.488,.99,0;%
LemonChiffon,1,.98,.804;%
LightBlue,.68,.848,.9;%
LightCoral,.94,.5,.5;%
LightCyan,.88,1,1;%
LightGoldenrod,.933,.867,.51;%
LightGoldenrodYellow,.98,.98,.824;%
LightGray,.828,.828,.828;%
LightGreen,.565,.932,.565;%
LightGrey,.828,.828,.828;%
LightPink,1,.712,.756;%
LightSalmon,1,.628,.48;%
LightSeaGreen,.125,.698,.668;%
LightSkyBlue,.53,.808,.98;%
LightSlateBlue,.518,.44,1;%
LightSlateGray,.468,.532,.6;%
LightSlateGrey,.468,.532,.6;%
LightSteelBlue,.69,.77,.87;%
LightYellow,1,1,.88;%
Lime,0,1,0;%
LimeGreen,.196,.804,.196;%
Linen,.98,.94,.9;%
Magenta,1,0,1;%
Maroon,.5,0,0;%
MediumAquamarine,.4,.804,.668;%
MediumBlue,0,0,.804;%
MediumOrchid,.73,.332,.828;%
MediumPurple,.576,.44,.86;%
MediumSeaGreen,.235,.7,.444;%
MediumSlateBlue,.484,.408,.932;%
MediumSpringGreen,0,.98,.604;%
MediumTurquoise,.284,.82,.8;%
MediumVioletRed,.78,.084,.52;%
MidnightBlue,.098,.098,.44;%
MintCream,.96,1,.98;%
MistyRose,1,.894,.884;%
Moccasin,1,.894,.71;%
NavajoWhite,1,.87,.68;%
Navy,0,0,.5;%
NavyBlue,0,0,.5;%
OldLace,.992,.96,.9;%
Olive,.5,.5,0;%
OliveDrab,.42,.556,.136;%
Orange,1,.648,0;%
OrangeRed,1,.27,0;%
Orchid,.855,.44,.84;%
PaleGoldenrod,.932,.91,.668;%
PaleGreen,.596,.985,.596;%
PaleTurquoise,.688,.932,.932;%
PaleVioletRed,.86,.44,.576;%
PapayaWhip,1,.936,.835;%
PeachPuff,1,.855,.725;%
Peru,.804,.52,.248;%
Pink,1,.752,.796;%
Plum,.868,.628,.868;%
PowderBlue,.69,.88,.9;%
Purple,.5,0,.5;%
Red,1,0,0;%
RosyBrown,.736,.56,.56;%
RoyalBlue,.255,.41,.884;%
SaddleBrown,.545,.27,.075;%
Salmon,.98,.5,.448;%
SandyBrown,.956,.644,.376;%
SeaGreen,.18,.545,.34;%
Seashell,1,.96,.932;%
Sienna,.628,.32,.176;%
Silver,.752,.752,.752;%
SkyBlue,.53,.808,.92;%
SlateBlue,.415,.352,.804;%
SlateGray,.44,.5,.565;%
SlateGrey,.44,.5,.565;%
Snow,1,.98,.98;%
SpringGreen,0,1,.498;%
SteelBlue,.275,.51,.705;%
Tan,.824,.705,.55;%
Teal,0,.5,.5;%
Thistle,.848,.75,.848;%
Tomato,1,.39,.28;%
Turquoise,.25,.88,.815;%
Violet,.932,.51,.932;%
VioletRed,.816,.125,.565;%
Wheat,.96,.87,.7;%
White,1,1,1;%
WhiteSmoke,.96,.96,.96;%
Yellow,1,1,0;%
YellowGreen,.604,.804,.196}
\usepackage{hyperref}
\usepackage{rcs}

%% Macros.
\newcommand{\HAX}{\text{HACS}\xspace}
\newcommand{\CRSX}{\text{CRSX}\xspace}
\newcommand{\ie}{\textit{i.e.}\xspace}
\newcommand{\eg}{\textit{e.g.}\xspace}
\newcommand{\etc}{\textit{etc.}\xspace}
\newcommand{\aka}{\textit{aka.}\xspace}
\newcommand{\TBD}[1][]{\textit{To Be Done…#1}\marginpar{\bf TBD}}

\newcommand*{\cf}[1]{\texttt{\small #1}}
\newcommand*{\si}[1]{\textit{\small #1}}
\newcommand*{\surl}[1]{\mbox{\small\url{#1}}}

\newcommand{\http}[1]{\href{http://#1}{\emph{#1}}}
\newcommand{\mailto}[1]{\href{mailto:#1}{\emph{#1}}}

\newcommand{\qqquad}{{\qquad\quad}}
\newcommand{\qqqquad}{{\qquad\qquad}}
\newcommand{\opt}{{\ensuremath{_{\text{\it opt}}}}}

\newcommand\caret{\mathbin{\char`\^}}

\def\SubScript{\textit{SubScript}\xspace}
\def\vacuum{\endgraf\vskip 0pt plus 1filll\relax}
\def\tup#1{\ensuremath{\left\langle{#1}\right\rangle}}
\def\mc#1{\fbox{\ensuremath{#1}}}

\def\first{\textsc{first}}
\def\follow{\textsc{follow}}
\def\tup#1{\ensuremath{\left\langle{#1}\right\rangle}}
\def\q#1{\,\text{``#1''}\,}
\def\qm#1{\,\text{``\ensuremath{#1}''}\,}
\def\t#1{\textbf{#1}}
\def\nt#1{{\textit{#1}}}

\def\alertonly#1#2#3{\alert<#1>{\only<#2>{#3}}}
\def\alertthen#1#2#3{\alert<#1>{\uncover<#2>{#3}}}

%% WORKING \Cases{...} (based on \cases from plain TeX -- the one in
%% AMS-LaTeX is seriously buggy):
\makeatletter
\def\Cases#1{\mbox{$
  \left\{\,\vcenter{\let\\=\cr \normalbaselines\m@th
    \ialign{$\vphantom(##\hfil$&\quad##\hfil\crcr#1\crcr}}\right.$}}
\makeatother

%% Environments.

%% Comments.
\RequirePackage{verbatim}
\def\ignoreenvironment#1{%
  \expandafter\let\csname #1\endcsname=\comment
  \expandafter\let\csname end#1\endcsname=\endcomment}

%% Problem/solution...
\def\setupdefault{\ignoreenvironment{solution}}
\def\setupsolutions{\newenvironment{solution}[1][Answer]{\color{gray}\proof[##1]}{\endproof}}
\def\afterdash#1-#2-#3@@{#2}
\csname setup\expandafter\afterdash\jobname-default-@@\endcsname
%\expandafter\def\expandafter\setupafterdashjobcs\expandafter{\csname setup\expandafter\afterdash\jobname-default-@@\endcsname}
%\setupafterdashjobcs

%% Old code listings.
\usepackage{fancyvrb}
\CustomVerbatimEnvironment{code}{Verbatim}{tabsize=4,fontsize=\small,numbers=left,numberblanklines=false,xleftmargin=\parindent}
\CustomVerbatimCommand{\inputcode}{VerbatimInput}{tabsize=4,fontsize=\footnotesize,numbers=left,numberblanklines=false,xleftmargin=\parindent}
%\DefineShortVerb{\"} % makes " work in verbatim even when \active

%% New code listings...
\usepackage{listings}
\lstset{basicstyle=\ttfamily}

%% HACS.
\def\Identifier#1{#1}
\def\Keyword#1{\textsf{\bf#1}}

\lstdefinelanguage{HACS}{basicstyle=\fontfamily{ccr}\selectfont,%
  extendedchars=true,inputencoding=utf8,sensitive,%
  identifierstyle=\Identifier,%
  string=[d]{"},morestring=[d]{'},upquote=true,stringstyle=\ttfamily,showstringspaces=true,%
  comment=[l]{//},morecomment=[s]{/*}{*/},commentstyle=\color{blue}\rm,%
  keywordstyle=\Keyword,%
  keywords={attribute,data,fragment,import,module,nested,property,rule,scheme,simplify,sort,%
    space,static,sugar,symbol,tag,token,default,error},%
  literate=%
    {⟦}{{\ensuremath{\llbracket}}}1 {⟧}{{\ensuremath{\rrbracket}}}1 {⟨}{{\ensuremath{\langle}}}1 {⟩}{{\ensuremath{\rangle}}}1 %
    {→}{{\ensuremath{\rightarrow}}}2 {↑}{{\ensuremath{\uparrow}}}1 {↓}{{\ensuremath{\downarrow}}}1 {¬}{{\ensuremath{\lnot}}}2 {¶}{{\P}}1 %
    {ε}{{\ensuremath{\varepsilon}}}1 {…}{{\ensuremath{\dots}}}2 {∧}{{\ensuremath{\!\wedge}}}2 {∨}{{\ensuremath{\!\vee}}}2 {↦}{{\ensuremath{\mapsto}}}2 %
    {¹}{{\ensuremath{^1}}}1 {²}{{\ensuremath{^2}}}2 %
}[comments,strings,keywords]
%%
\lstnewenvironment{hacs}[1][]{\lstset{language=HACS,basewidth={.55em,.45em},columns=flexible,#1}\upshape}{}
\newcommand\inputhacs[2][]{\lstinputlisting[language=HACS,basewidth={.55em,.45em},columns=flexible,#1]{#2}}
\newcommand\hacsc{\lstinline[language=HACS]}
%\lstMakeShortInline[language=HACS,columns=fullflexible]"

%% Tiger.
\lstdefinelanguage{Tiger}{basicstyle=\normalsize,%
  extendedchars=true,inputencoding=utf8,sensitive,%
  identifierstyle=\textit,%
  string=[d]{"},upquote=true,stringstyle=\ttfamily,showstringspaces=true,%
  comment=[s]{/*}{*/},commentstyle=\color{gray}\rm,%
  keywordstyle=\bfseries,%
  keywords={type,array,of,var,function,nil,let,in,end,if,then,else,break,while,do},%
  literate=%
    {≡}{{\ensuremath{\equiv}}}1 %
}[comments,strings,keywords]
%%
\lstnewenvironment{tiger}[1][]{\lstset{language=Tiger,basewidth={.5em,.35em},columns=flexible,#1}\upshape}{}
\newcommand\inputtiger[2][]{\lstinputlisting[language=Tiger,basewidth={.5em,.35em},columns=flexible,#1]{#2}}
\newcommand\tigerc{\lstinline[language=Tiger]}

\lstdefinelanguage{JavaScript}{
  keywords={break, case, catch, continue, debugger, default, delete, do, else, false, finally, for, function, if, in, instanceof, new, null, return, switch, this, throw, true, try, typeof, var, while, with, class, export, throw, implements, import, this, extends},
  morecomment=[l]{//},
  morecomment=[s]{/*}{*/},
  morestring=[b]',
  morestring=[b]",
  ndkeywords={boolean, any, number, string, void},
  keywordstyle=\color{blue}\bfseries,
  ndkeywordstyle=\color{darkgray}\bfseries,
  identifierstyle=\color{black},
  commentstyle=\color{purple}\ttfamily,
  stringstyle=\color{red}\ttfamily,
  sensitive=true
}

\lstnewenvironment{JavaScript}[1][]{\lstset{
   language=JavaScript,
   backgroundcolor=\color{lightgray},
   extendedchars=true,
   basicstyle=\ttfamily,
   showstringspaces=false,
   showspaces=false,
   numbers=left,
   numberstyle=\scriptsize,
   numbersep=7pt,
   numberblanklines=false,
   tabsize=2,
   breaklines=true,
   showtabs=false,
   captionpos=b,
   #1
}\upshape}{}

\newcommand\inputJavaScript[2][]{\lstset{
   language=JavaScript,
   backgroundcolor=\color{lightgray},
   extendedchars=true,
   basicstyle=\ttfamily,
   showstringspaces=false,
   showspaces=false,
   numbers=left,
   numberstyle=\scriptsize,
   numbersep=7pt,
   tabsize=2,
   breaklines=true,
   showtabs=false,
   captionpos=b,
   #1
}{#2}}

\newcommand\JS[1][]{\lstinline[
   language=JavaScript,
   backgroundcolor=\color{lightgray},
   extendedchars=true,
   basicstyle=\small\ttfamily,
   showstringspaces=false,
   showspaces=false,
   #1]}

%% Identifiers
\newcommand{\set}[1]{\textit{#1}}
\newcommand{\op}[1]{\ensuremath{\operatorname{#1}}}

\endinput

%% Last to pickup right catcodes...
\usepackage[graph,arrow,curve,color,tips,frame]{xy}
\def\mtxt#1{\hbox{\footnotesize$\begin{array}{l}#1\end{array}$}}

\let\next\relax %workaround...

%% T diagram helpers
\def\teediagram#1{\xygraph{!{(0.6,0):}#1}}
\newgraphescape{I}#1#2#3#4{
  []*+[#4]{#1}="I#1#2#3..s"
  [d]*+[#4]{#2}="I#1#2#3..m"
  (
  []!{ "I#1#2#3..m"."I#1#2#3..s"="I#1#2#3.m";
       "I#1#2#3..s"."I#1#2#3..m"="I#1#2#3.s" }
    , "I#1#2#3.s"!{+UL}
    -@[#4] "I#1#2#3.s"!{+UR}
    -@[#4] "I#1#2#3.s"!{+DR}
    -@[#4] "I#1#2#3.s"!{+DL}
    -@[#4] "I#1#2#3.s"!{+UL}
  )
}
\newgraphescape{C}#1#2#3#4#5{
  []*+[#5]{#1}="C#1#2#3#4..s" :@[#5] [rr] *+[#5]{#2}="C#1#2#3#4..t"
  [dl]*+[#5]{#3}="C#1#2#3#4.m"
  (
    []!{ "C#1#2#3#4..s"."C#1#2#3#4..t"="C#1#2#3#4.s";
         "C#1#2#3#4..t"."C#1#2#3#4..s"="C#1#2#3#4.t" }
    , "C#1#2#3#4.s"!{+UL}
    -@[#5] "C#1#2#3#4.s"!{+UR}
    -@[#5] "C#1#2#3#4.s"!{+DR}
    -@[#5] "C#1#2#3#4.m"!{+UR}
    -@[#5] "C#1#2#3#4.m"!{+DR}
    -@[#5] "C#1#2#3#4.m"!{+DL}
    -@[#5] "C#1#2#3#4.m"!{+UL}
    -@[#5] "C#1#2#3#4.s"!{+DL}
    -@[#5] "C#1#2#3#4.s"!{+UL}
  )
}
\newgraphescape{W}#1#2#3#4#5{
  []*+[#5]{#1}="I#1#2#3#4..s" :[rrrr] *+[#5]{#2}="I#1#2#3#4..t"
  [dll]*+[#5]{#3}="I#1#2#3#4.m"
  !{ "I#1#2#3#4..s"."I#1#2#3#4..t"="I#1#2#3#4.t"
     ."I#1#2#3#4..s"="I#1#2#3#4.s" }
  ( "I#1#2#3#4.s"!{+UL}
    -@[#5] "I#1#2#3#4.s"!{+UR}
    -@[#5] "I#1#2#3#4.s"!{+DR}
    -@[#5] "I#1#2#3#4.m"!{+UR}
    -@[#5] "I#1#2#3#4.m"!{+DR}
    -@[#5] "I#1#2#3#4.m"!{+DL}
    -@[#5] "I#1#2#3#4.m"!{+UL}
    -@[#5] "I#1#2#3#4.s"!{+DL}
    -@[#5] "I#1#2#3#4.s"!{+UL}
  )
}

\newcommand{\circled}[1]{\text{\scriptsize$\vcenter{\xygraph{*+[o][F-]{#1}}}$}}
\newcommand{\Circled}[1]{\text{\scriptsize$\vcenter{\xygraph{*+[o][F=]{#1}}}$}}


%% Style.
\usepackage[T1]{fontenc}
\usetheme{default}
\useinnertheme{rounded}
\setbeamercovered{transparent=5}
\def\vacuum{\endgraf\vskip 0pt plus 1filll\relax}
\setbeamercolor{alerted text}{fg=red}
\usepackage{charter}
\renewcommand{\ttdefault}{txtt}
\usefonttheme{serif}
\def\q#1{~\text{`\texttt{#1}'}~}
\def\t#1{\texttt{#1}}
\def\tup#1{\ensuremath{\left\langle{#1}\right\rangle}}

%% Topmatter
\title{Proposals for New \HAX2 Features}
\author[Kris Rose]{Kristoffer H. Rose\\Two Sigma}
\date{January 26, 2016\\CHUM 5, Two Sigma, New York}

%%
\begin{document}
\frame{\titlepage}

\begin{frame}{Outline}
  \tableofcontents
\end{frame}


\section{Lexing \& Parsing}
\begin{frame}{Outline}
  \tableofcontents[current]
\end{frame}

\begin{frame}[fragile]{Optional}
  \begin{hacs}[mathescape]
    sort … | scheme F($N$?);
    F(⟦ ⟨$N$ #1⟩ ⟧) →  …; //present
    F(⟦ ⟧) →  …; //absent
  \end{hacs}
\end{frame}

\begin{frame}[fragile]{Many}
  \begin{hacs}[mathescape]
    sort … | scheme F($N$*);
    F(⟦ ⟨$N$ #1⟩ ⟨$N$* #rest⟩ ⟧) →  …; //one more
    F(⟦ ⟧) →  …; //end
  \end{hacs}
\end{frame}

\begin{frame}[fragile]{List}
  \begin{hacs}[mathescape]
    sort … | scheme F($N$+$U$);
    F(⟦ ⟨$N$ #1⟩ $U$ ⟨$N$+$U$ #rest⟩ ⟧) →  …; //non-last
    F(⟦ ⟨$N$ #1⟩ ⟧) →  …; //last
  \end{hacs}
  \pause
  \begin{hacs}[mathescape]
    sort Bool | scheme IsSingle(Elem+[,]);
    IsSingle(⟦ ⟨Elem #1⟩ ,  ⟨Elem+[,] #rest⟩ ⟧) →  False;
    IsSingle(⟦ ⟨Elem #1⟩ ⟧) →  True;
  \end{hacs}
\end{frame}

\begin{frame}[fragile]{Optional List}
  \begin{hacs}[mathescape]
    sort … | scheme F($N$+$U$?) | scheme F2($N$+$U$);
    F(⟦ ⟨$N$+$U$ #⟩ ⟧) →  F2(#);

    F(⟦ ⟧) →  …;  //empty
    F2(⟦ ⟨$N$ #1⟩ $U$ ⟨$N$*$U$ #rest⟩ ⟧) →  …; //non-last
    F2(⟦ ⟨$N$ #1⟩ ⟧) →  …; //last
  \end{hacs}
\end{frame}


\section{Modules}
\begin{frame}{Outline}
  \tableofcontents[current]
\end{frame}

\begin{frame}[fragile]{Inclusion}
\begin{hacs}
    include org.crsx.hacs.Grammar;
\end{hacs}
\end{frame}

\begin{frame}[fragile]{Import}
\begin{hacs}
    import org.crsx.hacs.Grammar;
\end{hacs}
\end{frame}

\begin{frame}[fragile]{Private}
\begin{hacs}
    sort List α | private scheme ⟦ { ⟨List α⟩ } ⟨List α⟩ ⟧;
\end{hacs}
\end{frame}


\section{Attributes}
\begin{frame}{Outline}
  \tableofcontents[current]
\end{frame}

\begin{frame}[fragile]{Syntax}
  \begin{itemize}
  \item \hacsc"↓a{#}" to catch all binders.
  \item \hacsc"↓a{x:A, y:B}" same as \hacsc"↓a{x:A} ↓a{y:B}".
  \item Set membership changes to \hacsc"↓a{x:}".
  \item Data can carry inherited attributes.
  \end{itemize}
\end{frame}

\begin{frame}[fragile]{Inherited Attributions}
  \begin{equation*}
    \ov{{↓}i_0}  ~ d\!\left(\,\ov{\, \ov{{↓}i} \, t \, \ov{{↑}s} \,}\,\right) ~ \ov{{↑}s_0}
  \end{equation*}
  \pause
  \begin{definition}\label{def:L-attributed}
    A \HAX2 attribution is \emph{locally L-attributed} if
    \begin{align}
      ∀j, 1≤j≤\$\colon
      \op{mv}\bigl( ~ \ov{{↓}i_0} ~ \ov{{↓}i_1}…\ov{{↓}i_{j-1}} ~ \ov{{↑}s_1}…\ov{{↑}s_{j-1}} ~ \bigr)
      &⊇ \op{mv}\bigl( \ov{{↓}i_j} \bigr)
      \\
      ∀j, 1≤j≤\$\colon
      \op{mv}\bigl( ~ \ov{{↓}i_0} ~ \ov{{↓}i_1}…\ov{{↓}i_j} ~ \ov{{↑}s_1}…\ov{{↑}s_{j-1}} ~ \bigr)
      &⊇ \op{mv}\bigl( \ov{{↑}s_j} \bigr)
      \\
      \op{mv}\bigl( ~ \ov{{↓}i_0} ~ \ov{\ov{{↓}i}} ~ \ov{\ov{{↑}s}} ~ \bigr)
      &⊇ \op{mv}\bigl( \ov{{↑}s_0} \bigr)
    \end{align}
  \end{definition}
\end{frame}

\begin{frame}{SDD with Explicit Passes}
  \vspace*{-1em}
  \begin{equation*}
    \begin{array}{r@{\;}l|lr}
      \Xhline{2\arrayrulewidth}
      \multicolumn{2}{l|}{\textsc{Production}}  & \textsc{Semantic Rules} &\Bigstrut\\
      \hline\Bigstrut
      S &→ \textbf{id} := E_1; S_2
      & E_1.e = S.e; &\thetag{S1a}\\
      && S_2.e = \op{Extend}(S.e, \textbf{id}.sym, E_1.t) &\thetag{S1b}
      \\[\jot]
      &\mid \{~S_1~\}~S_2 & S_1.e = S.e; S_2.e = S.e &\thetag{S2}
      \\[\jot]
      &\mid ε & &\thetag{S3}
      \\[\jot]
      \hline\Bigstrut
      E &→ E_1 + E_2 & E_1.e=E.e; E_2.e=E.e; &\thetag{E1a}\\
      && E.t = \op{Unif}(E_1.t, E_2.t) &\thetag{E1b}\\[\jot]
      &\mid \textbf{int} & E.t = \op{Int}&\thetag{E2}\\[\jot]
      &\mid \textbf{id} & E.t = \op{Lookup}(E.e,\textbf{id}.sym)&\thetag{E3}
      \\[\jot]
      \Xhline{2\arrayrulewidth}
    \end{array}
  \end{equation*}
\end{frame}

\begin{frame}[fragile]{New SDD Encoding (I)}\small
  \begin{hacs}[texcl,numbers=right,xrightmargin=2em]
// Implementation of simple type checking SDD.
module org.crsx.hacs.samples2.Typing {

  // Grammar.
  token INT  | [0-9]+;
  token ID  | [A-Z] [A-Za-z0-9_]*;
  main sort S  | ⟦ ⟨ID⟩ := ⟨E⟩; ⟨S⟩ ⟧  | ⟦ { ⟨S⟩ } ⟨S⟩ ⟧  | ⟦⟧;
  sort E  | ⟦ ⟨E@1⟩ + ⟨E⟩ ⟧  | ⟦⟨INT⟩⟧@1  | ⟦⟨ID⟩⟧@1;

  // Attributes.
  sort T  | Int  | scheme Unif(T,T);
  attribute ↓e{ID:T};
  attribute ↑t(T);
  //…(rules for Unif omitted)
  \end{hacs}
\end{frame}

\begin{frame}[fragile]{New SDD Encoding (II)}\small
  \begin{hacs}[texcl,numbers=right,xrightmargin=2em]
  // Rules for statements.
  sort S | ↓e;

  // S1a: $E_1.e = S.e$
  ⟦ id := ⟨E#1  ↓e{#Se}⟩; ⟨S#2⟩ ⟧  ↓e{#Se} ;

  // S1b: $S_2.e = \op{Extend}(S.e, \textbf{id}.sym, E_1.t) $
  ⟦ id := ⟨E#1 ↑t(#E1t) ⟩; ⟨S#2  ↓e{#Se,⟦id⟧:#E1t}⟩ ⟧ ↓e{#Se} ;

  // S2: $S_1.e = S.e; S_2.e = S.e $
  ⟦ { ⟨S#1 ↓e{#Se}⟩ } ⟨S#2 ↓e{#Se}⟩ ⟧ ↓e{#Se} ;

  // S3:
  ⟦ ⟧ ↓e{#Se} ;
  \end{hacs}
\end{frame}

\begin{frame}[fragile]{New SDD Encoding (III)}\small
  \begin{hacs}[texcl,numbers=right,xrightmargin=2em]
  // Rules for expressions.
  sort E | ↓e | ↑t; // note: t depends on e from (E3)

  // E1a: $E_1.e=E.e; E_2.e=E.e$
  ⟦ ⟨E#1  ↓e{#Ee}⟩ + ⟨E#2  ↓e{#Ee}⟩ ⟧↓e{#Ee} ;

  // E1b: $E.t = \op{Unif}(E_1.t, E_2.t)$
  ⟦ ⟨E#1 ↑t(#E1t) ⟩ + ⟨E#2 ↑t(#E2t) ⟩ ⟧  ↑t(Unif(#E1t,#E2t));

  // E2: $E.t = \op{Int}$
  ⟦ ⟨INT#1⟩ ⟧  ↑t(Int);

  // E3: $E.t = \op{Lookup}(E.e,\textbf{id}.sym)$
  ⟦ id ⟧ ↓e{⟦id⟧ : #t}   ↑t(#t);
}
  \end{hacs}
\end{frame}


\section{Inference Rules}
\begin{frame}{Outline}
  \tableofcontents[current]
\end{frame}

\begin{frame}{New Syntax}
  \vspace*{-1em}
  \begin{equation*}
    \kw{sort}~S~;~~
    [R]~\kw{when}~[\ov{x~\kw{as}~S'}](\, C_1⇒P_1 ~\cdots~ C_n⇒P_n \,) ~ \kw{infer} ~ P_0 ⇒ C_{n+1} ~;
  \end{equation*}
  %% 
  \pause
  %% 
  \begin{equation*}
    \dfrac
    { ∀\,\ov{x} : (\, C_1⇒P_1 ~\cdots~ C_n⇒P_n \,) }
    { P_0 ⇒ C_{n+1}}
    ~(R)
    \label{eq:infer}
  \end{equation*}
  %% 
  \pause
  %% 
  \begin{itemize}
  \item $R$ is the unique name of the rule (as also allowed on other rules).
  \item $P_0$ and all of $C_1,…,C_n$ are \emph{function constructions}.
  \item $P_0$ is a \emph{pattern} and all of $P_1,…,P_n$ are \emph{pattern fragments}.
  \item $∀i\colon \op{mv}(C_i) ⊆ \op{mv}(P_0…P_{i-1})$.
  \item Each of the $C_i,P_i$ terms has some sort $S_i$.
  \item The variables $x_i$ can occur in all the $C_j,P_j$ of sort $S'_i$.
  \end{itemize}
\end{frame}

\begin{frame}{Translation}
  \vspace*{-1em}
  \begin{align*}
    &\kw{sort}~S~;\\
    & P_0 → R_1(P_0, [\ov{x}]C_1) \tag{$R_0$} ~;
    \\
    &\kw{|scheme} ~ R_1(S, [\ov{S'}]S_1) ~;\\
    &R_1(P_0, [\ov{x}]P_1) → R_2(P_0, [\ov{x}]P_1, [\ov{x}]C_2) \tag{$R_1$} ~;
    \\[-\jot]
    &~~\vdots\notag\\
    &\kw{|scheme} ~ R_{n-1}(S, [\ov{S'}]S_1,…, [\ov{S'}]S_{n-1}) ~;\\
    &R_{n-1}(P_0, [\ov{x}]P_1, …, [\ov{x}]P_{n-1}) → R_n(P_0, [\ov{x}]P_1, …, [\ov{x}]P_{n-1}, [\ov{x}]C_n) \tag{$R_{n-1}$} ~;
    \\
    &\kw{|scheme} ~ R_n(S, [\ov{S'}]S_1,…, [\ov{S'}]S_n) ~;\\
    &R_n(P_0, [\ov{x}]P_1, …, [\ov{x}]P_n) → C_{n+1} \tag{$R_n$} ~;
  \end{align*}
\end{frame}


\section{Miscellaneous}
\begin{frame}{Outline}
  \tableofcontents[current]
\end{frame}

\begin{frame}[fragile]{Other things…}
  \begin{itemize}
  \item Fix \kw{sugar}, allow short form \kw{scheme}.
  \item Fix Computed.
  \item Allow non-primary sorts for non-data declarations:
    \begin{hacs}
      sort α | scheme If(Bool, α, α);
      If(True, #1, #2) → #1;
      If(False, #1, #2) → #2;

      sort List Bool | scheme Flip(List Bool);
      Flip(#s) →  Map([x]Not(x), #s);
    \end{hacs}
  \end{itemize}
\end{frame}


\section{Core \HAX}
\begin{frame}{Outline}
  \tableofcontents[current]
\end{frame}

\begin{frame}\footnotesize
  \vspace*{-2em}
  \begin{align*}
    \lefteqn{\textsc{Production}}\qquad \tag{\textsc{Name}}
    \\
    M \tag{Module} &::= \kw{module} ~ q ~ \kw\{ ~ U^* ~ \kw\}
    \\
    U \tag{Unit} &::= \kw{import} ~ q ~ \kw; ~\bigm|~ \kw{attribute} ~ AR ~ AF ~ \kw; \\[-1pt]
    &~~\bigm|~ \kw{main}^? ~ \kw{sort} ~ SF ~ \kw; ~\bigm|~ \kw{\texttt{|}}\, AR ~ \kw; ~\bigm|~
    \kw{\texttt{|}}\, CF ~ \kw; ~\bigm|~ \kw{\texttt{|}}\, \kw{scheme} ~ CF ~ AR^* ~ \kw; ~\bigm|~
    \kw{\texttt{|}}\, \kw{variable} ~ \kw; \\[-1pt]
    &~~\bigm|~ \kw[~O^{*\kw,}~\kw] ~ T ~ \kwm{→} ~ T ~ \kw;
    \\[1ex]
    SF \tag{SortForm} &::= v ~\bigm|~ c ~ SA^*
    \\
    SA \tag{SortArgument} &::= v ~\bigm|~ c ~\bigm|~ \kw( ~ SF ~ \kw) 
    \\
    AR \tag{AttributeReference} &::= \kwm{↑} ~ v ~\bigm|~ \kwm{↓} ~ v
    \\
    AF \tag{AttributeForm} &::= \kw( ~ SF ~ \kw) ~\bigm|~ \kw\{ ~ SF ~ \kw: ~ SF ~ \kw\}
    \\
    CF \tag{ConstructorForm} &::= c ~ \kw( ~ PF^{*\kw,} ~ \kw) 
    \\
    PF \tag{PieceForm} &::= \kw[ ~ SF^{*\kw,} ~ \kw] ~ SF
    \\[1ex]
    T \tag{Term} &::= c ~ \kw( ~ P^{*\kw,} ~ \kw) ~ A^* ~\bigm|~ v ~ A^* ~\bigm|~ m ~ \kw[ ~
    T^{*\kw,} ~ \kw] ~ A^*
    \\
    P \tag{Piece} &::= \kw[ ~ v^{*\kw,} ~ \kw] ~ T
    \\
    A \tag{Attribute} &::= AR ~ \kw( ~ T ~ \kw) ~\bigm|~ AR ~ \kw\{ ~ T ~ \kw: ~ T ~ \kw\} ~\bigm|~
    AR ~ \kw\{ ~ \kwm{¬} ~ T ~ \kw\} ~\bigm|~ AR ~ \kw\{ ~ m ~ \kw\} ~\bigm|~ \kwm{↕} ~ m
    \\[1ex]
    O \tag{Option} &::= c ~\bigm|~ \kw{priority} ~\bigm|~ \kw{default} ~\bigm|~
    \kw{free}\kw(~v~\kw{as}~SF~\kw) ~\bigm|~ \kw{fresh}\kw(~v~\kw{as}~SF~\kw)
  \end{align*}
\end{frame}


\section{Implementation…}
\begin{frame}{Outline}
  \tableofcontents[current]
\end{frame}

\begin{frame}[shrink]{Hacs.mk is a mess…}
  \begin{itemize}

  \item Control program is shell script.

  \item User script is first parsed with generic \emph{RawHx.pg} parser to \CRSX term file.

  \item User ``raw \HAX'' term is processed by \emph{Prep.crs} to generate two derived parsers:
    \begin{enumerate}
    \item ``Meta-parser'' that combines raw parsing with custom parsing of the declared raw
      notations and of ``embedded'' fragment parsing user syntax with embedded references.
    \item User parser just for user syntax.
    \end{enumerate}

  \item User script is parsed again with the generated meta-parser to \CRSX term file.

  \item User ``custom parsed \HAX'' is processed by \emph{Cook.crs} to generate the rewrite system
    for the user's script, which is a \CRSX program.

  \item Finally, a custom shell script is generated that invokes the \CRSX interpreter configured to
    use the user parser (so interpreted).

  \end{itemize}
\end{frame}

\begin{frame}
  \begin{center}
    \alert{\emph{What do y'all think?}}
  \end{center}
\end{frame}
 

\end{document}



%%---------------------------------------------------------------------
% Local Variables:
% mode: latex
% fill-column:100
% TeX-master: t
% End:
