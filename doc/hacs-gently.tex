%% hacs-gently.tex
%%
\documentclass[11pt]{article} %style: font size.
%% $Id: setup.tex,v 1.11 2013/12/05 04:10:03 krisrose Exp $

%% Format.
\usepackage[utf8]{inputenc}
\usepackage{comment}
\usepackage{amsmath,amssymb,stmaryrd,textcomp}
\allowdisplaybreaks %style
\usepackage[normalem]{ulem}
\usepackage{pgfpages}
\usepackage{utf8math}
\usepackage{url}
\usepackage{xspace}
\usepackage{fancyvrb}
\usepackage{xcolor}
\usepackage{graphicx}
\usepackage{hyperref}
\usepackage{rcs}

%% Macros.
\newcommand{\HAX}{\text{HACS}\xspace}
\newcommand{\CRSX}{\text{CRSX}\xspace}
\newcommand{\ie}{\textit{i.e.}\xspace}
\newcommand{\eg}{\textit{e.g.}\xspace}
\newcommand{\etc}{\textit{etc.}\xspace}
\newcommand{\aka}{\textit{aka.}\xspace}
\newcommand{\TBD}[1][]{\textit{To Be Done…#1}\marginpar{\bf\color{red} TBD}}

%% Misc.
\def\Bigstrut{\ensuremath{\vphantom{\Bigm|}}}
\newcommand\caret{\mathbin{\char`\^}}

%% Units
\usepackage{amsthm}
\swapnumbers %style
\newtheorem{theorem}{Theorem}[section]
\newtheorem{lemma}[theorem]{Lemma}
\theoremstyle{definition}
\newtheorem{definition}[theorem]{Definition}
\newtheorem{proposition}[theorem]{Proposition}
\newtheorem{example}[theorem]{Example}
\newtheorem{practice}[theorem]{Practice}
\newtheorem{notation}[theorem]{Notation}
\newtheorem{manual}[theorem]{Manual}
%\newenvironment{manual}{\small\begin{manual-body}}{\end{manual-body}}
\newtheorem{invariant}[theorem]{Invariant}
\newtheorem{requirements}[theorem]{Requirements}
\newtheorem{limitations}[theorem]{Limitations}
\newtheorem{commands}[theorem]{Commands}
\newtheorem{error}[theorem]{Error}
\newtheorem{note}[theorem]{Note}
\newtheorem{plan}[theorem]{Plan}
%\theoremstyle{remark}
\newtheorem{remark}[theorem]{Remark}

%% Old code listings.
\usepackage{alltt}
\usepackage{fancyvrb}
\CustomVerbatimEnvironment{code}{Verbatim}{tabsize=4,fontsize=\small,numberblanklines=false,xleftmargin=\parindent}
\CustomVerbatimCommand{\inputcode}{VerbatimInput}{tabsize=4,fontsize=\footnotesize,numbers=left,numberblanklines=false,xleftmargin=\parindent}
%\DefineShortVerb{\"} % makes " work in verbatim even when \active


%% New code listings...
\usepackage{listings}
\lstset{basicstyle=\ttfamily}

%% HACS.
\def\Identifier#1{#1}
\def\Keyword#1{\textsf{\color{gray}#1}}

\lstdefinelanguage{HACS}{basicstyle=\fontfamily{ccr}\selectfont,%
  extendedchars=true,inputencoding=utf8,sensitive,%
  identifierstyle=\Identifier,%
  string=[d]{"},morestring=[d]{'},upquote=true,showstringspaces=false,%
  comment=[l]{//},morecomment=[s]{/*}{*/},commentstyle=\color{blue}\rm,%
  keywordstyle=\Keyword,%
  keywords={attribute,data,fragment,import,main,module,nested,property,rule,scheme,simplify,sort,%
    space,static,sugar,symbol,tag,token,default,error,binds,as},%
  literate=%
    {⟦}{{\ensuremath{\llbracket}}}1 {⟧}{{\ensuremath{\rrbracket}}}1 {⟨}{{\ensuremath{\langle}}}1 {⟩}{{\ensuremath{\rangle}}}1 %
    {→}{{\ensuremath{\rightarrow}}}2 {↑}{{\ensuremath{\uparrow}}}1 {↓}{{\ensuremath{\downarrow}}}1 {¬}{{\ensuremath{\lnot}}}2 {¶}{{\P}}1 %
    {ε}{{\ensuremath{\varepsilon}}}1 {…}{{\ensuremath{\dots}}}2 {∧}{{\ensuremath{\!\wedge}}}2 {∨}{{\ensuremath{\!\vee}}}2 {↦}{{\ensuremath{\mapsto}}}2 %
}[comments,strings,keywords]
%%
\lstnewenvironment{hacs}[1][]{\lstset{language=HACS,basewidth={.53em,.43em},columns=flexible,numberstyle=\scriptsize,numberblanklines=false,firstnumber=auto,xrightmargin=1pc,#1}\upshape}{}
\newcommand\inputhacs[2][]{\lstinputlisting[language=HACS,basewidth={.53em,.43em},columns=flexible,numberstyle=\scriptstyle,numberblanklines=false,xrightmargin=1pc,#1]{#2}}
\newcommand\hacsc{\lstinline[language=HACS]}
\lstMakeShortInline[language=HACS,columns=fullflexible]"


%%% \usepackage{listings}
%%% 
%%% %% HACS.
%%% \lstdefinelanguage{HACS}{basicstyle=\rm,%
%%%   numberstyle=\scriptsize,
%%%   numbersep=7pt,
%%%   numberblanklines=false,
%%%   extendedchars=true,inputencoding=utf8,sensitive,%
%%%   identifierstyle=\textit,%
%%%   string=[d]{"},morestring=[m]{'},upquote=true,stringstyle=\ttfamily,%showstringspaces=true,%
%%%   comment=[l]{//},morecomment=[s]{/*}{*/},commentstyle=\color{gray}\upshape,%
%%%   keywordstyle=\color{black}\bfseries,%
%%%   keywords={attribute,data,default,fragment,import,main,module,nested,property,rule,scheme,simplify,sort,space,static,sugar,symbol,tag,token},%
%%%   ndkeywordstyle={\textup},%
%%%   literate=%
%%%     {⟦}{{\ensuremath{\llbracket}}}1 {⟧}{{\ensuremath{\rrbracket}}}1 {⟨}{{\ensuremath{\langle}}}1 {⟩}{{\ensuremath{\rangle}}}1 %
%%%     {→}{\ensuremath{\rightarrow}}2 {↑}{\ensuremath{\uparrow}}1 {↓}{\ensuremath{\downarrow}}1 {ε}{\ensuremath{\varepsilon}}1 %
%%%     {…}{\ensuremath{\dots}}2 {¬}{\ensuremath{\lnot}}1 {∧}{\ensuremath{\wedge}}2 {∨}{\ensuremath{\vee}}2 {¶}{\P}1 {↦}{\ensuremath{\mapsto}}2 %
%%% }[comments,strings,keywords]
%%% %%
%%% \lstnewenvironment{hacs}[1][]{\lstset{language=HACS,basewidth={.55em,.4em},columns=flexible,#1}\upshape}{}
%%% \newcommand\inputhacs[2][]{\lstinputlisting[language=HACS,basewidth={.55em,.4em},columns=flexible,#1]{#2}}
%%% \newcommand\hacsc{\lstinline[language=HACS]}
%%% \lstMakeShortInline[language=HACS,columns=fullflexible]"
%%% 
%%% %% Tiger.
%%% \lstdefinelanguage{Tiger}{basicstyle=\normalsize,%
%%%   extendedchars=true,inputencoding=utf8,sensitive,%
%%%   identifierstyle=\textit,%
%%%   string=[d]{"},upquote=true,stringstyle=\ttfamily,showstringspaces=true,%
%%%   comment=[s]{/*}{*/},commentstyle=\color{grey}\rm,%
%%%   keywordstyle=\color{black}\bfseries,%
%%%   keywords={type,array,of,var,function,nil,let,in,end}%
%%% }[comments,strings,keywords]
%%% %%
%%% \lstnewenvironment{tiger}[1][]{\lstset{language=Tiger,basewidth={.5em,.4em},columns=flexible,#1}\upshape}{}
%%% \newcommand\inputtiger[2][]{\lstinputlisting[language=Tiger,basewidth={.5em,.4em},columns=flexible,#1]{#2}}
%%% \newcommand\tigerc{\lstinline[language=Tiger]}

%% Identifiers
\newcommand{\set}[1]{\textit{#1}}
\newcommand{\op}[1]{\ensuremath{\operatorname{#1}}}

\usepackage[all]{xy}
\SelectTips{eu}{}

\makeatletter
\def\compacttableofcontents{{\if@twocolumn\else \quotation\small \fi
    \noindent\textbf{Contents:}~%
    \def\protect{\relax}%
    \def\contentsline##1{\csname contentslineA##1\endcsname}%
    \def\numberline##1{\DN@{##1}\ifx\next@\empty\else##1.~\fi}%
    \def\fresh##1{\relax \let\fresh=\nextfresh}%
    \def\nextfresh##1{\relax##1 }%
    \def\contentslineAsection##1##2##3{\fresh,##1~(##2)\ignorespaces}%
    \def\contentslineAsubsection##1##2##3{\ignorespaces}%
    \@starttoc{toc}.\endquotation}}
\makeatother

\def\version{\thanks{\textbf{UNFINISHED DRAFT---Feedback Appreciated}.}}

%% Style.
\usepackage[margin=1in]{geometry}
\usepackage[T1]{fontenc}
\usepackage{charter}
\SelectTips{eu}{}
\bibliographystyle{plainurl}

%% Topmatter.
\title{A Gentle Introduction to\\\emph{Compiler Generation}\\Using \HAX\version}
\author{Kristoffer H. Rose}

\begin{document}
\maketitle

\begin{abstract}\noindent
  Higher-order Attribute Contraction Schemes---or \HAX---is a language for programming compilers.
  With \HAX it is possible to create a fully functional compiler from a single source file.  This
  document explains how to get \HAX up and running, and walk through the code of a simple example
  with each of the main stages of a compiler in \HAX: lexical analysis, syntax analysis, semantic
  analysis, intermediate code generation and optimization, and code generation.
\end{abstract}

\compacttableofcontents


\section{Introduction}

\HAX abbreviates \emph{Higher-order Attribute Contraction Schemes}, which is a formal system for
symbolic rewriting extended with programming idioms commonly used in compiler rewriting.  This
introduction briefly outlines how \HAX is structured and how to run the \HAX system.

A compiler written in \HAX consists of a single program with a series of formal sections, each
corresponding to a stage of the compiler.  Each section is written in a formal style suitable for
that stage of the compiler. Specifically, \HAX supports the following notations:
%%
\begin{description}

\item[Regular Expressions.] Used to describe how an input text is partitioned into tokens. The
  regular expresion notation of \HAX follows common
  conventions~\cite{Aho+:2006}. Section~\ref{sec:tokens} gives details of this notation.

\item[Context Free Grammars.] \HAX uses a form of BNF~\cite{NaurEtal:cacm1960} with common
  extensions to describe context free grammars. \HAX includes simple mechanisms for automatic
  resolution of operator precedence and directly recursive productions such that the transformation
  from token stream to abstract syntax can be formalized.  Detailed are in Section~\ref{sec:syntax}.

\item[Attribute Grammars.] Analyses can be described with attribute grammars in the style of
  \emph{Syntax-Directed Definitions}~\cite{Aho+:2006}, which describe how properties propagate
  through the abstract syntax tree.  Section~\ref{sec:synthesized} details how the basic propagation
  rules work for synthesized attributes, Section~\ref{sec:inherited} how inherited atributes are
  described, and Section~\ref{sec:symbols} the special considerations needed when managing scoped
  symbols and binders.

\item[Recursive Translation Schemes.] Code generation and other translations from one abstract form
  to another is formalized in \HAX using a variant of \emph{Syntax-Directed
    Translation}~\cite{Aho+:2006}, where the syntax of the source language is used to guide the
  generation of expressions in the target language. \HAX specifically uses higher order rewriting
  idioms to describe such translations as described in Section~\ref{sec:translation}.

\end{description}




In this section we make sure you have a functional \HAX installation on your computer.

\begin{requirements}
  To run the \HAX examples here you need a *nix system (including a shell and the usual utilities)
  with these common programs: a Java development environment (at
  least Java~1.6~SE SDK, with "java" and "javac" commands); a C99 compiler with the \emph{icu}
  libraries; the \emph{flex} lexical generator; the "wget" internet retrieval and "unzip" unpacking
  utility; and the GNU "make" command.  In addition, the setup process needs internet access to retrieve
  the JavaCC parser generator~\cite{javacc}.
\end{requirements}

\begin{commands}\label{com:all}
  Retrieve the \emph{hacs.zip} archive, extract it to a new directory, and intialize the "Makefile",
  for example with the following commands:\footnote{User input is \textcolor{blue}{blue}.}
  %%
\begin{code}[commandchars=\^\{\}]
$ ^textcolor{blue}{^texttt{mkdir myfirst}}
$ ^textcolor{blue}{^texttt{cd myfirst}}
$ ^textcolor{blue}{^texttt{wget http://krisrose.net/hacs.zip}}
$ ^textcolor{blue}{^texttt{unzip hacs.zip}}
$ ^textcolor{blue}{^texttt{cp Makefile-hacs-template Makefile}}
\end{code}
  %%
  Now check that all the commands defined in the "SETUP" section near the top of you freshly copied
  "Makefile" are correct; if any need changing then uncomment the line and change it to reflect the
  required commands on your system. Then check that it works with these command, which will need
  Internet access the first time:\footnote{Note that the spacing of the output may vary---this is
    not a stable capability yet.}
  %% 
\begin{code}[commandchars=\^\{\}]
$ ^textcolor{blue}{^texttt{make first.run}}
...
$ ^textcolor{blue}{^texttt{./first.run --action=Compile \}}
              ^textcolor{blue}{^texttt{--term="^{initial := 1; rate := 1.0; position := initial + rate * 60;^}"}}
  LDF T, #1
  STF name, T_51
  LDF T_84, #1.0
  STF name_40, T_56
  LDF T_98, name_43
  LDF T_70, name_3
  LDF T_90, #60
  MULF T_96, T_62, T_86
  ADDF T_50, T_82, T_177
  STF name_23, T_66                       
\end{code}
  %%
  Congratulations---you just built your first compiler! (It may have taken a few minutes, as \HAX
  also performed the initial setup, which is only needed on the first run.)
\end{commands}

\begin{example}[module wrapper]
\end{example}

\begin{manual}[grammar structure]\label{man:structure}
  A \HAX compiler is specified as a single \emph{.hx} module file with the following structure:
\begin{code}[commandchars=\\\{\}]
module {\it\color{blue}modulename}
\{
  // {\it\color{blue}Lexical Analysis}
  // {\it\color{blue}Syntax Analysis}
  // {\it\color{blue}Semantic Analysis}
  // {\it\color{blue}Code Generator}
  // {\it\color{blue}Main}
\}
\end{code}
  the \emph{modulename} should be a string with a Java style fully qualified class name, where the
  last component is capitalized, like \verb|"net.sf.crsx.samples.gentle.First"| in the example
  above. The individual sections specify the compiler, and the possible contents is documented in
  the manual blocks throughout this document.
\end{manual}

\begin{plan}
  In the remainder of this document we introduce the most important features of the \HAX language by
  explaining the corresponding parts of the \emph{first.hx} specification used above, which is
  adapted from \cite[Fig. 1.7]{Aho+:2006}, as well as several other minor examples.  We explain
  lexical analysis in Sec.~\ref{sec:tokens}, syntax analysis in Sec.~\ref{sec:syntax}, basic
  semantic sorts, schemes, and rules in Sec.~\ref{sec:schemes}, semantic analysis in
  Sec.~\ref{sec:analysis} and Sec.~\ref{sec:analysis2}, Code Generation in Sec.~\ref{sec:cg}, and
  the invokation conventions in Sec.~\ref{sec:main}.  Appendices include tables of Unicode
  characters and current limitations (\ref{app:tables}), some bonus examples (\ref{app:bonus}) and
  explanations for a few cryptic error messages (\ref{app:errors}).
\end{plan}


\section{Lexical Analysis}
\label{sec:tokens}

Lexical analysis is the process of splitting the input text into tokens. \HAX uses a usual variation
of \emph{regular expressions} for this.  Unique tokens such as keywords can be declared along with
the syntax productions where they are used.

\begin{example}[tokens and white space]\label{ex:lexical}
  %%
  The first part of \emph{first.hx} is the lexical analyzer. It uses regular expression notation to
  define the tokens used as ``terminal symbols'' by the compiler.  Here is the relevant fragment for
  setting up the concrete syntax of integers, basic floating point numbers, identifiers, and white
  space:
  %%
\begin{code}[xleftmargin=1.66em,numbers=left]
/* 1. LEXICAL ANALYSIS. */

space [ \t\n] ;

token Int    | ⟨Digit⟩+ ;
token Float  | ⟨Digit⟩* "." ⟨Digit⟩+ ;
token Id     | ⟨Lower⟩+ ('_'? ⟨Int⟩)? ;

token fragment Digit  | [0-9] ;
token fragment Lower  | [a-z] ;
\end{code}
  %%
  The example illustrates the following particulars of \HAX lexical expressions:
  %%
  \begin{itemize}

  \item Declarations generally start with a keyword or two and are terminated by a ";" (semicolon).

  \item "token" declarations in particular have the token name followed by a regular expression
    between a "|" (vertical bar) and a ";" (semicolon).

  \item A regular expressions is a sequence of units, corresponding to the concatenation of
    sequences of characters that match the units.  Each unit can be a \emph{character class} such as
    "[a-z]", which matches a single character in the indicated range, a string such as "'.'", or a
    reference to a token or fragment such as "⟨Lower⟩".\footnote{The used Unicode characters are
      summarized in Appendix~\ref{nota:unicode}.}

  \item A "token fragment" declaration means that the defined token can only be used in other token
    declarations, and not in syntax productions defined below.

  \item Every regular expression component can be followed by a repetition marker "?", "+", or~"*".

  \item The regular expression for white space is setup by "space" followed by the regular
    expression of what to skip -- here spaces, tabs, and newlines, where \HAX uses backslash for
    escaping in character classes with usual C-style language escapes.

  \item \HAX supports C/Java-style comments.

  \end{itemize}
  %%
\end{example}

\begin{commands}[lexical analysis]
  The generated command, \emph{first.run}, from above, can be used as a lexical analyzer with two
  arguments: a \emph{token sort} and a \emph{token term}.\footnote{The command has more options that
    we shall introduce as we need them.}  Execution then proceeds by parsing the string following
  the syntax of the token. We can, for example, check the lexical analysis of a number:
\begin{code}[commandchars=\\\{\}]
$ \textcolor{blue}{./first.run --sort=Float --term=34.56}
34.56
\end{code}
  If there is an error, the lexical analyzer will inform us of this:
\begin{code}[commandchars=\%\{\}]
$ %textcolor{blue}{./first.run --sort=Int --term=34.56}
Exception in thread "main" java.lang.RuntimeException: net.sf.crsx.CRSException:
net.sf.crsx.parser.ParseException:
 Parse error in embedded firstInt term at line 1, column 27: ⟦ 34.56 ⟧ ...
 Encountered " <T_T_M_FLOAT> "34.56 "" at line 1, column 29.
Was expecting one of:
    <T_meta1_T_M_INT> ...
    "\u27e9" ...
    <T_T_M_INT> ...
\end{code}
  (where the trail of Java exceptions has been truncated: the important information is in the first
  few lines).
\end{commands}

\begin{manual}[lexical declarations]\label{man:token}
  %%
  A token is declared with the keyword "token" followed by the token (sort) name, a "|" (vertical
  bar), and a \emph{regular expression}, which has one of the following forms (with increasing order
  of precedence):
  %%
  \begin{enumerate}

  \item Several alternative regular expressions can be combined with further "|" characters.

  \item Concatenation denotes the regular expression recognizing concatenations of what matches the
    subexpressions.

  \item A regular expression (of the forms following this one) can be followed by a \emph{repetition
      marker}: "?" for zero or one, "+" for one or more, and "*" for zero or more.

  \item A simple word without special characters stands for itself.

  \item A string in single or double quotes stands for the contents of the string except that "\"
    introduces an \emph{escape code} that stands for the encoded character in the string.

  \item A stand-alone "\" followed by an \emph{escape code} stands for that character: escape codes
    include the usual C and Java escapes: "\n", "\r", "\a", "\f", "\t", octal escapes like "\177",
    special character escapes like "\\", "\'", \verb|\"|, and Unicode hexadecimal escapes like
    "\u27e9".

  \item A \emph{character class} is given in "[]", with these rules:
   \begin{enumerate}
    \item if the first character is \verb"^" then the character class is negated;
    \item if the first (after \verb"^") character is "]" then that character is (not) permitted;
    \item if a "\" followed by an \emph{escape code} is encountered then it stands for the encoded
      character;
    \item if two characters are connected with a \verb"-" (dash) then all characters in the
      indicated \emph{range} (inclusive) are permitted (or excluded).
    \end{enumerate}
    Note that a character class cannot be empty.

  \item The "." (period) character stands for the character class "[^\n]".

  \item A nested regular expression can be given in "()".

  \item An entire other token "T" can be included (by literal substitution, so recursion is not
    allowed) by writing "⟨T⟩" (the angle brackets are unicode characters U+27E8 and U+27E9). As a
    special convenience, tokens declared with "token fragment" can \emph{only} be used this way.

  \item The special declaration "space" defines what constitutes white space for the generated
    grammar. (Note that this does not influence what is considered space in the specification
    itself.) A spacing declaration permits the special alternative "nested" declaration for nested
    comments, the following defines usual C/Java style spacing with comments, for example:
\begin{code}[]
space [ \t\f\r\n] | nested "/*" "*/" | "//" .* ;
\end{code}

  \end{enumerate}
  %%
  Notice that spacing is not significant in regular expressions, except (1) in character classes,
  (2) in literal strings, (3) if escaped (as in "\ ").
  %%
\end{manual}


\section{Syntax Analysis}
\label{sec:syntax}

Once we have tokens, we can use \HAX to program a complete syntax analysis with a grammar that
specifies how the input text is decomposed according to a \emph{concrete syntax} and how the desired
\emph{abstract syntax tree} (AST) is constructed from that. Notice that \HAX does not provide a
``parse tree'' in the traditional sense, \ie, a tree that represents the full concrete syntax parse:
only the AST is built.  Grammars are structured following the \emph{sorts} of AST nodes, with
concrete syntax details managed through annotations and ``syntactic sugar'' declarations.

\begin{example}\label{ex:syntax}
  %%
  The second part of our \emph{first.hx} example is the syntax analyzer. Our small example source
  language merely has blocks, statements, and a few forms of expression, like so:
  %%
\begin{code}[xleftmargin=1.66em,numbers=left]
/* 2. SYNTAX ANALYSIS. */

sort Stat  | ⟦ ⟨Name⟩ := ⟨Exp⟩ ; ⟧ | ⟦ { ⟨Stat*⟩ } ⟧ ;

sort Exp   | ⟦ ⟨Exp@1⟩ + ⟨Exp@2⟩ ⟧@1
           | ⟦ ⟨Exp@2⟩ * ⟨Exp@3⟩ ⟧@2
           | ⟦ ⟨Int⟩ ⟧@3
           | ⟦ ⟨Float⟩ ⟧@3
           | ⟦ ⟨Name⟩ ⟧@3
           | sugar ⟦ (⟨Exp#⟩) ⟧@3 → # ;

sort Name  | symbol ⟦ ⟨Id⟩ ⟧ ;
\end{code}
  %%
  The grammar structures the input as three sorts: "Stat" for statements, "Exp" for expressions, and
  "Name" for names (which we shall need later for symbol tables).  In addition a ``derived'' sort,
  "Stat*", is used. \HAX grammars follow these conventions:\footnote{As before the details of the
    Unicode characters are in the appendix.}
  %%
  \begin{itemize}

  \item Each sort is defined by a "sort" declaration followed by a number of \emph{productions},
    each introduced by a "|" (bar). (The first "|" corresponds to what is usually written ``::='' or
    ``→'' in grammars.)

  \item Concrete syntax is enclosed in "⟦…⟧" (``double'' or ``white'' brackets). Everything inside
    double brackets should be seen as literal syntax, even "\" (backslash), \emph{except} for \HAX
    white space (corresponding to "[ \t\n\r]"), which is ignored, and fragments in "⟨…⟩" (angle
    brackets), which are special.

  \item References to \emph{nonterminals} (other productions) are wrapped in "⟨…⟩" (angle brackets).

  \item \emph{Precedence} is indicated with "@"$n$, where higher numbers $n$ designate higher
    (tighter) precedence.  Any reference in as well as the alternative itself may have a precedence
    in this way; in the example we establish that "*" binds tighter than "+", and that both
    operators are left recursive. Note that we specify the precedence of both the entire expression,
    after the "⟦⟧"s, and of each component, inside the "⟨⟩". (In fact where the precedence is
    omitted we could have written "@0".)

  \item The special "sugar" declaration expresses that the concrete syntax can use parentheses to
    raise the precedence of the enclosed expression to~3: it is the first example of a \emph{rewrite
      rule} with a "→" that we see, where we remark that the expression is marked "#" so we can use
    the "#" to indicate that it is extracted as the abstract result of concrete syntax with
    parenthesis.  (In fact the general rule is that when an "→" is used then all sort specifiers
    must be ``disambiguated'' with distinct markers like "#" or "#5" in this way.)

  \item The "Name" sort is defined as a "symbol", which is only allowed because the underlying "Id"
    token permits a trailing "_"$n$ (underscore and count); this permits the use as binders and
    automatic symbol generation.

  \end{itemize}
\end{example}

\begin{commands}
  %%
  We can parse an expression from the command line:
  %%
\begin{code}[commandchars=\\\{\}]
$ \textcolor{blue}{./first.run --sort=Exp --term="(2+(3*(4+5)))"}
2 + 3 * ( 4 + 5 )
\end{code}
  %%
  Notice that the printout differs slightly from the input term as it has been ``resugared'' from
  the AST with minimal insertion of parentheses.
  %%
\end{commands}

\begin{manual}[syntactic sorts]\label{man:syntax}
  %%
  Formally, \HAX uses the following notations for specifying the syntax to use for terms.
  %%
  \begin{enumerate}

  \item \HAX \emph{production names} are capitalized words, so we can for example use "Exp" for the
    production of expressions.  The name of a production also serves as the name of its \emph{sort},
    \ie, the semantic category that is used internally for abstract syntax trees with that root
    production.  If particular instances of a sort need to be referenced later they can be
    \emph{disambiguated} with a "#"$i$ suffix, \eg, "Exp#2", where $i$ is an optional number or
    other simple word.

  \item A sort is declared by one or more "sort" declarations of the name optionally followed by a
    number of \emph{abstract syntax production} alternatives, each starting with a~"|". A sort
    declaration sets the \emph{current sort} for subsequent declarations and in particular any
    stand-alone production alternatives. All sort declarations for a sort are cumulative.

  \item Double square brackets "⟦…⟧" (unicode U+27E6 and U+27E7) are used for \emph{concrete syntax}
    but can contain nested angle brackets "⟨…⟩" (unicode U+27E8 and U+27E9) with \emph{production
      references} like "⟨Exp⟩" for an expression (as well as several other things that we will come
    to later).  We for example write "⟦⟨Exp⟩+⟨Exp⟩⟧" to describe the form where two expressions are
    separated by a "+" sign.

  \item Concrete syntax specification can include "¶" characters to indicate where \emph{newlines}
    should be inserted in the printed output. (The system can also control indentation but that is
    not enabled yet.)

  \item A trailing "@"$p$ for some precedence integer $p$ indicates that either the subexpression or
    the entire alternative (as appropriate) should be considered to have the indicated precedence,
    with higher numbers indicating higher precedence, \ie, tighter association.  (For details on the
    limitations of how the precedence and left recursion mechanisms are implemented, see
    Appendix~\ref{lim:list}.)

  \item "sugar ⟦…⟧→…" alternatives specify equivalent forms for existing syntax: anything matching
    the left alternative will be interpreted the same as the right one (which must have been
    previously defined); references must be disambiguated.

  \item A simple sort which contains only a reference to a token, where furthermore the token is
    defined such that it can end with "_"$n$ (an underscore followed by a count), then the sort can
    be qualified as a "symbol" sort, which can be used for variables and binders.

  \end{enumerate}
  %%
\end{manual}


\section{Semantic Sorts and Schemes}
\label{sec:schemes}

When analyzing the AST, which we cover in the next section, we shall need some utility sorts and
functions. This section details how these are defined.

\begin{example}\label{ex:types}
  %%
  The third part of the \emph{first.hx} example has the semantic sorts and operations used. For our
  toy language that just means the notion of \emph{type} used and the way that types are unified.
  %%
\begin{code}[xleftmargin=1.66em,numbers=left]
/* 3. SEMANTIC SORTS & SCHEMES. */

sort Type | Int | Float ;

| scheme Unif(Type,Type) ;
Unif(Int, Int) → Int;
Unif(#t1, Float) → Float;
Unif(Float, #t2) → Float;
\end{code}
  %%
  The code declares a new sort, "Type", which is a \emph{semantic} sort because it does not include
  any syntactic cases: all the possible values (as usual listed after leading "|"s) are simple term
  structures written without any ⟦⟧s.  Structures are written with a leading ``constructor,'' which
  should be a capitalized word (the same as sort names), optionally followed by some ``arguments''
  in "()"s, where the declaration gives the sort for each argument.

  The semantic sort also includes a "scheme" declaration for the "Unif" constructor followed by an
  argument list with two "Type" arguments. The scheme declaration is followed by \emph{rules} of the
  form ``pattern→replacement,'' which must specify for each possible shape of "Unif"-construction
  how it should be simplified by the scheme.  Rules may include ``meta-variables'' starting with "#"
  (hash), like "#t1", to designate ``function arguments'' that should be copied from the pattern to
  the replacement.
  %%
\end{example}

\begin{manual}[raw terms, schemes, and rules]\label{man:raw}
  %%
  ``Raw'' declarations consist of the following elements:
  %%
  \begin{enumerate}

  \item A \emph{constructor} is a capitalized word (similar to a sort name but in a separate name
    space).

  \item A \emph{variable} is a lower case word (subject to scoping, described below).

  \item A sort can be given a \emph{semantic production} as a "|" (bar) followed by a \emph{form},
    which consists of a constructor name, optionally followed by a "()"ed ","-separated list of
    \emph{scope forms}, which each consist of a \emph{sort} optionally preceded by a \emph{binder
      form}, which is a list of sorts followed by a "." (dot). Thus in the most general case, a
    semantic production has the form
    %%
    \begin{equation*}
      \texttt{|}~C~\texttt{(}
      ~S_{11}\cdots S_{1n_1}~\texttt{.}~S_1~\texttt{,}
      …~\texttt{,}
      ~S_{m1}\cdots S_{mn_m}~\texttt{.}~S_m
      \texttt{)}
    \end{equation*}
    %%
    with $C$ a constructor name and all $S_i$ and $S_{ij}$ sort names. The $S_i$ declares the
    \emph{argument sort} for the $i$th argument of the construction term, and the $S_{ij}$ is the
    \emph{binder sort} of the $j$th binder for the $i$th argument; $m$ is the \emph{arity} of the
    construction and $n_i$ the \emph{rank} of the $i$th argument.

  \item A semantic production can be qualified as a "scheme", which marks the declared construction
    as a candidate for rewrite rules (defined below).

  \item A \emph{raw term} is either a \emph{construction}, a \emph{variable use}, or a
    \emph{meta-application}, as follows
    %%
    \begin{enumerate}

    \item A \emph{construction} term is a constructor name followed by an optional "()"ed
      ","-separated list of \emph{scope arguments}, which each consist of a term optionally preceded
      by a \emph{binder list}, which is a list of variables followed by a "." (dot).  So in the most
      general case, a term looks like this:
      %% 
      \begin{equation*}
        C~\texttt{(}
        ~x_{11}\cdots x_{1n_1}~\texttt{.}~t_1~\texttt{,}
        ~…\texttt{,}
        ~x_{m1}\cdots x_{mn_m}~\texttt{.}~t_m
        \texttt{)}
      \end{equation*}
      %%
      The ``$C$-construction'' is said to have the \emph{subterms} $t_1,…,t_m$, and the arity $m$
      and ranks $n_1…n_m$ must correspond to a semantic production.  If present, the binder prefix
      of each introduces the specified variables \emph{only} for the appropriate subterm modulo
      usual renaming, \ie, writing \texttt{A(x y.x, x y.y)} and \texttt{A(a b.a, a b.b)} and even
      \texttt{A(s t.s, t s.s)} all denote the same term following the conventions of
      \emph{α-equivalence}.  In a scope argument $x\texttt{.}t$ we say that occurrences of $x$ in
      $t$ are \emph{bound} by the binder.

    \item A \emph{variable use} term is a variable, subject to the usual lexical scoping rules.

    \item A \emph{meta-application} term is a \emph{meta-variable}, consisting of a "#" (hash)
      followed by a number or word and optionally by a meta-argument list of ","-separated terms
      enclosed in "[]". Examples include "#t1" (with no arguments), "#[a,b,c]", and "#1[OK,#]".

    \end{enumerate}

  \item A term can have a \emph{sort prefix}. So the term "Type Unif(Type #t1, Type Float)" is the
    same as "Unif(#t1,Float)" provided "Unif" was declared with the raw production
    "|Unif(Type,Type)".

  \item A \emph{rewrite rule} is a pair of terms separatede by "→" (arrow, U+2192), with a few
    additional constraints: in the rule $p→t$, $p$ must be a \emph{pattern}, which means it must be
    a construction term that has been declared as a "scheme" (syntactic or raw) and with the
    restriction that all contained arguments to meta-applications must be bound variables, and all
    meta-applications in $t$ must have meta-variables that also occur in $p$ with the same number of
    meta-arguments.

    Rule declarations must either occur with the appropriate current sort or have a pattern with a
    sort prefix.

  \item One rule per scheme can be prefixed with the qualifier "default". If so then the pattern can
    have no structure: all subterms of the pattern scheme construction must be plain
    meta-applications. Such a default rule is applied \emph{after} it has been ensured that all
    other rules fail for the scheme.

  \item Finally, a rule can be prefixed with the word "rule" for clarity, which can be followed by a
    raw term and a ":" (TODO: document possible name/option choices).

  \end{enumerate}
  %%
  Rules are used for \emph{rewriting}, a definition of which is beyond the scope of this document;
  please refer to the literature on higher order rewriting for details~\cite{Klop+:tcs1993}.
  %%
\end{manual}


\section{Semantic Analysis I}
\label{sec:analysis}

Once the AST is loaded we can analyze it.  We demonstrate this by fully developing a simple type
checking analysis.  In this section we deal with analysis of expressions. 

\begin{example}\label{ex:analysis}
  %%
  In Ex.~\ref{ex:syntax} we presented the abstract syntax of the small language processed by
  \emph{first.hx}. A type analysis of the expressions of language might look as follows as a
  standard SDT (syntax directed translation), where we use $E$ for the "Exp" non-terminal, and these attributes:
  \begin{itemize}
  \item $E.e$ inherited ``environment'' symbol table mapping names to types.
  \item $E.t$ synthesized type of expression.
  \end{itemize}
  %%
  In the notations of \cite{Aho+:2006}, the SDT can be specified something like this:
  %% 
  \begin{equation*}
    \begin{array}{r@{\,}l|lr}
      \hline
      \hline
      \multicolumn{2}{c|}{\textsc{Production}}  & \textsc{Semantic Rules} &\\
      \hline
      E &→ E_1 + E_2 & E_1.e=E.e; E_2.e=E.e; E.t = \op{Unif}(E_1.t, E_2.t) &(E1)\\[\jot]
      &\mid E_1 \ast E_2 & E_1.e=E.e; E_2.e=E.e; E.t = \op{Unif}(E_1.t, E_2.t) &(E2)\\[\jot]
      &\mid \textbf{int} & E.t = \op{Int}&(E3)\\[\jot]
      &\mid \textbf{float} & E.t = \op{Float}&(E4)\\[\jot]
      &\mid \textbf{name} & E.t = \text{if}~\op{defined}(E.e,\textbf{name}.sym)~\text{then}~\op{lookup}(E.e,\textbf{name}.sym)&(E5)\\
      && \qquad\quad\text{else} \op{TypeError}\\
      \hline
    \end{array}
  \end{equation*}
  %%
  The SDT has one locations where the attribute dependencies are interesting: 
  $(E5)$ defines the synthesized attribute $E.t$ in terms of the inherited attribute $E.e$ in a
  ``down then up'' configuration.  To translate this into \HAX, we create a recursive scheme to
  propagate inherited attributes as needed.

  The following code contains the fourth part of \emph{first.hx}, which implements the SDT in
  \HAX.
  %%
%%  \begin{figure}[p]\small
\begin{code}[xleftmargin=1.66em,numbers=left]
/* 4. SEMANTIC EXPRESSION ANALYSIS. */

attribute ↑t(Type);       // synthesized expression type
attribute ↓e{Name:Type};  // inherited type environment

// TA, expression case.

sort Exp | scheme ⟦ TA ⟨Exp⟩ ⟧ ↓e ;

⟦ TA ⟨Int#⟩ ⟧ → ⟦ ⟨Int#⟩ ⟧;
⟦ TA ⟨Float#⟩ ⟧ → ⟦ ⟨Float#⟩ ⟧;
⟦ TA (⟨Exp#1⟩ + ⟨Exp#2⟩) ⟧ → ⟦ (TA ⟨Exp#1⟩) + (TA ⟨Exp#2⟩) ⟧;
⟦ TA (⟨Exp#1⟩ * ⟨Exp#2⟩) ⟧ → ⟦ (TA ⟨Exp#1⟩) * (TA ⟨Exp#2⟩) ⟧;
⟦ TA ⟨Name id⟩ ⟧ ↓e{id : #t} → ⟦ id ⟧ ↑t(#t);
⟦ TA ⟨Name id⟩ ⟧ ↓e{¬id} → error⟦Undefined identifier ⟨id⟩⟧;

sort Exp; ↑t;
⟦ ⟨Exp#1 ↑t(#t1)⟩ + ⟨Exp#2 ↑t(#t2)⟩ ⟧ ↑t(Unif(#t1,#t2));
⟦ ⟨Exp#1 ↑t(#t1)⟩ * ⟨Exp#2 ↑t(#t2)⟩ ⟧ ↑t(Unif(#t1,#t2));
⟦ ⟨Int#⟩ ⟧ ↑t(Int);
⟦ ⟨Float#⟩ ⟧ ↑t(Float);
\end{code}
%%    \caption{Semantic Expression Analysis.}
%%    \label{fig:semantic}
%%  \end{figure}
%%  %%
%%  Fig.~\ref{fig:semantic} contains the fourth part of \emph{first.hx}, which implements the SDT

%%  above in \HAX. 
  The first block sets up the synthesized $E.t$ and inherited $E.e$ attributes, using
  "↑" and "↓" for synthetic/inherited, respectively, and distinguishing between a value "(Type)" and
  symbol mapping "{Name:Type}".

  This is followed by a definition of the recursive "TA" scheme for \emph{distributing} all types in
  the symbol table to actual symbol occurrences. Specifically, "scheme⟦TA⟨Exp⟩⟧↓e" means that (the
  sort "Exp") has a scheme with the specified syntax that distributes the inherited attribute "e".
  Note that the scheme uses \emph{concrete syntax} for applications of the scheme: this is a
  convenience that makes it possible to invoke specific schemes directly from expressions entered in
  the command line. It does mean that we have to be careful with using the syntax correctly: we have
  not, for example, given a precedence to "TA", so in the recursive rules we have expressions like
\begin{code}[xleftmargin=5pc]
    ⟦ (TA⟨Exp#1⟩) + (TA⟨Exp#2⟩) ⟧
\end{code}
  which uses the "(…)" syntactic sugar to properly nest the expression.  We could also have written
\begin{code}[xleftmargin=5pc]
    ⟦ ⟨Exp ⟦TA ⟨Exp#1⟩⟧⟩ + ⟨Exp ⟦TA ⟨Exp#2⟩⟧⟩ ⟧
\end{code}
  using nested "⟨…⟩" constructs instead, which does not depend on the syntactic sugar.

  Otherwise, the scheme just exploits the syntax to express \emph{where} the recursion goes, which
  then automatically means that the inherited attribute will follow, as \HAX inherited attributes
  always follow a scheme.  The first two ``"TA"-rules'' have the cases for looking up a symbol in
  the environment: if it is there, \ie, if the "id" specified can make the environment pattern
  "↓{id:#t}" match, instantiating "#t" to the type of "id", then the first rule is used and the
  result is the "id" with the added synthesized attribute value "↑t(#t)", \ie, the $t$ attribute
  gets the value we value we extracted from the environment. The rule with the negated pattern
  "↓e{¬id}" will match if the "id" symbol is \emph{not} defined in the environment, then printing a
  simple error message.  In a sense the two "Name" cases correspond to $(E5)$ in the SDT.

  Finally, we associate the synthetic attribute "t" to the "Exp" sort---all synthetic attributes are
  associated with one or more sorts---and we give a special \emph{synthesis rule} that shows for
  each form of "Exp", where "t" is not populated by "TA", how synthetic attributes matches on
  fragments propagate to the entire expression, \eg, the rule for the SDT's $(1)$ is
\begin{code}[xleftmargin=1.66em]
⟦ ⟨Exp#1 ↑t(#t1)⟩ + ⟨Exp#2 ↑t(#t2)⟩ ⟧ ↑t(Unif(#t1,#t2));
\end{code}
  and should be read ``When considering an "Exp" (the current sort) which has the shape
  $⟦(⟨Exp⟩+⟨Exp⟩⟧$ where furthermore the first expression has a value matching "#t1" for the
  synthesized attribute "t", and the second expression has a value matching "#t2" for the
  synthesized attribute "t", then the entire expression has the value "Unif(#t1,#t2)" for the
  synthesized attribute "t".''  Notice that the attribute patterns have the same shape as the
  attribute declaration---a recurring feature of \HAX. The folowing three lines correspond in the
  same way to $(E2,E3,E4)$.
  %%
\end{example}

\begin{manual}[attributes and synthesis rules]\label{man:attributes}\leavevmode
  %%
  \begin{enumerate}

  \item Attributes are declared by "attribute" declarations followed by an \emph{attribute form} of
    one of the following shapes:
    %%
    \begin{enumerate}
    \item "↑Name(ValueSort)" defines that the synthesized attribute "Name" has "ValueSort" values;
    \item "↓Name(ValueSort)" similarly for a simple inherited attribute;
    \item "↓Name{SymbolSort:ValueSort}" defines the inherited symbol table attribute "Name" which
      for each constant or variable of "SymbolSort" has a distinct "ValueSort" value.
    \end{enumerate}

  \item One can add a simple \emph{synthesized attributes} after a raw data term as
    "↑"\emph{name}"("\emph{value}")", where the \emph{name} is an attribute name and the
    \emph{value} can be any term.

  \item Simple \emph{inherited attributes} are added similarly after a raw scheme term as
    "↓"\emph{name}"("\emph{value}")".

  \item An \emph{inherited symbol table attribute extension} is added to a raw scheme term as
    "↓"\emph{name}"{"\emph{symbol}":"\emph{value}"}", where the \emph{symbol} is either a variable
    or a constant (of the appropriate sort).

  \item A \emph{synthesized attribute reference} has the simple form "↑"\emph{name}";" and declares
    that the current sort synthesizes \emph{name} attributes.

  \item A scheme declaration can include \emph{inherited attribute references} of the form
    "↓"\emph{name}, which declares that the scheme inherits the \emph{name} attributes.

  \item A \emph{synthesis rule} is a special rule of the form $t↑name(t')$, where the term $t$ may
    contain subterms with attribute constraints. The rule specifies how terms of the current sort
    and shape $t$ synthesize \emph{name} attributes.

  \end{enumerate}
  %%
  Inherited attributes are managed with regular rules (for schemes) with inherited attribute
  constraints and extensions.
  %%
\end{manual}

\begin{manual}[parsed terms]\label{man:parsed}
  Man.~\ref{man:raw} documented terms without concrete syntax. The full term model combines this
  with \emph{parsed terms}.
  %%
  \begin{enumerate}

  \item Double square brackets "⟦…⟧" (unicode U+27E6 and U+27E7) can be used for \emph{concrete
      terms}, provided the \emph{sort} is clear, either
    \begin{enumerate}
    \item by immediately prefixing with the sort (as in "Exp⟦1+2⟧"), or
    \item by using as the argument of a defined constructor (as "IsType(⟦mytype⟧)"), or
    \item by usng as an attribute value, or
    \item by using as a top level rule pattern or replacement term with a defined current sort.
    \end{enumerate}

  \item Concrete terms can contain nested raw terms in "⟨…⟩" (unicode U+27E8 and U+27E9). Such
    nested raw terms \emph{must} have an explicit sort prefix.

  \item The special term "error⟦…⟧" will print the error message embedded in "⟦…⟧", where one is
    permitted to embed "symbol"-declared variables in "⟨…⟩".

  \end{enumerate}

\end{manual}


\section{Semantic Analysis II}
\label{sec:analysis2}

We continue the analysis by explaining how the environment is initialized and maintained across
statements, where new names are introduced.

\begin{example}\label{ex:analysis2}
  %%
  Our type analysis from Ex.~\ref{ex:analysis} is extended to $S,S^*$ for the non-terminals for the
  sorts "Stat" and "Stat*", respectively, with these attributes:
  \begin{itemize}
  \item $S.e, S^*.e$ inherited ``environment'' symbol table mapping names to types.
  \item $S.sym$ synthesized ``defines'' attribute with name introduced by $S$, or $ε$ if none.
  \item $S.t$ synthesized type of $S.sym$, if any.
  \end{itemize}
  %%
  with these rules:
  %% 
  \begin{equation*}
    \begin{array}{r@{\,}l|lr}
      \hline
      \hline
      \multicolumn{2}{c|}{\textsc{Production}}  & \textsc{Semantic Rules} &\\
      \hline
      S &→ \textbf{name} := E & E.e = S.e; S.sym = \textbf{name}.sym; S.t = E.t &(1)\\[\jot]
      &\mid \{~S^*~\} & S^*.e = S.e; S.sym = ε; S.t = ε &(2)\\[\jot]
      %%
      S^* &→ S_1~S^*_2 & S_1.e = S^*.e &(3)\\
      &&S^*_2.e = \text{if $S_1.sym≠ε$ then extend($S^*.e$, $S_1.sym$, $S_1.t$) else $S^*.e$}\\[\jot]
      &\mid ε & &(4)\\[\jot]
      \hline
    \end{array}
  \end{equation*}
  %%
  The SDT has one locations where the attribute dependencies are interesting: $(3)$ copies the
  synthesized $S_1.sym$ and $S_1.t$ ``left to right'' into the inherited attribute $S^*_2.e$.  To
  translate this into \HAX, we create a specific scheme for each ``left to right'' dependency.  Also
  note that in this particular SDT, the $S.sym$ and $S.t$ attributes are used in a trivial way to
  just allow the containing $S^*$ sequence to extract information from assigments $\textbf{name}:=E$
  and avoid nested blocks.

  Here is the code with the fifth part of \emph{first.hx}, which implements the SDT above in \HAX.
  %%
\begin{code}[xleftmargin=1.66em,numbers=left]
/* 5. SEMANTIC ANALYSIS. */

// TA, Statement case.

sort Stat | scheme ⟦ TA ⟨Stat⟩ ⟧ ↓e ;

⟦ TA ⟨Name id⟩ := ⟨Exp#2⟩; ⟧ → ⟦ ⟨Name id⟩ := TA ⟨Exp#2⟩; ⟧;

⟦ TA {} ⟧ → ⟦{}⟧;

⟦ TA { ⟨Name id⟩ := ⟨Exp#2⟩; ⟨Stat*#3⟩ } ⟧ → ⟦ TA2 { ⟨Name id⟩ := TA ⟨Exp#2⟩; ⟨Stat*#3⟩ } ⟧;
{
  | scheme ⟦ TA2 ⟨Stat⟩ ⟧ ↓e;
  ⟦ TA2 { ⟨Name id⟩ := ⟨Exp#2 ↑t(#t2)⟩; ⟨Stat*#3⟩ } ⟧ →
    ⟦ { ⟨Name id⟩ := ⟨Exp#2⟩; ⟨Stat ⟦TA {⟨Stat*#3⟩}⟧ ↓e{id:#t2}⟩ } ⟧;
}

⟦ TA { {⟨Stat*#1⟩} ⟨Stat*#2⟩ } ⟧ → ⟦ { TA {⟨Stat*#1⟩} TA {⟨Stat*#2⟩} } ⟧;
\end{code}
  For $(2,3,4)$ a slight refactoring is in order.  Imagine that we could have written the following
  pseudo-SDT instead, inlining the leading statement of a sequence and unfolding the rather silly
  ``one-step'' $S.sym$ and $S.t$ attributes and their ``absent'' $ε$ value:
  %%
  \begin{equation*}
    \begin{array}{r@{\,}l|lr}
      \hline
      S &→ \textbf{name} := E & E.e = S.e &(1)\\[\jot]
      &\mid \{~S^*~\} & S^*.e = S.e &(2)\\[\jot]
      S^* &→ ε & &(3)\\[\jot]
      &\mid \textbf{name} := E; S^*_2 & S^*_2.e = \op{extend}(S^*.e, \textbf{name}.sym, E.t) &(4.1)\\[\jot]
      &\mid \{S^*_1\} S^*_2 & S^*_2.e = S^*_1.e &(4.2)\\
      \hline
    \end{array}
  \end{equation*}
  %%
  This is not a usual SDT, because the cases do not follow the productions strictly, however, it is
  easy to see that these rules cover the full language just in a slightly different way.  This
  trick, which we can call ``inlining the sequence head,'' is frequently very useful.  For
  $(1,2,3,4.2)$ these modified rules are straightforward propagations, implemented directly in the
  code.  (The code in fact goes even further than this: in order to keep the "TA" scheme on just
  plain "Stat" they implement the cases for $S^*$ by reconstructing a single statement $\{S^*\}$ in
  each case.)

  Next, we handle $(4.1)$, the ``left-to-right'' dependency, where the inherited attribute $S^*_2.e$
  depends on the synthesized attribute $E.t$. In \HAX this means that a \emph{wrapper scheme} is
  needed, and we implement $(4.1)$ as follows (with "Stat" the current sort):
\begin{code}
⟦ TA { ⟨Name id⟩ := ⟨Exp#2⟩; ⟨Stat*#3⟩ } ⟧ → ⟦ TA2 { ⟨Name id⟩ := TA ⟨Exp#2⟩; ⟨Stat*#3⟩ } ⟧;
{
  | scheme ⟦ TA2 ⟨Stat⟩ ⟧ ↓e;
  ⟦ TA2 { ⟨Name id⟩ := ⟨Exp#2 ↑t(#t2)⟩; ⟨Stat*#3⟩ } ⟧ →
    ⟦ { ⟨Name id⟩ := ⟨Exp#2⟩; ⟨Stat ⟦TA {⟨Stat*#3⟩}⟧ ↓e{id:#t2}⟩ } ⟧;
}
\end{code}
  The first rule recurses the "TA" scheme into the expression to associate types with any names. A
  new, second scheme, "TA2", then takes over once the synthesized attribute has been instantiated
  for the expression, passing the environment extension to the subsequent statement.  (The "{}"
  wrappers serve as a grouping for the ``current sort'' status, which here is not in fact changed.)

  Here is the breakdown of the "TA2" rule. The pattern is an instance of the "TA2" syntactic scheme
  with these fragments:
  \begin{itemize}
  \item "TA2"---the prefix marking this as an instance of the scheme.
  \item "{"…"}" wrapper making the contained "Stat*" sequence into a single "Stat" as required.
  \item "⟨Name id⟩ :=" an identifier being assigned as part of the first assignment statement;
    matching will remember the identifier symbol as "id".
  \item "⟨Exp#2 ↑t(#t2)⟩" is the assigned expression, which includes a constraint that the rule can
    only fire if there is in fact a type annotation that can match "#t2", corresponding to the idea
    that it has already been analyzed.
  \item ";" the end of the assignment statement.
  \item "⟨Stat*#3⟩" is the following statement sequence that we shall annotate next.
  \end{itemize}
  Then comes the arrow ("→") followed by the replacement term:
  \begin{itemize}
  \item "{"…"}" wrapper making the contained "Stat*" sequence into a single result "Stat" as
    required.
  \item "⟨Name#1⟩ := ⟨Exp#2⟩;" is the assignment that we matched, unchanged (omitting the attribute
    means it is not modified).
  \item "⟨Stat …⟩" denotes the replacement follow-on (single) statement, where the ``"…"'' in this
    case is an embedded computation, namely---
  \item "⟦TA {⟨Stat*#3⟩}⟧ ↓e{id:#t2}" is the embedded computation to perform to annotate the
    following statements properly. It is a recursive call of "TA" on a single statement containing
    the remainder of the statements, annotated with the inherited type environment extended with the
    binding from the symbol to its type.
  \end{itemize}
  Notice how the environment extension uses nested ⟦⟧-brackets because the invocation of "TA" on the
  following statement needs the inherited attribute argument, which is not permitted in the language
  syntax.
  %%
\end{example}

\begin{manual}[attributes and synthesis rules]\leavevmode
  %%
  \begin{enumerate}

  \item Attributes are declared by "attribute" declarations followed by an \emph{attribute form} of
    one of the following shapes:
    %%
    \begin{enumerate}
    \item "↑Name(ValueSort)" defines that the synthesized attribute "Name" has "ValueSort" values;
    \item "↓Name(ValueSort)" similarly for a simple inherited attribute;
    \item "↓Name{SymbolSort:ValueSort}" defines the inherited symbol table attribute "Name" which
      for each constant or variable of "SymbolSort" has a distinct "ValueSort" value.
    \end{enumerate}

  \item One can add a simple \emph{synthesized attributes} after a raw data term as
    "↑"\emph{name}"("\emph{value}")", where the \emph{name} is an attribute name and the
    \emph{value} can be any term.

  \item Simple \emph{inherited attributes} are added similarly after a raw scheme term as
    "↓"\emph{name}"("\emph{value}")".

  \item An \emph{inherited symbol table attribute extension} is added to a raw scheme term as
    "↓"\emph{name}"{"\emph{symbol}":"\emph{value}"}", where the \emph{symbol} is either a variable
    or a constant (of the appropriate sort).

  \item A \emph{synthesized attribute reference} has the simple form "↑"\emph{name}";" and declares
    that the current sort synthesizes \emph{name} attributes.

  \item A scheme declaration can include \emph{inherited attribute references} of the form
    "↓"\emph{name}, which declares that the scheme inherits the \emph{name} attributes.

  \item A \emph{synthesis rule} is a special rule of the form $t↑name(t')$, where the term $t$ may
    contain subterms with attribute constraints. The rule specifies how terms of the current sort
    and shape $t$ synthesize \emph{name} attributes.

  \end{enumerate}
  %%
  Inherited attributes are managed with regular rules (for schemes) with inherited attribute
  constraints and extensions.
  %%
\end{manual}

\begin{manual}[special replacement terms]
  The special term "error⟦…⟧" will print the error message embedded in "⟦…⟧", where one is permitted
  to embed "symbol"-declared variables in "⟨…⟩".
\end{manual}


\section{Intermediate and Code Generation}
\label{sec:cg}

After the analysis we are ready for generating code.

\begin{figure}[p]\small
\begin{code}[xleftmargin=1.66em,numbers=left]
/* 6. INTERMEDIATE CODE GENERATION. */

token T | T ('_' ⟨Int⟩)? ; // temporary

// Concrete syntax & abstract syntax sorts.

sort I_Progr | ⟦⟨I_Instr⟩ ⟨I_Progr⟩⟧ | ⟦⟧ ;

sort I_Instr | ⟦⟨Tmp⟩ = ⟨I_Arg⟩ + ⟨I_Arg⟩; ¶⟧
             | ⟦⟨Tmp⟩ = ⟨I_Arg⟩ * ⟨I_Arg⟩; ¶⟧
             | ⟦⟨Tmp⟩ = ⟨I_Arg⟩; ¶⟧
             | ⟦⟨Name⟩ = ⟨Tmp⟩; ¶⟧ ;

sort I_Arg | ⟦⟨Name⟩⟧
           | ⟦⟨Float⟩⟧
           | ⟦⟨Int⟩⟧
           | ⟦⟨Tmp⟩⟧ ;

sort Tmp | symbol ⟦ ⟨T⟩ ⟧ ;

// Translation scheme.

attribute ↓TmpType{Tmp:Type} ;

sort I_Progr ;

| scheme ⟦ ICG ⟨Stat⟩ ⟧ ↓TmpType ;
⟦ ICG id := ⟨Exp#2 ↑HasType(#t2)⟩; ⟧
  → ⟦ { ⟨I_Progr ⟦ICGExp T ⟨Exp#2⟩⟧ ↓TmpType{T:#t2}⟩ } id = T; ⟧ ;
⟦ ICG { } ⟧ → ⟦ ⟧;
⟦ ICG { ⟨Stat#s⟩ ⟨Stat*#ss⟩ } ⟧ → ⟦ { ICG ⟨Stat#s⟩ } ICG { ⟨Stat*#ss⟩ } ⟧ ;

| scheme ⟦ ICGExp ⟨Tmp⟩ ⟨Exp⟩ ⟧ ;

⟦ ICGExp T ⟨Int#1⟩ ⟧ → ⟦ T = ⟨Int#1⟩; ⟧ ;
⟦ ICGExp T ⟨Float#1⟩ ⟧ → ⟦ T = ⟨Float#1⟩; ⟧ ;
⟦ ICGExp T id ⟧ → ⟦ T = id; ⟧ ;

⟦ ICGExp T ⟨Exp#1⟩ + ⟨Exp#2⟩ ⟧
  → ⟦ {ICGExp T_1 ⟨Exp#1⟩} {ICGExp T_2 ⟨Exp#2⟩} T = T_1 + T_2; ⟧ ;

⟦ ICGExp T ⟨Exp#1⟩ * ⟨Exp#2⟩ ⟧
  → ⟦ {ICGExp T_1 ⟨Exp#1⟩} {ICGExp T_2 ⟨Exp#2⟩} T = T_1 * T_2; ⟧ ;

// Helper to flatten code sequence.
| scheme ⟦ {⟨I_Progr⟩} ⟨I_Progr⟩ ⟧;
⟦ {} ⟨I_Progr#3⟩ ⟧ → #3 ;
⟦ {⟨I_Instr#1⟩ ⟨I_Progr#2⟩} ⟨I_Progr#3⟩ ⟧ → ⟦ ⟨I_Instr#1⟩ {⟨I_Progr#2⟩} ⟨I_Progr#3⟩ ⟧;
\end{code}
  \caption{Intermediate Code Generation.}
  \label{fig:icgen}
\end{figure}

\begin{example}\label{ex:icgen}
  %%
  The sixth part of \emph{first.hx} is the translation from abstract syntax to the intermediate
  representation, shown in Fig.~\ref{fig:icgen}. The fragment contains the usual components: a
  syntax specification, rewrite schemes, and rewrite rules for the "ICG" scheme.

  The code only uses two new features: "¶" markers in the syntax to indicate newlines, and rules
  that introduce \emph{fresh} variables (of "Tmp" sort): when the replacement of a rule uses a
  symbol, which was not in the pattern, then this corresponds to \emph{generating} a new globally
  unique symbol. So each time the rule
\begin{code}
⟦ ICG id := ⟨Exp#2 ↑HasType(#t2)⟩; ⟧
  → ⟦ { ⟨I_Progr ⟦ICGExp T ⟨Exp#2⟩⟧ ↓TmpType{T:#t2}⟩ } id = T; ⟧ ;
\end{code}
  is used, "T" denotes a new so-called ``fresh'' symbol.  When printed, the various incarnations of
  "T" will be named "T_1", "T_86", \etc
  %%
\end{example}

\begin{figure}[p]\small
\begin{code}[xleftmargin=1.66em,numbers=left]
/* 7. CODE GENERATOR. */

sort A_Progr | ⟦ ⟨A_Instr⟩ ⟨A_Progr⟩ ⟧ | ⟦⟧ ;

sort A_Instr | ⟦ LDF ⟨Tmp⟩, ⟨A_Arg⟩ ¶⟧
             | ⟦ STF ⟨Name⟩, ⟨Tmp⟩ ¶⟧
             | ⟦ ADDF ⟨A_Arg⟩, ⟨A_Arg⟩, ⟨A_Arg⟩ ¶⟧
             | ⟦ MULF ⟨A_Arg⟩, ⟨A_Arg⟩, ⟨A_Arg⟩ ¶⟧ ;

sort A_Arg | ⟦ #⟨Float⟩ ⟧ | ⟦ #⟨Int⟩ ⟧ | ⟦ ⟨Name⟩ ⟧ | ⟦ ⟨Tmp⟩ ⟧ ;

sort A_Progr | scheme ⟦ CG ⟨I_Progr⟩ ⟧ ;

⟦ CG ⟧ → ⟦⟧ ;

⟦ CG T = ⟨I_Arg#1⟩ + ⟨I_Arg#2⟩ ; ⟨I_Progr#⟩ ⟧
  → ⟦ ADDF T, [⟨I_Arg#1⟩], [⟨I_Arg#2⟩] CG ⟨I_Progr#⟩ ⟧ ;

⟦ CG T = ⟨I_Arg#1⟩ * ⟨I_Arg#2⟩ ; ⟨I_Progr#⟩ ⟧
  → ⟦ MULF T, [⟨I_Arg#1⟩], [⟨I_Arg#2⟩] CG ⟨I_Progr#⟩ ⟧ ;
  
⟦ CG T = ⟨I_Arg#1⟩ ; ⟨I_Progr#⟩ ⟧
  → ⟦ LDF T, [⟨I_Arg#1⟩] CG ⟨I_Progr#⟩ ⟧ ;

⟦ CG name = T ; ⟨I_Progr#⟩ ⟧
  → ⟦ STF name, T CG ⟨I_Progr#⟩ ⟧ ;

sort A_Arg | scheme ⟦ [⟨I_Arg⟩] ⟧ ;
⟦ [T] ⟧ → ⟦ T ⟧ ;
⟦ [name] ⟧ → ⟦ name ⟧ ;
⟦ [⟨Float#1⟩] ⟧ → ⟦ #⟨Float#1⟩ ⟧ ;
⟦ [⟨Int#1⟩] ⟧ → ⟦ #⟨Int#1⟩ ⟧ ;
\end{code}
  \caption{Code Generation.}
  \label{fig:cgen}
\end{figure}

\begin{example}\label{ex:cgen}
  The seventh part of \emph{first.hx} is the final translation "CG" from the intermediate
  representation to assembly code. This uses no new features, and is shown in Fig.~\ref{fig:cgen},
  however, it is still worth a sanity check, walking through the "CG" scheme and checking that all
  syntactic cases are covered.
\end{example}

\begin{remark}[concrete \emph{vs.} raw syntax]
  In the presentation we have chosen to use \emph{concrete syntax} even for semantic
  operations. This has the advantage of allowing direct invocation of even complex structured
  calculations from the command line but it does ``pollute'' the syntax of the defined language.
  (Production versions of \HAX (not yet released) will have the option of generating parsers that
  ignore concrete syntax of schemes when running the compiler.) It is sometimes practical to define
  ``bridge schemes'' that make schemes available both in syntax and raw; we give an example of this
  in the following section.
\end{remark}


\section{Main}
\label{sec:main}

Finally, we put everything together.

\begin{example}
  The eighth and last fragment of \emph{first.hx} is the main compilation function, which pulls
  together the three stages defined in previous sections.
\begin{code}[xleftmargin=1.66em,numbers=left]
/* 8. MAIN. */

sort A_Progr | scheme ⟦ Compile ⟨Stat⟩ ⟧ ;
⟦ Compile ⟨Stat#1⟩ ⟧ → ⟦ CG ICG TA ⟨Stat#1⟩ ⟧ ;
\end{code}
  It is also possible to define wrapper schemes that can be invoked with an option, this is useful
  when the input term is read from a file where it is not obvious to include the compiler
  instructions.
\begin{code}[xleftmargin=1.66em,numbers=left,firstnumber=last]
| scheme Compile(Stat);
Compile(#stat) → ⟦ Compile ⟨Stat#stat⟩ ⟧ ;
\end{code}
  This is the "--action" we invoked back in Com.~\ref{com:all}.  Such wrapper raw schemes must have
  a single argument.
\end{example}

\begin{manual}[building and running]\label{man:run}\leavevmode
  %%
  To use \HAX you need a copy of the \emph{hacs} directory somewhere. If you are working on a \HAX
  specification, say \emph{mycompiler.hx}, then your working directory should have a \emph{Makefile}
  containing (at least) the following:
  %%
\begin{code}[xleftmargin=1.66em,numbers=left,commandchars=\\\{\}]
# Makefile for mycompiler.

# HACS configuration.
HACS = $(abspath \textcolor{blue}{\emph{hacs}})
include $(HACS)/Makefile-hx

# Dependencies.
mycompiler.run: mycompiler.hx
\end{code}
  %%
  where you have replaced \texttt{\textcolor{blue}{\emph{hacs}}} with the path to the \HAX directory
  on your system (as written the system expects to find \emph{hacs} as a local subdirectory of your
  working directory). With this setup, and a suitable \emph{mycompiler.hx}, you have the following
  options:
  %%
  \begin{enumerate}

  \item "make mycompiler.run" will generate the script \emph{mycompiler.run}, which implements the
    compiler you specify in \emph{mycompiler.hx}. Generation creates a number of support files,
    specifically
    \begin{enumerate}
    \item \emph{build} subdirectory has runtime resources needed by the script.
    \item \emph{src} subdirectory contains auxiliary Java resources used by the build process.
    \item \emph{mycompiler.crs-installed} and \emph{mycompiler.pg-installed} record when the
      generated parser and rewrite system were installed.
    \end{enumerate}
    Note that the first time, some utility programs are compiled from C to ensure that \HAX is fully
    enabled on your system.

  \item "make clean" will remove all temporaries not needed for running the compiler script.

  \item "make realclean" will remove all traces of the generated compiler.

  \item "make distclean" will remove all traces of the generated compiler as well as all generated
    \HAX tooling files. Do not do this unless you really mean it: recovering a useable \HAX system
    requires getting a fresh copy of \emph{hacs.zip} or half an hour of CPU time on nontrivial
    hardware.

  \end{enumerate}
  %%
  The generated script refers absolutely to files under the \emph{hacs} directory, so the generated
  \emph{mycompiler.run} script itself can be moved but the \emph{hacs} directory cannot.

  The script accepts a number of options:
  %% 
  \begin{enumerate}

  \item "--sort="\emph{Sort} sets the expected sort (and thus parser productions) for the input to
    \emph{Sort}. The input is read, normalized, and printed.

  \item "--action="\emph{Constructor} sets the computation for the compiler to \emph{Constructor},
    which must be a unary raw scheme; the argument sort of \emph{Constructor} defines the parser
    productions to use.  The input is read, wrapped in the action, normalized, and printed.

  \item "--term="\emph{term} use the \emph{term} as the input.

  \item "--input="\emph{file} reads the input from \emph{file}.

  \item "--output="\emph{file} sends the input to \emph{file} (the default is the standard output).

  \item "--verbose="\emph{n} sets the verbosity of the underlying \CRSX rewrite engine to $n$. The
    default is 0 (quiet) but 1--3 are useful (above 3 you get a lot of low level diagnostic output).

  \item "--parse-verbose" activates verbose output from JavaCC of the parsing.

  \end{enumerate}
  %%
  You must provide one of "--sort" or "--action", and one of "--term" and "--input".

  \HAX will eventually contain a convention for defining ``main'' sorts and schemes such that
  defaults can be provided for the configuration options.
\end{manual}



\appendix\small

\section{Tables}
\label{app:tables}

\begin{notation}[used unicode characters]\label{nota:unicode}\leavevmode
\begin{displaymath}
  \begin{tabular}{|c|c|c|}
    \hline
    \emph{Glyph} & \emph{Code Point} & \emph{Character} \\
    \hline
    {¬} & U+00AC & logical negation sign \\
    {¶} & U+00B6 & paragraph sign \\
    ↑ & U+2191 & upwards arrow \\
    → & U+2192 & rightwards arrow \\
    ↓ & U+2193 & downwards arrow \\
    ⟦ & U+27E6 & mathematical left white square bracket \\
    ⟧ & U+27E7 & mathematical right white square bracket \\
    ⟨ & U+27E8 & mathematical left angle bracket \\
    ⟩ & U+27E9 & mathematical right angle bracket \\
    \hline
  \end{tabular}
\end{displaymath}
If you are using a Linux computer (or any where the display is controlled by the X Window System),
then you can create a file called \emph{.XCompose}, which has your special characters. A minimal one
looks something like this (adjust for your locale, \etc):
\begin{code}[fontsize=\footnotesize]
include "/usr/share/X11/locale/en_US.UTF-8/Compose"
<Multi_key> <P> <P>                       : "¶" U00B6 # PARAGRAPH SIGN
<Multi_key> <u> <greater>                 : "↑" U2191 # UPWARDS ARROW
<Multi_key> <minus> <greater>             : "→" U2192 # RIGHTWARDS ARROW
<Multi_key> <d> <greater>                 : "↓" U2193 # DOWNWARDS ARROW
<Multi_key> <bracketleft> <bracketleft>   : "⟦" U27E6 # MATHEMATICAL LEFT WHITE SQUARE BRACKET
<Multi_key> <bracketright> <bracketright> : "⟧" U27E7 # MATHEMATICAL RIGHT WHITE SQUARE BRACKET
<Multi_key> <less> <period>               : "⟨" U27E8 # MATHEMATICAL LEFT ANGLE BRACKET
<Multi_key> <greater> <period>            : "⟩" U27E9 # MATHEMATICAL RIGHT ANGLE BRACKET
\end{code}
(We have included a slightly more extensive one in the \emph{gentle} directory.) You should then map
the \emph{MultiKey} key to some key on your keyboard in the keyboard options (this is done in the
keyboard settings), and then you are ready to enter the fancy characters as the indicated three-key
combinations. In some programs (like Eclipse) you need to select the ``Input Method'' to be the ``X
Input Method'' for this to work.
\end{notation}

\begin{manual}[limitations]\label{lim:list}\leavevmode
  \begin{itemize}

  \item At most one "nested" declaration per "token".

  \item Precedence can only be used on self references, \ie, "⟨E@2⟩" can only occur inside
    productions for the sort "E".

  \item It is not possible to use binders and left recursion in the same production with the same
    precedence.

  \item Only \emph{direct} left recursion is currently supported, \ie, the left recursion should be
    within a single production.

  \item Productions can share a prefix but only within productions for the same sort, and the prefix
    has to be literally identical unit by unit, \ie,
\begin{code}
sort S | ⟦ ⟨A⟩ then ⟨B⟩ then C ⟧
       | ⟦ ⟨A⟩ then ⟨B⟩ or else D ⟧ ;
\end{code}
    is fine but
\begin{code}
sort S | ⟦ ⟨A⟩ then ⟨B⟩ then C ⟧
       | ⟦ ⟨A⟩ ⟨ThenB⟩ or else D ⟧ ;
sort ThenB | ⟦ then ⟨B⟩ ⟧;
\end{code}
    is not.

  \item It is not possible to left-factor a binder (so multiple binding constructs cannot have the
    same binder prefix).

  \item Binders must occur to the left of all their occurrences.

  \item Repeated production references cannot be used in sorts that also use precedence, \ie, you
    cannot mix "⟨E*⟩" and "⟨E@2⟩" inside the productions for the sort "E".

  \item Variables embedded in "error⟦…⟧" instructions must start with a lower case letter.

  \item When using the "symbol" qualifier on a reference to a token then the token \emph{must} allow
    ending in "_"$n$ for $n$ any natural number.

  \item When using the same name for a symbol inside of "⟦…⟧" and the corresponding raw variable
    outside of the ⟦⟧, then the common symbol and variable name must be a plain word starting with a
    lower case letter.

  \item Special terms like "error⟦…⟧" cannot be used as raw subterms.

  \item The "default" rule qualifier is rather fragile and does not yet always work.

  \end{itemize}
\end{manual}

\section{Bonus Examples}
\label{app:bonus}

Some additional examples. \emph{Note: still need to be worked through.}

\begin{example}[\emph{crsx/samples/gentle/bool.hx}]\leavevmode
\inputhacs{../samples/bool.hx}
\end{example}

\begin{example}[\emph{crsx/samples/gentle/deriv.hx}]\leavevmode
\inputhacs{../samples/deriv.hx}
\end{example}


\section{Common Errors}
\label{app:errors}

\begin{error}[\HAX syntax]\leavevmode
\begin{code}
Exception in thread "main" java.lang.RuntimeException: net.sf.crsx.CRSException:
 Encountered " "." ". "" at line 35, column 6.
Was expecting one of:
    <MT_Repeat> ...
    "%Repeat" ...
    <MT_Attributes> ...
\end{code}
  Indicates a simple syntax errors in the \emph{.hx} file.
\end{error}

\begin{error}[user syntax]\leavevmode
\begin{code}
Exception in thread "main" java.lang.RuntimeException:
  net.sf.crsx.CRSException: net.sf.crsx.parser.ParseException:
mycompiler.crs: Parse error in embedded myDecSome term at line 867, column 42:
 ⟦ $TA_Let2b ⟨Dec (#d)⟩{ ⟨DecSome (#ds)⟩} ⟧ at line 867, column 42
 Encountered " "\u27e9" "\u27e8Dec (#d)\u27e9 "" at line 867, column 53
…
\end{code}
  This indicates a concrete syntax error in some parsed syntax---inside "⟦…⟧"---in the \emph{.hx}
  file. The offending fragment is given in double angles in the message. Check that it is correctly
  entered in the \HAX specification in a way that corresponds to a syntax production. Note that the
  line/column numbers refer to the generated \emph{build/…Rules.crs} file, which us not immediately
  helpful (this is a known bug). In error messages a sort is typically referenced as a lower case
  prefix followed by the sort name---here "myDecSome" indicates that the problem is with parsing the
  "DecSome" sort of the "My" parser.
\end{error}

\begin{error}[JavaCC noise]\leavevmode
\begin{code}
Java Compiler Compiler Version ??.??_?? (Parser Generator)
(type "javacc" with no arguments for help)
Reading from file net/sf/crsx/samples/gentle/FirstParser.jj . . .
Warning: Line 769, Column 51: Non-ASCII characters used in regular expression.
Please make sure you use the correct Reader when you create the parser,
 one that can handle your character set.
File "TokenMgrError.java" does not exist.  Will create one.
File "ParseException.java" does not exist.  Will create one.
File "Token.java" does not exist.  Will create one.
File "SimpleCharStream.java" does not exist.  Will create one.
Parser generated with 0 errors and 1 warnings.
Note: net/sf/crsx/samples/gentle/FirstParser.java uses or overrides a deprecated API.
Note: Recompile with -Xlint:deprecation for details.
\end{code}
  These are ``normal'' messages from JavaCC.
\end{error}

\begin{error}[missing library]\leavevmode
\begin{code}
gcc -std=c99 -g    -c -o crsx_scan.o crsx_scan.c
crsx.c:11:30: fatal error: unicode/umachine.h: No such file or directory
\end{code}
  The \HAX tools only use one library in C: ICU. You should get the \emph{libicu-dev} package (or
  similar) for your system.
\end{error}

\begin{error}[meta-variable mistake]\leavevmode
\begin{code}
Error in rule Tiger-Ty2222222222111_9148-1: contractum uses undefined meta-variable (#es)
Errors prevent normalization.
make: *** [pr3.crs-installed] Error 1
\end{code}
  A rule uses the metavariable "#es" in the replacement without defining it in the corresponding
  pattern.
\end{error}

\begin{error}[]\leavevmode
\begin{code}
/home/krisrose/Desktop/teaching/.../hacs/cookmain PG pr3.hxt > pr3.pg
cookmain: crsx.c:528: bufferEnd: Assertion
   `(((childTerm)->descriptor == ((void *)0)) ? 0 :
        (childTerm)->descriptor->arity) == bufferTop(buffer)->index' failed.
/bin/sh: line 1: 14278 Aborted
  (core dumped) /home/krisrose/Desktop/teaching/.../hacs/cookmain PG pr3.hxt > pr3.pg
\end{code}
  This indicates an arity error: a raw term in the \emph{.hx} file does not have the right number of
  arguments.
\end{error}


\bibliography{crs}

\end{document}


%% $Log: hacs-gently.tex,v $
%% Revision 1.29  2014/02/11 15:49:11  krisrose
%% Hash.crs fixed to allow linear use of hash.
%%
%% Revision 1.28  2014/01/26 21:14:34  krisrose
%% Compiled NumericEqual primitive fixed for rounding errors.
%%
%% Revision 1.27  2014/01/21 18:40:46  krisrose
%% Regenerated rulecompiler.
%%
%% Revision 1.26  2014/01/16 14:07:17  krisrose
%% Update compiled $[Decimal] to also handle double.
%%
%% Revision 1.25  2013/12/05 04:10:03  krisrose
%% Manual fix.
%%
%% Revision 1.24  2013/12/03 21:41:55  krisrose
%% Update of hacs.zip.
%%
%% Revision 1.23  2013/12/02 12:21:21  krisrose
%% Option fixes.
%%
%% Revision 1.22  2013/11/25 06:35:09  krisrose
%% HACS cleanup.
%%
%% Revision 1.21  2013/11/21 21:10:55  krisrose
%% Duplicate table row removed.
%%
%% Revision 1.20  2013/11/21 06:19:39  krisrose
%% HACS documentation rough version done.
%%
%% Revision 1.19  2013/11/21 04:01:32  krisrose
%% Newline support in HACS.
%%
%% Revision 1.18  2013/11/20 05:08:17  krisrose
%% HACS functional.
%%
%% Revision 1.17  2013/11/19 21:00:12  krisrose
%% first.hx refactored for hacs-gently runs again.
%%
%% Revision 1.16  2013/11/18 02:28:07  krisrose
%% HACS fully functional.
%%
%% Revision 1.15  2013/11/17 03:20:20  krisrose
%% reify option respects simple-terms.
%% Main .dr not output when modules generated.
%%
%% Revision 1.14  2013/10/30 04:28:28  krisrose
%% HACS synthesized attributes almost ready!
%%
%% Revision 1.13  2013/10/21 18:02:12  krisrose
%% HACS attribute grammar fixed.
%%
%% Revision 1.12  2013/10/17 16:51:43  krisrose
%% Locify avoids $[PassLocationProperties] when possible.
%%
%% Revision 1.11  2013/10/12 21:35:40  krisrose
%% Location and HACS work.
%% Checkpoint before major attribute distribution fix.
%%
%% Revision 1.10  2013/10/06 15:38:28  krisrose
%% First attempt at VariableUse with propserties.
%% Preparing for HACS for unmetaness.
%%
%% Revision 1.9  2013/09/30 11:06:42  krisrose
%% All HACS examples work again.
%%
%% Revision 1.8  2013/09/29 05:49:58  krisrose
%% Attributes next...
%%
%% Revision 1.7  2013/09/27 10:05:12  krisrose
%% Document RegExps.
%%
%% Revision 1.6  2013/09/25 17:14:15  krisrose
%% Change default action to Drop.
%%
%% Revision 1.5  2013/09/20 08:32:04  krisrose
%% Formatting.
%%
%% Revision 1.4  2013/09/20 08:17:57  krisrose
%% Drop diagnostic output.
%%
%% Revision 1.3  2013/09/20 08:11:58  krisrose
%% jar refresh
%%
%% Revision 1.2  2013/09/20 08:03:50  krisrose
%% Ready for the NYU students...
%%
%% Revision 1.1  2013/09/18 14:42:24  krisrose
%% Moved 'dragon' to 'gentle'.
%%
%% New.

%%---------------------------------------------------------------------
% Tell Emacs that this is a LaTeX document and how it is formatted:
% Local Variables:
% mode:latex
% fill-column:100
% End:
