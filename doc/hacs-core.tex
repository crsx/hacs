\documentclass[letterpaper,11pt]{article}
%% $Id: setup.tex,v 1.11 2013/12/05 04:10:03 krisrose Exp $

%% Format.
\usepackage[utf8]{inputenc}
\usepackage{comment}
\usepackage{amsmath,amssymb,stmaryrd,textcomp}
\allowdisplaybreaks %style
\usepackage[normalem]{ulem}
\usepackage{pgfpages}
\usepackage{utf8math}
\usepackage{url}
\usepackage{xspace}
\usepackage{fancyvrb}
\usepackage{xcolor}
\usepackage{graphicx}
\usepackage{hyperref}
\usepackage{rcs}

%% Macros.
\newcommand{\HAX}{\text{HACS}\xspace}
\newcommand{\CRSX}{\text{CRSX}\xspace}
\newcommand{\ie}{\textit{i.e.}\xspace}
\newcommand{\eg}{\textit{e.g.}\xspace}
\newcommand{\etc}{\textit{etc.}\xspace}
\newcommand{\aka}{\textit{aka.}\xspace}
\newcommand{\TBD}[1][]{\textit{To Be Done…#1}\marginpar{\bf\color{red} TBD}}

%% Misc.
\def\Bigstrut{\ensuremath{\vphantom{\Bigm|}}}
\newcommand\caret{\mathbin{\char`\^}}

%% Units
\usepackage{amsthm}
\swapnumbers %style
\newtheorem{theorem}{Theorem}[section]
\newtheorem{lemma}[theorem]{Lemma}
\theoremstyle{definition}
\newtheorem{definition}[theorem]{Definition}
\newtheorem{proposition}[theorem]{Proposition}
\newtheorem{example}[theorem]{Example}
\newtheorem{practice}[theorem]{Practice}
\newtheorem{notation}[theorem]{Notation}
\newtheorem{manual}[theorem]{Manual}
%\newenvironment{manual}{\small\begin{manual-body}}{\end{manual-body}}
\newtheorem{invariant}[theorem]{Invariant}
\newtheorem{requirements}[theorem]{Requirements}
\newtheorem{limitations}[theorem]{Limitations}
\newtheorem{commands}[theorem]{Commands}
\newtheorem{error}[theorem]{Error}
\newtheorem{note}[theorem]{Note}
\newtheorem{plan}[theorem]{Plan}
%\theoremstyle{remark}
\newtheorem{remark}[theorem]{Remark}

%% Old code listings.
\usepackage{alltt}
\usepackage{fancyvrb}
\CustomVerbatimEnvironment{code}{Verbatim}{tabsize=4,fontsize=\small,numberblanklines=false,xleftmargin=\parindent}
\CustomVerbatimCommand{\inputcode}{VerbatimInput}{tabsize=4,fontsize=\footnotesize,numbers=left,numberblanklines=false,xleftmargin=\parindent}
%\DefineShortVerb{\"} % makes " work in verbatim even when \active


%% New code listings...
\usepackage{listings}
\lstset{basicstyle=\ttfamily}

%% HACS.
\def\Identifier#1{#1}
\def\Keyword#1{\textsf{\color{gray}#1}}

\lstdefinelanguage{HACS}{basicstyle=\fontfamily{ccr}\selectfont,%
  extendedchars=true,inputencoding=utf8,sensitive,%
  identifierstyle=\Identifier,%
  string=[d]{"},morestring=[d]{'},upquote=true,showstringspaces=false,%
  comment=[l]{//},morecomment=[s]{/*}{*/},commentstyle=\color{blue}\rm,%
  keywordstyle=\Keyword,%
  keywords={attribute,data,fragment,import,main,module,nested,property,rule,scheme,simplify,sort,%
    space,static,sugar,symbol,tag,token,default,error,binds,as},%
  literate=%
    {⟦}{{\ensuremath{\llbracket}}}1 {⟧}{{\ensuremath{\rrbracket}}}1 {⟨}{{\ensuremath{\langle}}}1 {⟩}{{\ensuremath{\rangle}}}1 %
    {→}{{\ensuremath{\rightarrow}}}2 {↑}{{\ensuremath{\uparrow}}}1 {↓}{{\ensuremath{\downarrow}}}1 {¬}{{\ensuremath{\lnot}}}2 {¶}{{\P}}1 %
    {ε}{{\ensuremath{\varepsilon}}}1 {…}{{\ensuremath{\dots}}}2 {∧}{{\ensuremath{\!\wedge}}}2 {∨}{{\ensuremath{\!\vee}}}2 {↦}{{\ensuremath{\mapsto}}}2 %
}[comments,strings,keywords]
%%
\lstnewenvironment{hacs}[1][]{\lstset{language=HACS,basewidth={.53em,.43em},columns=flexible,numberstyle=\scriptsize,numberblanklines=false,firstnumber=auto,xrightmargin=1pc,#1}\upshape}{}
\newcommand\inputhacs[2][]{\lstinputlisting[language=HACS,basewidth={.53em,.43em},columns=flexible,numberstyle=\scriptstyle,numberblanklines=false,xrightmargin=1pc,#1]{#2}}
\newcommand\hacsc{\lstinline[language=HACS]}
\lstMakeShortInline[language=HACS,columns=fullflexible]"


%%% \usepackage{listings}
%%% 
%%% %% HACS.
%%% \lstdefinelanguage{HACS}{basicstyle=\rm,%
%%%   numberstyle=\scriptsize,
%%%   numbersep=7pt,
%%%   numberblanklines=false,
%%%   extendedchars=true,inputencoding=utf8,sensitive,%
%%%   identifierstyle=\textit,%
%%%   string=[d]{"},morestring=[m]{'},upquote=true,stringstyle=\ttfamily,%showstringspaces=true,%
%%%   comment=[l]{//},morecomment=[s]{/*}{*/},commentstyle=\color{gray}\upshape,%
%%%   keywordstyle=\color{black}\bfseries,%
%%%   keywords={attribute,data,default,fragment,import,main,module,nested,property,rule,scheme,simplify,sort,space,static,sugar,symbol,tag,token},%
%%%   ndkeywordstyle={\textup},%
%%%   literate=%
%%%     {⟦}{{\ensuremath{\llbracket}}}1 {⟧}{{\ensuremath{\rrbracket}}}1 {⟨}{{\ensuremath{\langle}}}1 {⟩}{{\ensuremath{\rangle}}}1 %
%%%     {→}{\ensuremath{\rightarrow}}2 {↑}{\ensuremath{\uparrow}}1 {↓}{\ensuremath{\downarrow}}1 {ε}{\ensuremath{\varepsilon}}1 %
%%%     {…}{\ensuremath{\dots}}2 {¬}{\ensuremath{\lnot}}1 {∧}{\ensuremath{\wedge}}2 {∨}{\ensuremath{\vee}}2 {¶}{\P}1 {↦}{\ensuremath{\mapsto}}2 %
%%% }[comments,strings,keywords]
%%% %%
%%% \lstnewenvironment{hacs}[1][]{\lstset{language=HACS,basewidth={.55em,.4em},columns=flexible,#1}\upshape}{}
%%% \newcommand\inputhacs[2][]{\lstinputlisting[language=HACS,basewidth={.55em,.4em},columns=flexible,#1]{#2}}
%%% \newcommand\hacsc{\lstinline[language=HACS]}
%%% \lstMakeShortInline[language=HACS,columns=fullflexible]"
%%% 
%%% %% Tiger.
%%% \lstdefinelanguage{Tiger}{basicstyle=\normalsize,%
%%%   extendedchars=true,inputencoding=utf8,sensitive,%
%%%   identifierstyle=\textit,%
%%%   string=[d]{"},upquote=true,stringstyle=\ttfamily,showstringspaces=true,%
%%%   comment=[s]{/*}{*/},commentstyle=\color{grey}\rm,%
%%%   keywordstyle=\color{black}\bfseries,%
%%%   keywords={type,array,of,var,function,nil,let,in,end}%
%%% }[comments,strings,keywords]
%%% %%
%%% \lstnewenvironment{tiger}[1][]{\lstset{language=Tiger,basewidth={.5em,.4em},columns=flexible,#1}\upshape}{}
%%% \newcommand\inputtiger[2][]{\lstinputlisting[language=Tiger,basewidth={.5em,.4em},columns=flexible,#1]{#2}}
%%% \newcommand\tigerc{\lstinline[language=Tiger]}

%% Identifiers
\newcommand{\set}[1]{\textit{#1}}
\newcommand{\op}[1]{\ensuremath{\operatorname{#1}}}

\usepackage[all]{xy}
\SelectTips{eu}{}

\makeatletter
\def\compacttableofcontents{{\if@twocolumn\else \quotation\small \fi
    \noindent\textbf{Contents:}~%
    \def\protect{\relax}%
    \def\contentsline##1{\csname contentslineA##1\endcsname}%
    \def\numberline##1{\DN@{##1}\ifx\next@\empty\else##1.~\fi}%
    \def\fresh##1{\relax \let\fresh=\nextfresh}%
    \def\nextfresh##1{\relax##1 }%
    \def\contentslineAsection##1##2##3{\fresh,##1~(##2)\ignorespaces}%
    \def\contentslineAsubsection##1##2##3{\ignorespaces}%
    \@starttoc{toc}.\endquotation}}
\makeatother

\usepackage{cite}
\usepackage{stmaryrd}

%% Style.
\bibliographystyle{plain}
\usepackage{charter}
\usepackage{geometry}

%% Topmatter.
\title{ Higher-order Attribute Contraction Schemes \\ Core Calculus }
\author{ Kristoffer H. Rose \\
 Two Sigma Investments, LLC \& New York University
 % \and
 % Cynthia Kop \\
 % Institute of Computer Science, University of Innsbruck \\
}

\begin{document}
\maketitle

\begin{abstract}\noindent
  Higher-order Attribute Contraction Schemes---or \HAX---is a practical formal language for
  specifying executable compilers and other program manipulation tools.
  %%
  Formally, \HAX is a general higher order rewriting formalism with extensions for managing the
  special data structures used by compilers, notably custom syntax with regular expressions, symbol
  tables (\aka environments), attribute grammars, and recursive translation schemes.
  %%
  This paper presents a \emph{core} calculus of \HAX, called \hax, a plank on which \HAX can be
  based, which encodes the low-level operational properties of \HAX into a small formal system.
\end{abstract}

%------------------------------------------------------------------------

\section{Introduction}\label{sec:intro}

Several systems that manipulate programs, so-called \emph{meta-programming} systems, have emerged
over the years, ranging from generic specification languages, where the goal is not to define how
but only declare the semantics of the program manipulation, all the way to tools that support
specific aspects of program execution or compiler generation (we comment on several in the
conclusion).

One such system was the CRSX system~\cite{Rose:1996} developed for industrial use at IBM Research by
a team led by the author~\cite{Rose:hor2010,Rose:rta2011,crsx} .  CRSX is a language based on
\emph{higher-order rewriting}~\cite{Jouannaud:klop2005} combined with \emph{higher-order abstract
  syntax} (HOAS) \cite{PfenningElliot:pldi1988} and extensions for handling environments and using
pluggable parsers.  The programming of the IBM Data\-Power XQuery compiler~\cite{dp60:ibm2013} using
CRSX proved that the approach can drastically reduce the development time of a compiler (the cited
XQuery compiler was estimated to be developed in a quarter of the traditional development time) as
well as resulting in a rather more compact and high level source program.

However, the CRSX notation, based on combinatory reduction systems~\cite{Klop:1980,Klop+:tcs1993},
which combines λ calculus~\cite{Church:1941,Barendregt:1984} and term rewriting
systems~\cite{Klop:1992:Handbook}, has proven to be unwieldy for several reasons, first of all by
being quite different from standard notations used in compiler construction reference
works~\cite{Aho+:2006}. Adding a polymorphic sort system to CRSX (effectively changing them to
\emph{contraction systems}~\cite{Aczel:1978} and be similar to Inductive-type-systems
\cite{BlanquiJouannaudOkada:tcs2002}) helped but did not make the system easy enough to, for
example, teach compiler programming.

\HAX stands for \emph{Higher-order Attribute Contraction Schemes}, and started as an attempt to
remedy this by providing a front-end to CRSX that allows the use of standard notations and concepts
of (formal) programming language descriptions to more directly program compilers and other systems
for manipulating code. As the name suggests, the notation combines HOAS with concepts from attribute
grammars~\cite{Knuth:mst1968} as well as other common compiler specification
notations~\cite{Aho+:2006}. \HAX has been successfully used to teach the graduate computer science
compiler construction class at New York University~\cite{RoseRose:cims2015}.

In this paper we specify \HAX through a ``core calculus'' as is common for functional languages, and
give a semantics for this core language.
\TBD{Plan}%
and finally Section~\ref{sec:conc} concludes and compares to related work.


%------------------------------------------------------------------------

\section{\hax Syntax}
\label{sec:hacs}

In this section we define the \hax calculus, with a minimal syntax that supports the \HAX source
language.

\begin{figure*}[t]
  \begin{align}
    \tag{Declaration}
    D &::= \kw{sort}~S \,\kw\{\, C^{*\kw,} \,\kw\}
    \bigm| \kw{attribute}~a~\set{AS}\,\kw;
    \\[-1pt]\notag~&
    \bigm| \kw{data}~c\,\kw(\,\set{BS}^{*\kw,}\,\kw)~a^*\,\kw;
    \bigm| \kw{scheme}~f\,\kw(\,\set{BS}^{*\kw,}\,\kw)~a^*\,\kw;
    \bigm| \kw{variable}~a^*\,\kw;
    \bigm| \kw{rule}~T~\kw{$→$}~T\,\kw;
    \\[\jot]
    \tag{BindersSort}
    \set{BS} &::= \kw[\,S^{*\kw,}\,\kw]\,S
    \\
    \tag{AttributeSort}
    \set{AS} &::= \kw(\,S\,\kw)
    \bigm| \kw\{\,S\,\kw:\,S\,\kw\}
    \\
    S &::= s\,\kw[\,S^{*\kw,}\,\kw]
    \bigm| α
    \tag{Sort}
    \\[\jot]
    \tag{Term}
    T &::= c\,\kw(\,B^{*\kw,}\,\kw)~A^*
    \bigm| v ~ A^*
    \bigm| m\,\kw[\,T^{*\kw,}\,\kw]~A^*
    \\
    \tag{Binders}
    B &::= \kw[\,v^{*\kw,}\,\kw]\,T
    \\
    A &::= a~\set{V}
    \tag{Attribute}
    \\
    \set{AV} &::= \kw(\,T\,\kw)
    \bigm| \kw\{\,T\,\kw:\,T\,\kw\}
    \bigm| \kw\{\kw\}
    \bigm| \kw\{\,\kw{$¬$}\,T\,\kw\}
    \bigm| \kw\{\kw:\,m\,\kw\}
    \tag{AttributeValue}
  \end{align}
  \caption{\hax syntax.}
  \label{fig:plank}
\end{figure*}

\begin{definition}[\hax]
  %%
  The \hax grammar is shown in Figure~\ref{fig:plank}. The top level of a \hax script is a sequence
  of declarations $D^*$, and the grammar assumes that we have three categories of tokens defined:
  %%
  \begin{itemize}

  \item $s,f,c ∈ \mathbb{C}$ stands for \emph{constructor} tokens, which are capitalized words like
    "Integer" or "A" or "CamelCaseWord", used for sort, function symbol, and data symbol names
    (respectively).

  \item $v,a,α ∈ \mathbb{V}$ stands for \emph{variable} tokens, which are lower case words like "x"
    or "foo" or even "lowWord", used for term variables, attributes, and sort variables (respectively).

  \item $m ∈ \mathbb{M}$ stands for \emph{meta-variable} tokens, which are tokens that start with
    "#" like "#arg" or "#BigFoot" or even just "#1" or merely~"#".

  \end{itemize}
  %%
  The ``$A^{*\kw,}$'' notation means zero or more $A$s separated by commas.
  %%
\end{definition}

\begin{figure*}[p]\small

  \begin{hacs}[numbers=right,texcl]
    sort N[] {                                                        // natural number sort
      data Z() ;                                                      // zero
      data S([]N[]) ;                                                // successor
      scheme Plus([]N[], []N[]) ;                                   // $+$ operator
      rule Plus([]Z(), []#2[]) →  #2[] ;                            // $0+x = x$
      rule Plus([]S([]#1[]), []#2[]) →  S([]Plus([]#1[], []#2[])) ; // $(1+x)+y=1+(x+y)$
    }
  \end{hacs}
  \vspace*{-1em}
  \caption{Peano numerals in \hax.}
  \label{fig:peano}

  \vspace{1em}
  \begin{hacs}[numbers=right,texcl]
    sort Λ[] {                                                       // sort of λ terms
      variable ;                                                     // permit variables
      scheme Ap([]Λ[], []Λ[]) ;                                     // application
      data Lm([Λ[]]Λ[]) ;                                          // abstraction
      rule Ap([]Lm([x]#body[x]), []#arg[]) → #body[#arg[]] ;  // β-reduction
    }
  \end{hacs}
  \vspace*{-1em}
  \caption{Untyped λ calculus in \hax.}
  \label{fig:lambda}

  \vspace{1em}
  \begin{hacs}[numbers=right,texcl]
    sort List[α] { data Nil(); data Cons([]α, []List[α]);
      scheme Appnd([]List[α], []List[α]);
      rule Appnd([]Nil(), []#2[]) →  #2[];
      rule Appnd([]Cons([]#11[], []#12[]), []#2[]) →  Cons([]#11[], Appnd([]#12[], []#2[]));
      scheme Map([β]α, []List[β]);
      rule Map([x]#[x], []Nil()) →  Nil();
      rule Map([x]#[x], []Cons([]#1[], []#2[])) →  Cons([]#[#1[]], Map([x]#[x], []#2[]));
    }
  \end{hacs}
  \vspace*{-1em}
  \caption{Polymorphic lists in \hax.}
  \label{fig:list}

  \vspace{1em}
  \begin{hacs}[numbers=right,texcl]
    sort {
    }
  \end{hacs}
  \vspace*{-1em}
  \caption{Attributes…}
  \label{fig:attrib}

\end{figure*}

\begin{example}[Peano]\label{ex:peano}
  The classical first order Peano arithmetic rules are a simple \hax system, shown in
  Figure~\ref{fig:peano}.  The example illustrates how a sort is defined with data constructors and
  a scheme for rewriting, essentially following standard notations for many-sorted term rewriting,
  except for the slightly unusual (meta)variable notation with "#"s. The example illustrates the
  following points:
  %%
  \begin{itemize}

  \item The defined sort is "N", written "N[]", as it has no sort parameters. All the other
    declarations define artifacts of the "N" sort, so are inside the \kw{sort} declaration's
    \kw{\{\}}s.

  \item There are two \kw{data} constructors: "Z" with no parameters, and "S" with a single numeric
    argument. The single numeric argument is specified as "[]N[]" because it has no locally scoped
    binders ("[]"$_-$) and the "N" sort has no sort parameters ($_-$"[]").

  \item There is a single function (\kw{scheme}) constructor: "Plus" with two numeric arguments.

  \item There are two \kw{rule}s for the function symbol, implementing the usual Peano addition rules.

  \end{itemize}
  %%
  A sample rewrite sequence using this system, corresponding to the computation $1+1=2$, is
  %%
  \begin{displaymath}
    "Plus([]S([]Z()), []S([]Z()))" →
    "S([]Plus([]Z(), []S([]Z())))" →
    "S([]S([]Z()))"
  \end{displaymath}
\end{example}

\begin{remark}
  One difference between the CRSX family, including \hax, and other higher order rewriting
  formalisms, is that the binder mechanism is part of the parent construction, \eg, the sort of the
  "S" constructor defines that instances must have the shape "S([]…)" with "…" being itself a Peano
  number.  Otherwise, binding and substitution are in the style of CRS higher order rewrite
  systems~\cite{Klop+:tcs1993}.\footnote{The notation does differ from the original CRS notation in
    that we use ``\#'' as a marker for meta-variables instead of the original reserved use of $Z$,
    and we use square brackets for substitution variables instead of round.}
\end{remark}

\begin{example}[untyped λ calculus]\label{ex:lambda}
  The untyped λ calculus is shown in Figure~\ref{fig:lambda} in \hax.  The declarations can be
  explained as follows:
  %%
  \begin{itemize}

  \item The "Λ" sort includes a special "variable" case to indicate that variables can occur in
    terms.

  \item "Λ" also includes a usual case for application construction, which is a "scheme" because it
    can (sometimes) be rewritten.

  \item "Λ" includes an abstraction construction, which is a "data" case, and which includes a
    subterm with a single binder scoped over that subterm. The scoped subterm is written as
    "[Λ[]]Λ[]", which should be read as ``a subterm with a locally bound variable of sort "Λ" and a
    body of sort "Λ" in which it can occur.''

  \item We specify one rewrite "rule" for "Λ": β-reduction. As usual, the rule specifies how an
    application of an abstraction is simplified. The interesting aspect of the pattern is how the
    abstraction is matched: the part of the pattern "[x]#body[x]" means ``the scoped subterm with
    binder "x" and subterm "#bind" wherein "x" may occur.'' Note the similarity to the declaration
    of the subterm of the "Lm" constructor.

  \item Once an application of an abstraction is matched, the "rule" gives the result of
    simplification as "#body[#arg[]]", which means that we construct a copy of "#body" except all
    occurrences of the variable we matched in the pattern "#body[x]" is substituted with what
    "#arg[]" matched.

  \end{itemize}
  %%
  A usual λ term like $(λx.x x)(λy.y)$ is represented as
  \begin{displaymath}
    "Ap([]Lm([x]Ap([]x, []x)), []Lm([y]y))"    
  \end{displaymath}
  and simplifies like this:
  %%
  \begin{displaymath}
    "Ap([]Lm([x]Ap([]x, []x)), []Lm([y]y))" →
    "Ap([]Lm([y]y), []Lm([y]y))" →
    "Lm([y]y))"
  \end{displaymath}
  %%
\end{example}

\begin{example}[lists]\label{ex:list}
  Figure~\ref{fig:list} shows a standard definition of polymorphic lists over an arbitrary element
  sort.
  %%
  \begin{itemize}

  \item The target sort is "List[α]" which is a usual polymorphic way to express a list of members
    of an unspecified parameter sort~"α".

  \item Inside the scope of the "sort" declaration, "α" denotes the member sort of the result list
    of all constructs. So the declaration of "data Cons([]α, []List[α])" means that the "Cons"
    constructor takes one argument of the same sort as the member of the result list, as well as an
    argument of the same sort as the result, "List[α]".

  \end{itemize}
  %%
\end{example}

\begin{example}[synthesis]\label{ex:attrib}
  Figure~\ref{fig:attrib} shows attributes…
\end{example}

In order to present more realistic examples, we first need to define some syntactic sugar to make
the notation less unwieldy.

\begin{notation}
  We will use \emph{vector notation} with $\ov{X}$ denoting $X_1,…,X_n$ for some $n≥0$; $ε$ denotes
  all empty vectors. We use $·$ (dot) for vector \emph{catenation}, including adding single new
  elements, and will freely abuse this notation and in particular write $\ov{[\ov{x}]x}$ as an
  abbreviation of $[x_{11}…x_{1m_1}]x_1…[x_{n1}…x_{nm_n}]x_n$ for suitable $n,m_1,…,m_n ≥ 0$.
\end{notation}

\begin{notation}
  %% 
  \HAX permits some syntactic sugar:
  %% 
  \begin{itemize}

  \item Empty parenthesis and brackets can be omitted.

  \item The \kw{data} and \kw{rule} keywords can be omitted.

  \item \TBD{Update to simplified syntax.} Sort cases ca be written with \kw{\texttt{|}} and without
    \kw{\texttt{\{\}}}…

  \item "attribute" declarations can be combined: $"attribute"~a_1~\set{AF}_1,…, a_n~\set{AF}_n ;$
    is an abbreviation of $"attribute"~a_1~\set{AF}_1;…; "attribute"~a_n~\set{AF}_n;$.

  \end{itemize}
\end{notation}

\begin{example}
  With syntactic sugar, we can simplify the Peano example to
  %%
  \begin{hacs}[numbers=right]
    sort N | Z | S(N) | scheme Plus(N, N) ;
    Plus(Z, S(#2)) → #2 ; 
    Plus(S(#1), #2) → S(Plus(#1, #2)) ;
  \end{hacs}
\end{example}

\begin{example}[untyped λ calculus]
  The untyped λ calculus is specified as follows in \hax with syntactic sugar:
  %%
  \begin{hacs}
    sort L | variable | Lm([x]Lam[x as L]) | scheme Ap(L, L) | x ;
    rule Ap(Lm([x]#body[x]), #arg) →  #body[#arg] ;
  \end{hacs}
  %%
  The example shows how binding is declared and used for substitution in the style of CRS
  systems~\cite{Klop+:tcs1993}:\footnote{The notation differs from the original CRS notation in that
    we use ``\#'' as a marker for meta-variables instead of the original reserved use of $Z$, and we
    use square brackets for substitution variables instead of round.}
  %%
  (Note that the same example in the (non-raw) full \HAX notation can include a parser for the
  native custom λ calculus syntax:
  %%
  \begin{hacs}
    token ID | [a-z] [a-z0-9_]* ;
    sort V | symbol ⟦⟨ID⟩⟧ ;
    sort L | ⟦λ⟨V binds x⟩.⟨L[x as L]⟩⟧ | scheme ⟦⟨L@1⟩⟨L@2⟩⟧@1
           | ⟦⟨V⟩⟧@2 | sugar ⟦(⟨L#⟩)⟧@2 →  #;
    ⟦ (λx.⟨L#body[x]⟩) ⟨L#arg⟩ ⟧ →  #body[#arg] ;
  \end{hacs}
  %%
  including precedence markers, syntactic sugar, \etc; for details see the full \HAX
  manual. Essentially the syntax-rich example translates to a parser and the former.)
\end{example}

%------------------------------------------------------------------------

\section{\HAX Typing}
\label{sec:typing}


\begin{definition}
  %%
  Given a sequence of declarations $\ov{D}$ and an attribute name~$a$, we define the following
  notions:
  %%
  \begin{enumerate}

  \item $\op{carriers}(\ov{D}, a)$ is the set of all constructors $c$ for which there is a
    declaration in $\ov{D}$ of the form ``$\kw{data}~c~…~a~…\kw;$'' or
    ``$\kw{scheme}~c~…~a~…\kw;$.''

  \item $\op{form}(\ov{D}, a)$ is the \set{AS} of the (single!) declaration
    ``$\kw{attribute}~a~\set{AS}\kw;$.''

  \end{enumerate}
  %%
  
  %%
\end{definition}



\begin{definition}[sort environment]
  Figure~\ref{fig:sortenv} gives the rules for construction of the sort environment, which has the
  following structure:
  \begin{align*}
    Γ &::= \bigl(\mathbb{C} ↦ α × (∀S^*.\,\set{BS}^*) × \set{DF}\bigr)
         + \bigl(\mathbb{V}↦(\set{AF}+ε, \mathbb{C}^*)\bigr)
    \tag{SortEnvironment}
    \\
    α &::= ∀\ov{v}.\,S
    \tag{SortScheme}
    \\
    \set{DF} &≡ \op{fun}(\mathbb{V}^*) + \op{data}
    \tag{DataOrFunction}
  \end{align*}
  We use usual map notation for $Γ$, with $Γ(x)$ denoting what $x$ maps to and $Γ[x↦X]$ standing for
  the map obtained from $Γ$ by adding (or changing) the binding of $x$ to map to~$X$.
\end{definition}






\begin{figure*}[p]
  \begin{gather*}
    \intertext{\shoveright{\hfil\fbox{$ Γ,S ⊢ \ov{D} ⇒ Γ $}}}\\
    %
    \tag{Sort1}
    \dfrac
    { Γ, S' ⊢ \ov{D'} ⇒ Γ' \qquad Γ', S ⊢ \ov{D} ⇒ Γ'' }
    { Γ, S ⊢ \kw{sort}~S' ~\kw\{\, \ov{D'} \,\kw\}\,\kw; ~ \ov{D} ⇒ Γ'' }
    \\[1ex]
    \tag{Attribute1}
    \qquad
    \dfrac
    { Γ, s[\ov{v}] ⊢ \ov{D} ⇒ Γ }
    { Γ, s[\ov{v}] ⊢ \kw{attribute}~a~\set{AS}; \, \ov{D} ⇒ Γ[a ↦ ⟨\set{AS}, \ov{s}⟩] }
    \quad Γ(a)=⟨ε,\ov{s}⟩
%
    \\[1ex]
    % 
    \tag{Data}
    \dfrac
    { Γ, s[\ov{v}] ⊢ \ov{D} }
    { Γ, s[\ov{v}] ⊢ \,\texttt{|}\; \kw{data}~c(\ov{[\ov{S}]S}) ; \ov{D} }
    \quad\Cases{
      Γ(c) = α\\
      \op{con}(α) = ∀\ov{v}.\bigl( \ov{[\ov{S}]S} → s[\ov{v}] \bigr)
    }
    \\[1ex]
    %
    \tag{Scheme}
    \dfrac
    { Γ, s[\ov{v}] ⊢ \ov{D} }
    { Γ, s[\ov{v}] ⊢ \,\texttt{|}\; \kw{scheme}~f(\ov{[\ov{S}]S})~{↓}a_1…{↓}a_n ; \ov{D} }
    \quad\Cases{
        Γ(f) = α\\
        \op{con}(α) = ∀\ov{v}.\bigl( \ov{[\ov{S}]S} → s[\ov{v}] \bigr)\\
        \op{atts}(α) = {↓}a_1…{↓}a_n
    }
%      ⇒ Γ[f ↦ ⟨∀\ov{v}\ov{v}.s, .s[\ov{v}], ∀\ov{v}.\ov{\set{BS}}, \op{fun}(\ov{a})⟩] }
    \\[1ex]
    \qquad
    \dfrac
    { Γ, s[\ov{v}] ⊢ \ov{D} ⇒ Γ }
    { Γ, s[\ov{v}] ⊢ \,\texttt{|}\; {↑}a; \ov{D} ⇒ Γ[a ↦ ⟨\set{AF}, \ov{s}·s⟩] }
    \quad Γ(a)=⟨\set{AF},\ov{s}⟩
    \tag{Synthesize1}
    \\[1ex]
    \dfrac
    { s[\ov{v}] ⊢ \ov{D} ⇒ Γ }
    { s[\ov{v}] ⊢ \kw{rule}~P→T; \, \ov{D} ⇒ Γ }
    \tag{Ignore1}
    \\[1ex]
    \dfrac
    { }
    { s[\ov{v}] ⊢ ε ⇒ \emptyset }
    \tag{Done1}
  \end{gather*}
  \caption{\HAX Sort Environment Construction (``Pass 1'').}
  \label{fig:sortenv}

  \begin{gather*}
    \hbox to\textwidth{\hfil\fbox{$Γ, S ⊢ \ov{D}$}}
    \\
    \dfrac
    { Γ ⊢ P : S ⇒ Δ
      \quad Γ, Δ ⊢ T : S \quad Γ, S ⊢ \ov{D} }
    { Γ, S ⊢ \kw{rule}~P→T; \, \ov{D} }
    \tag{Rule2}
    \\[1ex]
    \dfrac
    { Γ, S ⊢ \ov{D} }
    { Γ, S ⊢ \kw{sort}~s'[\ov{v'}] ; \ov{D} }
    \quad
    \dfrac
    { Γ, S ⊢ \ov{D} }
    { Γ, S ⊢ \,\texttt{|}\;C ; \ov{D} }
    \quad
    \dfrac
    { Γ, S ⊢ \ov{D} }
    { Γ, S ⊢ \kw{attribute}~a~\set{AF}; \, \ov{D} }
    \tag{Ignore2}
    \\[1ex]
    \dfrac
    { }
    { Γ, S ⊢ ε }
    \tag{Done2}
    \\[1ex]
    \hbox to\textwidth{\hfil\fbox{$Γ ⊢ P : α ⇒ Δ$}}
    \\
    \dfrac
    { ∀i\bigl[\, Γ[\ov{v_i}↦\ov{S_i}],Δ_{i-1} ⊢ P_i : S_i ⇒ Δ_i \,\bigr]
      \quad Γ,Δ_n ⊢ \ov{\set{PA}} : \ov{\set{AA}} ⇒ Δ }
    { Γ ⊢ c\bigl(\ov{[\ov{v}]P}\bigr)~\ov{\set{PA}} : ∀\ov{w}.S ⇒ Δ }
    \tag{PatternCons}
    \\[-3ex]
    \intertext{\shoveright{\hfil
    where $
    Δ_0 = \emptyset,
    Γ(c) = ⟨∀\ov{w}.S, ∀\ov{w}.\,\ov{[\ov{S}]S}, \set{DF}⟩,
    $\qquad}}
    %%
    \dfrac
    { Γ ⊢ P : s ⇒ Δ }
    { Γ ⊢ v~\ov{\set{PA}} : s ⇒ Δ }
    \tag{PatternVar}
    \\[1ex]
    \dfrac
    { Γ ⊢ P : s ⇒ Δ }
    { Γ ⊢ m\,[\ov{P}]~\ov{\set{PA}} : s ⇒ Δ }
    \tag{PatternMeta}
  \end{gather*}
  \caption{\HAX Sorting Rules (``Pass 2'').}
  \label{fig:sorting}




\end{figure*}

The intent with the $Γ$ structure is to map each defined symbol as follows:
%%
\begin{itemize}

\item Every constructor and function symbol is mapped to a triple
  $⟨s, ∀\ov{v}.\,\set{BS}, \set{DF}⟩$, which contains
  \begin{itemize}
  \item the \emph{sort} $s$ of the constructor;
  \item the \emph{parameter sort scheme} $∀\ov{v}.\,\set{BS}$ with the sorts of all parameters (more
    on that below);
  \item the \emph{constructor category} of the symbol (function with specific synthesized
    attributes, or data).
  \end{itemize}

\item Every attribute name is mapped to a pair $⟨\set{AF},\ov{s}⟩$, which contains
  \begin{itemize}
  \item the \emph{attribute form} that determines if the attribute is a simple value or a map;
  \item the \emph{synthesis sorts} of the attribute, with the sorts for which the attribute is
    synthesized on all data.
  \end{itemize}

\end{itemize}
%%
Constructors belong to a sort $s$, which may be a \emph{parametric sort}. This shows up in the sorts
of the subterms of the constructor, which are specified as $∀\ov{v}.\,\set{BS}$. Note that this 

\begin{definition}[sort check]
  Figure~\ref{fig:sorting} gives rules for sorting…
\end{definition}

\begin{figure*}[t]
\end{figure*}








%------------------------------------------------------------------------

\section{Conclusion}
\label{sec:conc}



We would like to give credit to SIS~\cite{Mosses:daimi1979}, which shares the use of
\emph{simplification} to a λ-calculus based formalism with \HAX, as illustrated, for example, by how
the interpretation rule can be encoded (as shown in the previous section).

The most prominent system that supports implementation of compilers in formal (attribute grammar)
form is ASF+SDF~\cite{Brand+:toplas2002}, which is based on first order rewriting. While modules
have been added for symbol table management, these lack the full integration and easy way to handle
scoped intermediate languages. The successor, Rascal~\cite{Bos+:eptcs2011} adds a module for HOAS,
but Rascal specifications operate in a world of side effects, which we find hard to reconcile with
higher-order term structures (with scopes).

The notion of ``higher-order'' used by \HAX is similar to but not quite the same as in higher-order
attribute grammars (HAG)~\cite{VogtSwierstraKuiper:pldi1989}. Like HAGs, \HAX specifications permit
constructing and passing of abstract syntax fragments in attributes -- indeed our example uses this
in the type synthesis rules -- but the ``higher order'' aspect of \HAX also covers the rewriting
side, where we can build parameterized abstractions over any part of a specification, including with
attributes. Indeed, one can use substitution inside attributes, have synthetic attributes block
schemes (as we have seen), have lack of substitution block synthesis (leave ``unsynthesizable
variables'' in a term), \etc

\HAX supports full HOAS, even with nested reductions under unevaluated binders, yet the applications
we consider are quite close to the examples presented for PHOAS by
Chlipala~\cite{Chlipala:icfp2008}. We are hoping to understand how many of the results of Chlipala
carry over to \HAX.

Finally, it is clear that the formal foundation of \HAX, while based in the solid culture of higher
order rewriting, still needs to be fully qualified and categorized in this world, so we can express
precisely (and hopefully also implement) standard tools such as termination checking for the
language. This is the subject of our upcoming EU-funded Marie Skłodowska-Curie ``HORIP'' action
collaboration with Cynthia Kop.

\paragraph*{Acknowledgements.} The author would like to thank my coteacher at NYU, Eva Rose, our
grader José Pablo Cambronero, as well as our students in the compiler construction class, for
constructive comments to and insightful questions on \HAX.
%%
The initial implementation of \HAX would not have been remotely possible without the CRSX team at
IBM: Morris Matsa, Scott Boag, and especially Lionel Villard, who suffered through understanding and
then programming both core fragments of and using the CRSX system that still provides the
implementation of \HAX.
%%
And finally \HAX owes its sanity to a collaboration with Cynthia Kop, both as an intern with the
Watson team in 2011, which created the polymorphic sort checker in her thesis work~\cite{Kop:2012},
and our continuing collaboration.


%------------------------------------------------------------------------

\bibliography{crs}


\appendix
\section{More Sugar}

\begin{notation}
  %% 
  \HAX permits some additional syntactic sugar:
  %% 
  \begin{itemize}

  \item An additional SortCase contruct is allowed:
    \begin{displaymath}
      \kw{template}~P~\kw{$→$}~T\,\kw;  \quad⇒\quad
      \kw{scheme}~P'\,\kw; ~ \kw{\texttt{|}} ~ \kw{data} ~ T'\,\kw; ~ \kw{rule} ~ P~\kw{$→$}~T\,\kw;
    \end{displaymath}
    where $P'$ and $T'$ are variants of $P$ and $P$ that have been cleared of "#"s.

  \item Finally, we rewrite special \emph{synthesis rules}:
    \begin{displaymath}
      c(\,\ov{\set{PB}}\,)~→~\ov{↑a~\set{V}}\,\kw;
      \quad⇒\quad
      \kw{rule}~c(\,\ov{\set{PB}}\,)~→~c(\,\ov{\set{PB}'}\,)~\ov{↑a~\set{V}}\,\kw;
    \end{displaymath}
    where $\set{PB}_i'$ is $\set{PB}_i$ with negative attribute patterns $\{¬{…}\}$ removed.

  \item We rewrite special \emph{inheritance rules}:
    \begin{displaymath}
      c(\,\ov{\set{PB}}\,)~→~\ov{↓v~\set{V}}\,\kw;
      \quad⇒\quad
      \kw{rule}~c(\,\ov{\set{PB'}}\,)~\ov{↓v~\set{V}}~→~c(\,\ov{\set{PB}'}\,)\,\kw;
    \end{displaymath}
    where 

  \item As defined below, every term $f(…)\ov{A}$ for a symbol $f$ defined with
    \begin{displaymath}
      \kw{\texttt{|} scheme}~f\,\kw(…\kw)~\ov{\set{AI}}
    \end{displaymath}
    must include an $A_i$ instance for every attribute defined by some $\set{AI}_i$.  The system
    automatically includes missing inherited attributes as follows:
    \begin{itemize}
    \item In patterns, a missing attribute is inserted as "↓a(#a)" or "↓a{:#a}" as appropriate.
    \item Missing attributes that were defined in the pattern are inserted the same.
    \item Remaining missing attributes must be of a map type and are inserted as~"↓a{}".
    \end{itemize}

  \item We allow two more forms of \set{PA}:
    \begin{displaymath}
      \set{PA} ::= {…} \bigm| \kw{$↑$}\,m \bigm| \kw{$↓$}\,m
    \end{displaymath}
    They expand to all relevant synthesized and inherited attributes, respectively.

  \end{itemize}
\end{notation}

\section{Inference Systems}
\label{sec:infer}

A common notation for specifying program analysis and …

A set of of inference rules is ...

\begin{definition}
  %%
  A \HAX \emph{ground sequential inference rule} has the form
  %%
  \begin{equation}
    \dfrac{ T_1 ⇒ P_1 \quad\cdots\quad T_n ⇒ P_n }{ P_0 ⇒ T_{n+1} }
    \label{eq:infer}
  \end{equation}
  %%
  where $n≥0$ and
  %%
  \begin{enumerate}
  \item $P_0$ (the \emph{initial} pattern) can be any \HAX pattern.
  \item $T_i$ (the \emph{tests}, $1≤i≤n$) are \HAX terms, and are allowed occurrences of meta-variables from $P_0…P_{i-1}$.
  \item $P_i$ (the \emph{constraints}, $i>0$) must be \HAX subpatterns.
  \item $T_{n+1}$ (the \emph{conclusion}) is allowed occurrences of all meta-variables.
  \end{enumerate}
  %%
  An inference rule with $n=0$ is called an \emph{axiom}, with $n>0$ a \emph{proper inference}.
  %%
  The $T_i⇒P_i$ over the line are called \emph{premise judgments}. The $P_0⇒T_{n+1}$ under the line
  is called the \emph{conclusion judgment}.
  %%
  Below we will vary the indexing scheme to capture various enumerations of sets of inference rules.
\end{definition}

\begin{definition}[pattern family]\label{def:patfam}
  Consider a set of ground sequential inference rules.  Such a set can obviously be indexed by the
  distinct initial patterns into families of rules
  \begin{displaymath}
    \dfrac{ T_{ij1} ⇒ P_{ij1} \cdots T_{ijn_{ij}} ⇒ P_{ijn_{ij}} }{ P_i ⇒ T_{ij} } ~(L_{ij})
  \end{displaymath}
  where all the $P_i$ used to index the family are pairwise distinct.  The groups obtained in this
  way are a \emph{pattern family}.
\end{definition}

\begin{definition}[leftmost matching]
  Consider a pattern family as in Definition~\ref{def:patfam}.  The family is said to be
  \emph{leftmost matching} if the group of rules for each $P_i$ satisfies one of the following
  conditions:
  \begin{itemize}
  \item either the family contains a single axiom, \ie, $1≤j≤1$ with $n_{ij}=0$,
  \item or the family contains only proper inferences, \ie, $1≤j≤m_i$ with $n_{ij}>0$.
  \end{itemize}
\end{definition}

%% Sub-index each non-axiom $P_i$-group by leftmost premise construction into sub-families of
%%    the form
%%    \begin{displaymath}
%%      \dfrac{ T_{ij} ⇒ P_{ijk1} \quad T_{ijk2} ⇒ P_{ijk2} \quad\cdots\quad T_{ijkn_{ijk}} ⇒ P_{ijkn_{ijk}} }{ P_i ⇒ T_{ijk} }
%%    \end{displaymath}
%%    where all the $T_{ij}$ used to index the sub-family within a $P_i$-group are pairwise disjoint.

\begin{definition}[left-operations]
  Given a leftmost-matching pattern family. Each $P$-indexed group of proper inference rules will
  have the following form:
  \begin{displaymath}
    \dfrac{ T_{j1} ⇒ P_{j1} \quad T_{j2} ⇒ P_{j2} \quad\cdots\quad T_{jn_j} ⇒ P_{jn_j} }{ P ⇒ T_j } ~ (L_j)
  \end{displaymath}
  with one $j$ per rule in the family, which is by definition non-empty ($1≤j≤n$) and contains only
  proper inferences ($n_j≥1$).
  %%
  \begin{enumerate}

  \item Given a label $L$, the \emph{left-flattening} rewrite rules of the group are
    %% 
    \begin{align*}
      L(P) &→ L'(T_{11}, …,T_{n1}, P) \\[\jot]
      L'(P_{11}, m_2, …, m_n, P) &→ L_1(P, P_{11}) \\[-\jot]
      &~\vdots\\
      L'(m_1, …, m_{n-1}, P_{n1}, P) &→ L_n(P, P_{n1})
    \end{align*}
    %% 
    with $L'$ a fresh symbol associated with the group and the $m_j$ fresh meta-variables.

  \item The \emph{left-eliminated} rules for the group are the rules
    \begin{displaymath}
      \dfrac{ T_{j2} ⇒ P_{j2} \quad\cdots\quad T_{jn_j} ⇒ P_{jn_j} }{ L_j(P, P_{j1}) ⇒ T_j } ~ (L'_j)
    \end{displaymath}
    with the $L'_j$ fresh labels derived from the $L_j$ labels.

  \end{enumerate}
\end{definition}

\begin{definition}[left-unfolded]
  Given a leftmost-matching pattern family and consider the proper inference rule group indexed by
  $P$. The following system is the \emph{left-unfolded} inference system for the $P$-indexed group:
  \begin{enumerate}

  \item The left-flattening rewrite rules of the group for a fresh~$L$.

  \item The new inference rule (which refers to the rewrite rules)
    \begin{displaymath}
      \dfrac{ L(P) → m \quad m ⇒ m' }{ P ⇒ m } ~ (L)
    \end{displaymath}

  \item The left-eliminated inference rules (which may be axioms or proper inferences).

  \end{enumerate}
\end{definition}

\begin{proposition}
  Given a set of ground sequential inference rules, which is a leftmost-matching pattern
  family. Pick one group of proper inference rules, indexed by the initial pattern $P$. The original
  system and the system where the group has been replaced with the left-unfolded group have the same
  normal forms.
\end{proposition}
\begin{proof}
  Easy: full proof tree before corresponds to proof tree after with simple conversion of the
  eliminated eliminated premise to an application of the new rule and a single use of the rewrite
  rule.
\end{proof}

Note that we can only prove that full ``big-step'' evaluations are equivalent: the new rules may
``get stuck'' in interesting ways (\TBD{example with overlapping patterns}).

\begin{lemma}
  Start with any leftmost matching ground sequential inference rule system and apply left-unfolding
  repeatedly except on axioms and the introduced inference rules
  \begin{displaymath}
    \dfrac{ L(P) → m \quad m ⇒ m' }{ P ⇒ m }
  \end{displaymath}
  The resulting system has the same normal forms as the original system.
\end{lemma}

To finish the transform from inference rules to \HAX rules we need two additional notions.

\begin{definition}
  An inference system is \emph{rooted} if it has one rule that occurs as the root rule of every
  proof tree.
\end{definition}

\begin{definition}
  A leftmost-matching ground sequential inference rule system is \emph{left deterministic} if the
  left-flattening 
\end{definition}

\begin{theorem}
  A rooted and leftmost matching ground sequential inference system can be implemented by a 
\end{theorem}



\end{document}


%------------------------------------------------------------------------
% Tell Emacs that this is a LaTeX document and how it is formatted:
% Local Variables:
% mode:latex
% fill-column:100
% TeX-master: t
% TeX-auto-untabify: nil
% End:
